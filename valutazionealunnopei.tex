 un interessante passo della nota Prot. n. 1075/C27 dell’USR della Liguria del 21.2.2011 che ha per oggetto “La continuità educativa a favore degli alunni disabili”:
“…Nel caso di alunni con esigenze educative particolari, si ricorda che nulla vieta che il PEI possa prevedere un percorso fortemente individualizzato, senza che questo comporti la necessità di
rallentare o posticipare l'avvio del percorso scolastico. Analoga attenzione deve essere posta alla regolarità e fluidità del percorso scolastico, che deve consentire, anche agli alunni disabili, di poter usufruire di tutte le opportunità che il sistema scolastico e formativo offrono. Con ciò non si esclude la possibilità di ripetenza, ma pare opportuno ricordare che la promozione o meno dell'alunno, sia pure disabile, è competenza esclusiva degli organi collegiali nella sola componente docente.
L'alunno sarà valutato in riferimento non ad obiettivi standard, ma agli obiettivi didattici previsti espressamente per lui nel PEI. Non si ritiene che l'alunno possa essere respinto qualora nella definizione degli obiettivi del PEI siano state fissate mete non raggiungibili per l'alunno stesso.
La valutazione, e quindi l'esito scolastico, non può essere condizionato da considerazioni e pregiudizi rispetto all'idoneità o meno della struttura di futura frequenza.”