\chapter{Legge 28 marzo 2003, n. 53}
\label{Legge2003_53}


"Delega al Governo per la definizione delle norme generali sull'istruzione e dei livelli essenziali delle prestazioni in materia di istruzione e formazione professionale"

pubblicata nella Gazzetta Ufficiale n. 77 del 2 Aprile 2003

Art. 1.

(Delega in materia di norme generali sull'istruzione e di livelli essenziali delle prestazioni in materia di istruzione e di formazione professionale)

1. Al fine di favorire la crescita e la valorizzazione della persona umana, nel rispetto dei ritmi dell'età evolutiva, delle differenze e dell'identità di ciascuno e delle scelte educative della famiglia, nel quadro della cooperazione tra scuola e genitori, in coerenza con il principio di autonomia delle istituzioni scolastiche e secondo i principi sanciti dalla Costituzione, il Governo è delegato ad adottare, entro ventiquattro mesi dalla data di entrata in vigore della presente legge, nel rispetto delle competenze costituzionali delle regioni e di comuni e province, in relazione alle competenze conferite ai diversi soggetti istituzionali, e dell'autonomia delle istituzioni scolastiche, uno o più decreti legislativi per la definizione delle norme generali sull'istruzione e dei livelli essenziali delle prestazioni in materia di istruzione e di istruzione e formazione professionale.

2. Fatto salvo quanto specificamente previsto dall'articolo 4, i decreti legislativi di cui al comma 1 sono adottati su proposta del Ministro dell'istruzione, dell'università e della ricerca, di concerto con il Ministro dell'economia e delle finanze, con il Ministro per la funzione pubblica e con il Ministro del lavoro e delle politiche sociali, sentita la Conferenza unificata di cui all'articolo 8 del decreto legislativo 28 agosto 1997, n. 281, e previo parere delle competenti Commissioni della Camera dei deputati e del Senato della Repubblica da rendere entro sessanta giorni dalla data di trasmissione dei relativi schemi; decorso tale termine, i decreti legislativi possono essere comunque adottati. I decreti legislativi in materia di istruzione e formazione professionale sono adottati previa intesa con la Conferenza unificata di cui al citato decreto legislativo n. 281 del 1997.

3. Per la realizzazione delle finalità della presente legge, il Ministro dell'istruzione, dell'università e della ricerca predispone, entro novanta giorni dalla data di entrata in vigore della legge medesima, un piano programmatico di interventi finanziari, da sottoporre all'approvazione del Consiglio dei ministri, previa intesa con la Conferenza unificata di cui al citato decreto legislativo n. 281 del 1997, a sostegno:

a) della riforma degli ordinamenti e degli interventi connessi con la loro attuazione e con lo sviluppo e la valorizzazione dell'autonomia delle istituzioni scolastiche;

b) dell'istituzione del Servizio nazionale di valutazione del sistema scolastico;

c) dello sviluppo delle tecnologie multimediali e della alfabetizzazione nelle tecnologie informatiche, nel pieno rispetto del principio di pluralismo delle soluzioni informatiche offerte dall'informazione tecnologica, al fine di incoraggiare e sviluppare le doti creative e collaborative degli studenti;

d) dello sviluppo dell'attività motoria e delle competenze ludico-sportive degli studenti;

e) della valorizzazione professionale del personale docente;

f) delle iniziative di formazione iniziale e continua del personale;

g) del concorso al rimborso delle spese di auto aggiornamento sostenute dai docenti;

h) della valorizzazione professionale del personale amministrativo, tecnico ed ausiliario (ATA);

i) degli interventi di orientamento contro la dispersione scolastica e per assicurare la realizzazione del diritto - dovere di istruzione e formazione;

l) degli interventi per lo sviluppo dell'istruzione e formazione tecnica superiore e per l'educazione degli adulti;

m) degli interventi di adeguamento delle strutture di edilizia scolastica.

4. Ulteriori disposizioni, correttive e integrative dei decreti legislativi di cui al presente articolo e all'articolo 4, possono essere adottate, con il rispetto dei medesimi criteri e principi direttivi e con le stesse procedure, entro diciotto mesi dalla data della loro entrata in vigore.

Art. 2.

(Sistema educativo di istruzione e di formazione)

1. I decreti di cui all'articolo 1 definiscono il sistema educativo di istruzione e di formazione, con l'osservanza dei seguenti principi e criteri direttivi:

a) è promosso l'apprendimento in tutto l'arco della vita e sono assicurate a tutti pari opportunità di raggiungere elevati livelli culturali e di sviluppare le capacità e le competenze, attraverso conoscenze e abilità, generali e specifiche, coerenti con le attitudini e le scelte personali, adeguate all'inserimento nella vita sociale e nel mondo del lavoro, anche con riguardo alle dimensioni locali, nazionale ed europea;

b) sono promossi il conseguimento di una formazione spirituale e morale, anche ispirata ai principi della Costituzione, e lo sviluppo della coscienza storica e di appartenenza alla comunità locale, alla comunità nazionale ed alla civiltà europea;

c) è assicurato a tutti il diritto all'istruzione e alla formazione per almeno dodici anni o, comunque, sino al conseguimento di una qualifica entro il diciottesimo anno di età; l'attuazione di tale diritto si realizza nel sistema di istruzione e in quello di istruzione e formazione professionale, secondo livelli essenziali di prestazione definiti su base nazionale a norma dell'articolo 117, secondo comma, lettera m), della Costituzione e mediante regolamenti emanati ai sensi dell'articolo 17, comma 2, della legge 23 agosto 1988, n. 400, e garantendo, attraverso adeguati interventi, l'integrazione delle persone in situazione di handicap a norma della legge 5 febbraio 1992, n. 104. La fruizione dell'offerta di istruzione e formazione costituisce un dovere legislativamente sanzionato; nei termini anzidetti di diritto all'istruzione e formazione e di correlativo dovere viene ridefinito ed ampliato l'obbligo scolastico di cui all'articolo 34 della Costituzione, nonché l'obbligo formativo introdotto dall'articolo 68 della legge 17 maggio 1999, n. 144, e successive modificazioni. L'attuazione graduale del diritto-dovere predetto è rimessa ai decreti legislativi di cui all'articolo 1, commi 1 e 2, della presente legge correlativamente agli interventi finanziari previsti a tale fine dal piano programmatico di cui all'articolo 1, comma 3, adottato previa intesa con la Conferenza unificata di cui all'articolo 8 del decreto legislativo 28 agosto 1997, n. 281, e coerentemente con i finanziamenti disposti a norma dell'articolo 7, comma 6, della presente legge;

d) il sistema educativo di istruzione e di formazione si articola nella scuola dell'infanzia, in un primo ciclo che comprende la scuola primaria e la scuola secondaria di primo grado, e in un secondo ciclo che comprende il sistema dei licei ed il sistema dell'istruzione e della formazione professionale;

e) la scuola dell'infanzia, di durata triennale, concorre all'educazione e allo sviluppo affettivo, psicomotorio, cognitivo, morale, religioso e sociale delle bambine e dei bambini promuovendone le potenzialità di relazione, autonomia, creatività, apprendimento, e ad assicurare un'effettiva eguaglianza delle opportunità educative; nel rispetto della primaria responsabilità educativa dei genitori, essa contribuisce alla formazione integrale delle bambine e dei bambini e, nella sua autonomia e unitarietà didattica e pedagogica, realizza la continuità educativa con il complesso dei servizi all'infanzia e con la scuola primaria. È assicurata la generalizzazione dell'offerta formativa e la possibilità di frequenza della scuola dell'infanzia; alla scuola dell'infanzia possono essere iscritti secondo criteri di gradualità e in forma di sperimentazione le bambine e i bambini che compiono i 3 anni di età entro il 30 aprile dell'anno scolastico di riferimento, anche in rapporto all'introduzione di nuove professionalità e modalità organizzative;

f) il primo ciclo di istruzione è costituito dalla scuola primaria, della durata di cinque anni, e dalla scuola secondaria di primo grado della durata di tre anni. Ferma restando la specificità di ciascuna di esse, la scuola primaria è articolata in un primo anno, teso al raggiungimento delle strumentalità di base, e in due periodi didattici biennali; la scuola secondaria di primo grado si articola in un biennio e in un terzo anno che completa prioritariamente il percorso disciplinare ed assicura l'orientamento ed il raccordo con il secondo ciclo; nel primo ciclo è assicurato altresì il raccordo con la scuola dell'infanzia e con il secondo ciclo; è previsto che alla scuola primaria si iscrivano le bambine e i bambini che compiono i sei anni di età entro il 31 agosto; possono iscriversi anche le bambine e i bambini che li compiono entro il 30 aprile dell'anno scolastico di riferimento; la scuola primaria promuove, nel rispetto delle diversità individuali, lo sviluppo della personalità, ed ha il fine di far acquisire e sviluppare le conoscenze e le abilità di base fino alle prime sistemazioni logico-critiche, di far apprendere i mezzi espressivi, ivi inclusa l'alfabetizzazione in almeno una lingua dell'Unione europea oltre alla lingua italiana, di porre le basi per l'utilizzazione di metodologie scientifiche nello studio del mondo naturale, dei suoi fenomeni e delle sue leggi, di valorizzare le capacità relazionali e di orientamento nello spazio e nel tempo, di educare ai principi fondamentali della convivenza civile; la scuola secondaria di primo grado, attraverso le discipline di studio, è finalizzata alla crescita delle capacità autonome di studio ed al rafforzamento delle attitudini alla interazione sociale; organizza ed accresce, anche attraverso l'alfabetizzazione e l'approfondimento nelle tecnologie informatiche, le conoscenze e le abilità, anche in relazione alla tradizione culturale e alla evoluzione sociale, culturale e scientifica della realtà contemporanea; è caratterizzata dalla diversificazione didattica e metodologica in relazione allo sviluppo della personalità dell'allievo; cura la dimensione sistematica delle discipline; sviluppa progressivamente le competenze e le capacità di scelta corrispondenti alle attitudini e vocazioni degli allievi; fornisce strumenti adeguati alla prosecuzione delle attività di istruzione e di formazione; introduce lo studio di una seconda lingua dell'Unione europea; aiuta ad orientarsi per la successiva scelta di istruzione e formazione; il primo ciclo di istruzione si conclude con un esame di Stato, il cui superamento costituisce titolo di accesso al sistema dei licei e al sistema dell'istruzione e della formazione professionale;

g) il secondo ciclo, finalizzato alla crescita educativa, culturale e professionale dei giovani attraverso il sapere, il fare e l'agire, e la riflessione critica su di essi, è finalizzato a sviluppare l'autonoma capacità di giudizio e l'esercizio della responsabilità personale e sociale; in tale ambito, viene anche curato lo sviluppo delle conoscenze relative all'uso delle nuove tecnologie; il secondo ciclo è costituito dal sistema dei licei e dal sistema dell'istruzione e della formazione professionale; dal compimento del quindicesimo anno di età i diplomi e le qualifiche si possono conseguire in alternanza scuola-lavoro o attraverso l'apprendistato; il sistema dei licei comprende i licei artistico, classico, economico, linguistico, musicale e coreutico, scientifico, tecnologico, delle scienze umane; i licei artistico, economico e tecnologico si articolano in indirizzi per corrispondere ai diversi fabbisogni formativi; i licei hanno durata quinquennale; l'attività didattica si sviluppa in due periodi biennali e in un quinto anno che prioritariamente completa il percorso disciplinare e prevede altresì l'approfondimento delle conoscenze e delle abilità caratterizzanti il profilo educativo, culturale e professionale del corso di studi; i licei si concludono con un esame di Stato il cui superamento rappresenta titolo necessario per l'accesso all'università e all'alta formazione artistica, musicale e coreutica; l'ammissione al quinto anno dà accesso all'istruzione e formazione tecnica superiore;

h) ferma restando la competenza regionale in materia di formazione e istruzione professionale, i percorsi del sistema dell'istruzione e della formazione professionale realizzano profili educativi, culturali e professionali, ai quali conseguono titoli e qualifiche professionali di differente livello, valevoli su tutto il territorio nazionale se rispondenti ai livelli essenziali di prestazione di cui alla lettera c); le modalità di accertamento di tale rispondenza, anche ai fini della spendibilità dei predetti titoli e qualifiche nell'Unione europea, sono definite con il regolamento di cui all'articolo 7, comma 1, lettera c); i titoli e le qualifiche costituiscono condizione per l'accesso all'istruzione e formazione tecnica superiore, fatto salvo quanto previsto dall'articolo 69 della legge 17 maggio 1999, n. 144; i titoli e le qualifiche conseguiti al termine dei percorsi del sistema dell'istruzione e della formazione professionale di durata almeno quadriennale consentono di sostenere l'esame di Stato, utile anche ai fini degli accessi all'università e all'alta formazione artistica, musicale e coreutica, previa frequenza di apposito corso annuale, realizzato d'intesa con le università e con l'alta formazione artistica, musicale e coreutica, e ferma restando la possibilità di sostenere, come privatista, l'esame di Stato anche senza tale frequenza;

i) è assicurata e assistita la possibilità di cambiare indirizzo all'interno del sistema dei licei, nonché di passare dal sistema dei licei al sistema dell'istruzione e della formazione professionale, e viceversa, mediante apposite iniziative didattiche, finalizzate all'acquisizione di una preparazione adeguata alla nuova scelta; la frequenza positiva di qualsiasi segmento del secondo ciclo comporta l'acquisizione di crediti certificati che possono essere fatti valere, anche ai fini della ripresa degli studi eventualmente interrotti, nei passaggi tra i diversi percorsi di cui alle lettere g) e h); nel secondo ciclo, esercitazioni pratiche, esperienze formative e stage realizzati in Italia o all'estero anche con periodi di inserimento nelle realtà culturali, sociali, produttive, professionali e dei servizi, sono riconosciuti con specifiche certificazioni di competenza rilasciate dalle istituzioni scolastiche e formative; i licei e le istituzioni formative del sistema dell'istruzione e della formazione professionale, d'intesa rispettivamente con le università, con le istituzioni dell'alta formazione artistica, musicale e coreutica e con il sistema dell'istruzione e formazione tecnica superiore, stabiliscono, con riferimento all'ultimo anno del percorso di studi, specifiche modalità per l'approfondimento delle conoscenze e delle abilità richieste per l'accesso ai corsi di studio universitari, dell'alta formazione, ed ai percorsi dell'istruzione e formazione tecnica superiore;

l) i piani di studio personalizzati, nel rispetto dell'autonomia delle istituzioni scolastiche, contengono un nucleo fondamentale, omogeneo su base nazionale, che rispecchia la cultura, le tradizioni e l'identità nazionale, e prevedono una quota, riservata alle regioni, relativa agli aspetti di interesse specifico delle stesse, anche collegata con le realtà locali.

Art. 3.

(Valutazione degli apprendimenti e della qualità del sistema educativo di istruzione e di formazione)

1. Con i decreti di cui all'articolo 1 sono dettate le norme generali sulla valutazione del sistema educativo di istruzione e di formazione e degli apprendimenti degli studenti, con l'osservanza dei seguenti principi e criteri direttivi:

a) la valutazione, periodica e annuale, degli apprendimenti e del comportamento degli studenti del sistema educativo di istruzione e di formazione, e la certificazione delle competenze da essi acquisite, sono affidate ai docenti delle istituzioni di istruzione e formazione frequentate; agli stessi docenti è affidata la valutazione dei periodi didattici ai fini del passaggio al periodo successivo; il miglioramento dei processi di apprendimento e della relativa valutazione, nonché la continuità didattica, sono assicurati anche attraverso una congrua permanenza dei docenti nella sede di titolarità;

b) ai fini del progressivo miglioramento e dell'armonizzazione della qualità del sistema di istruzione e di formazione, l'Istituto nazionale per la valutazione del sistema di istruzione effettua verifiche periodiche e sistematiche sulle conoscenze e abilità degli studenti e sulla qualità complessiva dell'offerta formativa delle istituzioni scolastiche e formative; in funzione dei predetti compiti vengono rideterminate le funzioni e la struttura del predetto Istituto;

c) l'esame di Stato conclusivo dei cicli di istruzione considera e valuta le competenze acquisite dagli studenti nel corso e al termine del ciclo e si svolge su prove organizzate dalle commissioni d'esame e su prove predisposte e gestite dall'Istituto nazionale per la valutazione del sistema di istruzione, sulla base degli obiettivi specifici di apprendimento del corso ed in relazione alle discipline di insegnamento dell'ultimo anno.

Art. 4.

(Alternanza scuola-lavoro)

1. Fermo restando quanto previsto dall'articolo 18 della legge 24 giugno 1997, n. 196, al fine di assicurare agli studenti che hanno compiuto il quindicesimo anno di età la possibilità di realizzare i corsi del secondo ciclo in alternanza scuola-lavoro, come modalità di realizzazione del percorso formativo progettata, attuata e valutata dall'istituzione scolastica e formativa in collaborazione con le imprese, con le rispettive associazioni di rappresentanza e con le camere di commercio, industria, artigianato e agricoltura, che assicuri ai giovani, oltre alla conoscenza di base, l'acquisizione di competenze spendibili nel mercato del lavoro, il Governo è delegato ad adottare, entro il termine di ventiquattro mesi dalla data di entrata in vigore della presente legge e ai sensi dell'articolo 1, commi 2 e 3, della legge stessa, un apposito decreto legislativo su proposta del Ministro dell'istruzione, dell'università e della ricerca, di concerto con il Ministro del lavoro e delle politiche sociali e con il Ministro delle attività produttive, d'intesa con la Conferenza unificata di cui all'articolo 8 del decreto legislativo 28 agosto 1997, n. 281, sentite le associazioni maggiormente rappresentative dei datori di lavoro, nel rispetto dei seguenti princìpi e criteri direttivi:

a) svolgere l'intera formazione dai 15 ai 18 anni, attraverso l'alternanza di periodi di studio e di lavoro, sotto la responsabilità dell'istituzione scolastica o formativa, sulla base di convenzioni con imprese o con le rispettive associazioni di rappresentanza o con le camere di commercio, industria, artigianato e agricoltura, o con enti pubblici e privati ivi inclusi quelli del terzo settore, disponibili ad accogliere gli studenti per periodi di tirocinio che non costituiscono rapporto individuale di lavoro. Le istituzioni scolastiche, nell'ambito dell'alternanza scuola-lavoro, possono collegarsi con il sistema dell'istruzione e della formazione professionale ed assicurare, a domanda degli interessati e d'intesa con le regioni, la frequenza negli istituti d'istruzione e formazione professionale di corsi integrati che prevedano piani di studio progettati d'intesa fra i due sistemi, coerenti con il corso di studi e realizzati con il concorso degli operatori di ambedue i sistemi;

b) fornire indicazioni generali per il reperimento e l'assegnazione delle risorse finanziarie necessarie alla realizzazione dei percorsi di alternanza, ivi compresi gli incentivi per le imprese, la valorizzazione delle imprese come luogo formativo e l'assistenza tutoriale;

c) indicare le modalità di certificazione dell'esito positivo del tirocinio e di valutazione dei crediti formativi acquisiti dallo studente.

2. I compiti svolti dal docente incaricato dei rapporti con le imprese e del monitoraggio degli allievi che si avvalgono dell'alternanza scuola-lavoro sono riconosciuti nel quadro della valorizzazione della professionalità del personale docente.

Art. 5.

(Formazione degli insegnanti)

1. Con i decreti di cui all'articolo 1 sono dettate norme sulla formazione iniziale dei docenti della scuola dell'infanzia, del primo ciclo e del secondo ciclo, nel rispetto dei seguenti principi e criteri direttivi:

a) la formazione iniziale è di pari dignità per tutti i docenti e si svolge nelle università presso i corsi di laurea specialistica, il cui accesso è programmato ai sensi dell'articolo 1, comma 1, della legge 2 agosto 1999, n. 264, e successive modificazioni. La programmazione degli accessi ai corsi stessi è determinata ai sensi dell'articolo 3 della medesima legge, sulla base della previsione dei posti effettivamente disponibili, per ogni ambito regionale, nelle istituzioni scolastiche;

b) con uno o più decreti, adottati ai sensi dell'articolo 17, comma 95, della legge 15 maggio 1997, n. 127, anche in deroga alle disposizioni di cui all'articolo 10, comma 2, e all'articolo 6, comma 4, del regolamento di cui al decreto del Ministro dell'università e della ricerca scientifica e tecnologica 3 novembre 1999, n. 509, sono individuate le classi dei corsi di laurea specialistica, anche interfacoltà o interuniversitari, finalizzati anche alla formazione degli insegnanti di cui alla lettera a) del presente comma. Per la formazione degli insegnanti della scuola secondaria di primo grado e del secondo ciclo le classi predette sono individuate con riferimento all'insegnamento delle discipline impartite in tali gradi di istruzione e con preminenti finalità di approfondimento disciplinare. I decreti stessi disciplinano le attività didattiche attinenti l'integrazione scolastica degli alunni in condizione di handicap; la formazione iniziale dei docenti può prevedere stage all'estero;

c) l'accesso ai corsi di laurea specialistica per la formazione degli insegnanti è subordinato al possesso dei requisiti minimi curricolari, individuati per ciascuna classe di abilitazione nel decreto di cui alla lettera b) e all'adeguatezza della personale preparazione dei candidati, verificata dagli atenei;

d) l'esame finale per il conseguimento della laurea specialistica di cui alla lettera a) ha valore abilitante per uno o più insegnamenti individuati con decreto del Ministro dell'istruzione, dell'università e della ricerca;

e) coloro che hanno conseguito la laurea specialistica di cui alla lettera a), ai fini dell'accesso nei ruoli organici del personale docente delle istituzioni scolastiche, svolgono, previa stipula di appositi contratti di formazione lavoro, specifiche attività di tirocinio. A tale fine e per la gestione dei corsi di cui alla lettera a), le università, sentita la direzione scolastica regionale, definiscono nei regolamenti didattici di ateneo l'istituzione e l'organizzazione di apposite strutture di ateneo o d'interateneo per la formazione degli insegnanti, cui sono affidati, sulla base di convenzioni, anche i rapporti con le istituzioni scolastiche;

f) le strutture didattiche di ateneo o d'interateneo di cui alla lettera e) promuovono e governano i centri di eccellenza per la formazione permanente degli insegnanti, definiti con apposito decreto del Ministro dell'istruzione, dell'università e della ricerca;

g) le strutture di cui alla lettera e) curano anche la formazione in servizio degli insegnanti interessati ad assumere funzioni di supporto, di tutorato e di coordinamento dell'attività educativa, didattica e gestionale delle istituzioni scolastiche e formative.

2. Con i decreti di cui all'articolo 1 sono dettate norme anche sulla formazione iniziale svolta negli istituti di alta formazione e specializzazione artistica, musicale e coreutica di cui alla legge 21 dicembre 1999, n. 508, relativamente agli insegnamenti cui danno accesso i relativi diplomi accademici. Ai predetti fini si applicano, con i necessari adattamenti, i principi e criteri direttivi di cui al comma 1 del presente articolo.

3. Per coloro che, sprovvisti dell'abilitazione all'insegnamento secondario, sono in possesso del diploma biennale di specializzazione per le attività di sostegno di cui al decreto del Ministro della pubblica istruzione 24 novembre 1998, pubblicato nella Gazzetta Ufficiale n. 131 del 7 giugno 1999, e al decreto del Presidente della Repubblica 31 ottobre 1975, n. 970, nonché del diploma di laurea o del diploma di istituto superiore di educazione fisica (ISEF) o di Accademia di belle arti o di Istituto superiore per le industrie artistiche o di Conservatorio di musica o Istituto musicale pareggiato, e che abbiano superato le prove di accesso alle scuole di specializzazione all'insegnamento secondario, le scuole medesime valutano il percorso didattico teorico-pratico e gli esami sostenuti per il conseguimento del predetto diploma di specializzazione ai fini del riconoscimento dei relativi crediti didattici, anche per consentire loro un'abbreviazione del percorso degli studi della scuola di specializzazione previa iscrizione in soprannumero al secondo anno di corso della scuola. I corsi di laurea in scienze della formazione primaria di cui all'articolo 3, comma 2, della legge 19 novembre 1990, n. 341, valutano il percorso didattico teorico-pratico e gli esami sostenuti per il conseguimento del diploma biennale di specializzazione per le attività di sostegno ai fini del riconoscimento dei relativi crediti didattici e dell'iscrizione in soprannumero al relativo anno di corso stabilito dalle autorità accademiche, per coloro che, in possesso di tale titolo di specializzazione e del diploma di scuola secondaria superiore, abbiano superato le relative prove di accesso. L'esame di laurea sostenuto a conclusione dei corsi in scienze della formazione primaria istituiti a norma dell'articolo 3, comma 2, della legge 19 novembre 1990, n. 341, comprensivo della valutazione delle attività di tirocinio previste dal relativo percorso formativo, ha valore di esame di Stato e abilita all'insegnamento, rispettivamente, nella scuola materna o dell'infanzia e nella scuola elementare o primaria. Esso consente altresì l'inserimento nelle graduatorie permanenti previste dall'articolo 401 del testo unico di cui al decreto legislativo 16 aprile 1994, n. 297, e successive modificazioni. Al fine di tale inserimento, la tabella di valutazione dei titoli è integrata con la previsione di un apposito punteggio da attribuire al voto di laurea conseguito. All'articolo 3, comma 2, della legge 19 novembre 1990, n. 341, le parole: «I concorsi hanno funzione abilitante» sono soppresse.

Art. 6.

(Regioni a statuto speciale e province autonome di Trento e di Bolzano)

1. Sono fatte salve le competenze delle regioni a statuto speciale e delle province autonome di Trento e di Bolzano, in conformità ai rispettivi statuti e relative norme di attuazione, nonché alla legge costituzionale 18 ottobre 2001, n.3.

Art. 7.

(Disposizioni finali e attuative)

1. Mediante uno o più regolamenti da adottare a norma dell'articolo 117, sesto comma, della Costituzione e dell'articolo 17, comma 2, della legge 23 agosto 1988, n. 400, sentite le Commissioni parlamentari competenti, nel rispetto dell'autonomia delle istituzioni scolastiche, si provvede:

a) alla individuazione del nucleo essenziale dei piani di studio scolastici per la quota nazionale relativamente agli obiettivi specifici di apprendimento, alle discipline e alle attività costituenti la quota nazionale dei piani di studio, agli orari, ai limiti di flessibilità interni nell'organizzazione delle discipline;

b) alla determinazione delle modalità di valutazione dei crediti scolastici;

c) alla definizione degli standard minimi formativi, richiesti per la spendibilità nazionale dei titoli professionali conseguiti all'esito dei percorsi formativi, nonché per i passaggi dai percorsi formativi ai percorsi scolastici.

2. Le norme regolamentari di cui al comma 1, lettera c), sono definite previa intesa con la Conferenza permanente per i rapporti tra lo Stato, le regioni e le province autonome di Trento e di Bolzano, di cui al decreto legislativo 28 agosto 1997, n. 281.

3. Il Ministro dell'istruzione, dell'università e della ricerca presenta ogni tre anni al Parlamento una relazione sul sistema educativo di istruzione e di formazione professionale.

4. Per gli anni scolastici 2003-2004, 2004-2005 e 2005-2006 possono iscriversi, secondo criteri di gradualità e in forma di sperimentazione, compatibilmente con la disponibilità dei posti e delle risorse finanziarie dei comuni, secondo gli obblighi conferiti dall'ordinamento e nel rispetto dei limiti posti alla finanza comunale dal patto di stabilità, al primo anno della scuola dell'infanzia i bambini e le bambine che compiono i tre anni di età entro il 28 febbraio 2004, ovvero entro date ulteriormente anticipate, fino alla data del 30 aprile di cui all'articolo 2, comma 1, lettera e). Per l'anno scolastico 2003-2004 possono iscriversi al primo anno della scuola primaria, nei limiti delle risorse finanziarie di cui al comma 5, i bambini e le bambine che compiono i sei anni di età entro il 28 febbraio 2004.

5. Agli oneri derivanti dall'attuazione dell'articolo 2, comma 1, lettera f), e dal comma 4 del presente articolo, limitatamente alla scuola dell'infanzia statale e alla scuola primaria statale, determinati nella misura massima di 12.731 migliaia di euro per l'anno 2003, 45.829 migliaia di euro per l'anno 2004 e 66.198 migliaia di euro a decorrere dall'anno 2005, si provvede mediante corrispondente riduzione dello stanziamento iscritto, ai fini del bilancio triennale 2003-2005, nell'ambito dell'unità previsionale di base di parte corrente «Fondo speciale» dello stato di previsione del Ministero dell'economia e delle finanze per l'anno 2003, allo scopo parzialmente utilizzando l'accantonamento relativo al Ministero dell'istruzione, dell'università e della ricerca. Il Ministro dell'istruzione, dell'università e della ricerca provvede a modulare le anticipazioni, anche fino alla data del 30 aprile di cui all'articolo 2, comma 1, lettera f), garantendo comunque il rispetto del predetto limite di spesa.

6. All'attuazione del piano programmatico di cui all'articolo 1, comma 3, si provvede, compatibilmente con i vincoli di finanza pubblica, mediante finanziamenti da iscrivere annualmente nella legge finanziaria, in coerenza con quanto previsto dal Documento di programmazione economico-finanziaria.

7. Lo schema di ciascuno dei decreti legislativi di cui agli articoli 1 e 4 deve essere corredato da relazione tecnica ai sensi dell'articolo 11-ter, comma 2, della legge 5 agosto 1978, n. 468, e successive modificazioni.

8. I decreti legislativi di cui al comma 7 la cui attuazione determini nuovi o maggiori oneri per la finanza pubblica sono emanati solo successivamente all'entrata in vigore di provvedimenti legislativi che stanzino le occorrenti risorse finanziarie.

9. Il parere di cui all'articolo 1, comma 2, primo periodo, è espresso dalle Commissioni parlamentari competenti per materia e per le conseguenze di carattere finanziario.

10. Con periodicità annuale, il Ministero dell'istruzione, dell'università e della ricerca ed il Ministero dell'economia e delle finanze procedono alla verifica delle occorrenze finanziarie, in relazione alla graduale attuazione della riforma, a fronte delle somme stanziate annualmente in bilancio per lo stesso fine. Le eventuali maggiori spese dovranno trovare copertura ai sensi dell'articolo 11-ter, comma 7, della legge 5 agosto 1978, n. 468, e successive modificazioni.

11. Il Ministro dell'economia e delle finanze è autorizzato ad apportare, con propri decreti, le occorrenti variazioni di bilancio.

12. La legge 10 febbraio 2000, n. 30, è abrogata.

13. La legge 20 gennaio 1999, n. 9, è abrogata.