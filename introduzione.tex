\chapter[BES]{Bisogni\xheadbreak{}
	Educativi\xheadbreak{}Speciali}
\label{BES}
\section{Introduzione}
\label{sec:introduzione}
Dalla lettura di quello che segue, apparirà evidente che questo non è un saggio su i bisogni educativi speciali, ma una raccolta, sicuramente parziale, di norme che riguardano l'argomento. Manca completamente la parte sulle strutture di supporto che sarà aggiunta in seguito in seguito. Errori, sviste sicuramente ve ne sono e me ne scuso. Il lavoro non è stato fatto per essere stampato ma per essere usato come è, ad ogni nota è infatti associato un link attivo che permette di consultare la norma associata. Ogni segnalazione è gradita.
Il senso di tutto quello che segue si trova negli articoli 3 e 34 comma 1 della nostra Costituzione
\begin{quote}
	\begin{itemize}
		\item Tutti i cittadini hanno pari dignità sociale e sono eguali davanti alla legge, senza distinzione di sesso, di razza, di lingua, di religione, di opinioni politiche, di condizioni personali e sociali.
		\item È compito della Repubblica rimuovere gli ostacoli di ordine economico e sociale, che, limitando di fatto la libertà e l'eguaglianza dei cittadini, impediscono il pieno sviluppo della persona umana e l'effettiva partecipazione di tutti i lavoratori all'organizzazione politica, economica e sociale del Paese.
	\end{itemize}
\end{quote}
\begin{quote}
	\begin{itemize}
		\item La scuola è aperta a tutti.
	\end{itemize}
\end{quote}
\chapter{Cosa è un BES}
\label{CosaBes}
Definiamo cosa sono i Bisogni Educativi Speciali:
\begin{quote}
ogni alunno, con continuità o per determinati periodi, può manifestare Bisogni Educativi
Speciali: o per motivi fisici, biologici, fisiologici o anche per motivi psicologici, sociali, rispetto ai quali è
necessario che le scuole offrano adeguata e personalizzata risposta~\footcite{dir27Dic2012}
\end{quote}
\begin{quote}
Il Bisogno Educativo Speciale (Special Educational Need) è
qualsiasi difficoltà evolutiva, in ambito educativo ed
apprenditivo, espressa in un funzionamento (nei vari ambiti
della salute secondo il modello \glslink{icfa}{ICF} dell'Organizzazione
mondiale della sanità) problematico anche per il soggetto, in
termini di danno, ostacolo o stigma sociale,
indipendentemente dall'eziologia, e che necessita di
educazione speciale individualizzata.\mancatesto
Un alunno con bisogni educativi speciali è un alunno con apprendimento, sviluppo e comportamento in uno o più dei vari ambiti e competenze, rallentato o problematico, e questa problematicità è riconosciuta per i danni che causa al soggetto stesso, non soltanto tramite il confronto con la normalità. Questi rallentamenti o problematicità possono essere globali e pervasivi (es. Autismo), specifici (es. Dislessia), settoriali (es. Disturbi da deficit attentivi con iperattività) e, naturalmente, più o meno gravi, permanenti o transitori. I fattori causali possono essere a livello organico, psicologico, familiare, sociale, culturale, ecc.~\footcite{ianes2005bisogni}
\end{quote}
La Direttiva di Dicembre divide i BES in tre grandi categorie: 
\begin{itemize}
	\item Disabilità 
	\item Disturbi evolutivi specifici
	\item Svantaggio socio
	economico, linguistico, culturale
\end{itemize}
Questi punti sono già in parte coperti dalla legge. Infatti per la disabilità abbiamo la 104/92\footcite{Legge_104_92}, e fra i disturbi evolutivi specifici troviamo i \glslink{dsaa}{DSA} che sono garantiti dalla 170/10\footcite{legge170}, per lo svantaggio lingiustico valgono le \caporali{Linee guida per l'accoglienza e l'integrazione degli alunni stranieri}~\footcite{lin_1_marzo_2006}, per gli studenti ricoverati in ospedale sono definiti dei percorsi\footcite{cm_24_11} alternativi ma le situazioni borderline, per la disabilità non certificate, \cit{non DSA}, o di svantaggio socio economico, non avevano diritto, prima di questa direttiva, ad alcun aiuto. 
 
La direttiva e la CM 8/2013~\footcite{cm8_2013} hanno lo scopo di estendere ai soggetti BES, non tutelati, quanto previsto dalla legge 170 del 2010 e dal DM 5669/2011~\footcite{decreto5669_2011} e allegate linee guida\footcite{LineGuida2011}. Un'interessante discussione su cosa si intenda per Bisogni Educativi Speciali è stata esposta in uno scritto\footcite{USRperLEmiliaRomagna2013a} pubblicato dall'USR per l'Emilia Romagna che è stato inserito fra le appendici. 
La nota del 22 novembre 2013\footcite{Nota_2563_2013} è intervenuta per spiegare che spetta al Consiglio di Classe di intervenire, magari a seguito di criteri stabiliti dal Collegio Docenti, se attivare percorsi di studio personalizzati ed individualizzati.   