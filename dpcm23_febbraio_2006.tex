\chapter[DPCM 23 febbraio 2006, n. 185]{Decreto del Presidente del Consiglio dei Ministri 23 febbraio 2006, n. 185}
\begin{center}
(in GU 19 maggio 2006, n. 115)
\end{center}

Regolamento recante modalità e criteri per l'individuazione dell'alunno come soggetto in situazione di handicap, ai sensi dell'articolo 35, comma 7, della legge 27 dicembre 2002, n. 289
\begin{center}
IL PRESIDENTE DEL CONSIGLIO DEI MINISTRI
\end{center}
Visto l'articolo 35, comma 7, della legge 27 dicembre 2002, n. 289, che prevede la definizione, con decreto del Presidente del Consiglio dei Ministri, di modalita' e criteri per l'individuazione, da parte delle Aziende Sanitarie Locali, dell'alunno come soggetto portatore di handicap;

Vista la legge 5 febbraio 1992, n. 104, legge-quadro per l'assistenza, l'integrazione sociale e i diritti delle persone handicappate;
Visti, in particolare, gli articoli 3, 12 e 13 della suddetta legge;

Visto il decreto del Presidente della Repubblica del 24 febbraio 1994, concernente l'atto di indirizzo e coordinamento relativo ai compiti delle unita' sanitarie locali in materia di alunni portatori di handicap;

Visto il decreto-legge 3 luglio 2001, n. 255, convertito, con modificazioni, dalla legge 20 agosto 2001, n. 333;
Vista la legge 8 novembre 2000, n. 328, concernente la legge quadro per la realizzazione del sistema integrato di interventi e servizi sociali;

Visto l'articolo 17, comma 3, della legge 23 agosto 1988, n. 400 e successive modificazioni;

Acquisita l'intesa con la Conferenza unificata di cui all'articolo 8 del decreto legislativo 28 agosto 1997, n. 281, sancita nella seduta del 16 giugno 2005 ai sensi dell'articolo 8, comma 6, della legge 5 giugno 2003, n. 131;
Udito il parere del Consiglio di Stato, espresso nella Sezione consultiva per gli atti normativi nell'adunanza del 29 agosto 2005;

Acquisiti i pareri delle competenti commissioni della Camera dei deputati e del Senato della Repubblica, espressi da entrambe le commissioni nelle rispettive sedute del 9 novembre 2005;
Sulla proposta del Ministro dell'istruzione, dell'universita' e della ricerca e del Ministro della salute;
\begin{center}
A d o t t a
\end{center}
il seguente regolamento:
\begin{description}
	\item[Art. 1.] Finalità
	\begin{enumerate}
		\item Il presente decreto stabilisce le modalità e i criteri per l'individuazione dell'alunno in situazione di handicap, a norma di quanto previsto dall'articolo 35, comma 7, della legge 27 dicembre 2002, n. 289.
		\end{enumerate}
	\item [Art. 2.] Modalità e criteri
	\begin{enumerate}
		\item Ai fini della individuazione dell'alunno come soggetto in situazione di handicap, le Aziende Sanitarie dispongono, su richiesta documentata dei genitori o degli esercenti la potesta' parentale o la tutela dell'alunno medesimo, appositi accertamenti collegiali, nel
		rispetto di quanto previsto dagli articoli 12 e 13 della legge 5 febbraio 1992, n. 104.
		\item Gli accertamenti di cui al comma 1, da effettuarsi in tempi utili rispetto all'inizio dell'anno scolastico e comunque non oltre trenta giorni dalla ricezione della richiesta, sono documentati attraverso la redazione di un verbale di individuazione dell'alunno come soggetto in situazione di handicap ai sensi dell'articolo 3, comma 1 della legge 5 febbraio 1992, n. 104, e successive modificazioni. Il verbale, sottoscritto dai componenti il collegio, reca l'indicazione della patologia stabilizzata o progressiva accertata con riferimento alle classificazioni internazionali dell'Organizzazione Mondiale della Sanità nonché la specificazione dell'eventuale carattere di particolare gravita' della medesima, in presenza dei presupposti previsti dal comma 3 del predetto articolo 3. Al fine di garantire la congruenza degli interventi cui gli accertamenti sono preordinati, il verbale indica l'eventuale termine di rivedibilità dell'accertamento effettuato.
		\item Gli accertamenti di cui ai commi precedenti sono propedeutici alla redazione della diagnosi funzionale dell'alunno, cui provvede l'unita' multidisciplinare, prevista dall'articolo 3, comma 2 del decreto del Presidente della Repubblica 24 febbraio 1994, anche secondo i criteri di classificazione di disabilita' e salute previsti dall'Organizzazione Mondiale della Sanità. Il verbale di accertamento, con l'eventuale termine di rivedibilità' ed il documento relativo alla diagnosi funzionale, sono trasmessi ai genitori o agli esercenti la potestà parentale o la tutela dell'alunno e da questi all'istituzione scolastica presso cui l'alunno va iscritto, ai fini della tempestiva adozione dei provvedimenti conseguenti.
	\end{enumerate}
	\item [Art. 3.]Attivazione delle forme di integrazione e di sostegno
	\begin{enumerate}
		\item Alle attività di cui ai commi 1 e 3 del precedente articolo 2 fa seguito la redazione del profilo dinamico funzionale e del piano educativo individualizzato previsti dall'articolo 12, comma 5, della legge 5 febbraio 1992, n. 104, da definire entro il 30 luglio per gli effetti previsti dalla legge 20 agosto 2001, n. 333.
		\item I soggetti di cui all'articolo 5, comma 2, del decreto del Presidente della Repubblica 24 febbraio 1994, in sede di formulazione del piano educativo individualizzato, elaborano proposte relative alla individuazione delle risorse necessarie, ivi compresa l'indicazione del numero delle ore di sostegno.
		\item Gli Enti locali, gli Uffici Scolastici Regionali e le Direzioni Sanitarie delle Aziende Sanitarie, nel quadro delle finalità della legislazione nazionale e regionale vigente in materia adottano accordi finalizzati al coordinamento degli interventi di rispettiva competenza per garantire il rispetto dei tempi previsti per la definizione dei provvedimenti relativi al funzionamento delle classi, ai sensi del decreto-legge 3 luglio 2001, n. 255, convertito, con modificazioni, dalla legge 20 agosto 2001, n. 333. Gli accordi sono finalizzati anche all'organizzazione di sistematiche verifiche in ordine agli interventi realizzati ed alla influenza esercitata dall'ambiente scolastico sull'alunno in situazione di handicap, a norma dell'articolo 6 del decreto del Presidente della Repubblica 24 febbraio 1994.
	\end{enumerate}
	\item [Art. 4.] Situazione di handicap di particolare gravità ed autorizzazione al funzionamento dei posti di sostegno in deroga
	\begin{enumerate}
		\item L'autorizzazione all'attivazione di posti di sostegno in deroga al rapporto insegnanti/alunni, a norma dell'articolo 35, comma 7, della legge 27 dicembre 2002, n. 289, e' disposta dal dirigente preposto all'Ufficio Scolastico Regionale sulla base della certificazione attestante la particolare gravita' di cui all'articolo 2, comma 2 del presente decreto.
	\end{enumerate}
	\item [Art. 5.] Disposizioni finali 
	\begin{enumerate}
		\item Le disposizioni del presente decreto si applicano agli accertamenti da effettuarsi successivamente alla sua entrata in vigore.
	\end{enumerate}
\end{description}

Il presente decreto, munito del sigillo dello Stato, sarà inserito nella Raccolta ufficiale degli atti normativi della Repubblica italiana. E' fatto obbligo a chiunque spetti di osservarlo e farlo osservare.

    Roma, 23 febbraio 2006

    p. Il Presidente
    
    del Consiglio dei Ministri
    
    Letta

    Il Ministro dell'istruzione dell'università e della ricerca
    
    Moratti

    Il Ministro della salute
    
    Storace

    Visto, il Guardasigilli: Castelli
    
    Registrato alla Corte dei conti il 4 maggio 2006
    Ufficio di controllo preventivo sui Ministeri dei servizi 
    alla persona e dei beni culturali, registro n. 2, foglio n. 36   
    
     Avvertenza:

    Il testo delle note qui pubblicato e' stato redatto dall'amministrazione competente per materia, ai sensi dell'art. 10, comma 3, del testo unico delle disposizioni sulla promulgazione delle leggi, sull'emanazione dei decreti del Presidente della Repubblica e sulle pubblicazioni ufficiali della Repubblica italiana, approvato con decreto del Presidente della Repubblica 28 dicembre 1985, n. 1092, al solo fine di facilitare la lettura delle disposizioni di legge alle quali e' operato il rinvio. Restano invariati il valore e l'efficacia degli atti legislativi qui trascritti.

    - Il testo dell'art. 35, comma 7, della legge 27 dicembre 2002, n. 289 (Disposizioni per la formazione del bilancio annuale e pluriennale dello Stato (legge finanziaria 2003) pubblicata nella Gazzetta Ufficiale 31 dicembre 2002, n. 305, S.O.), e' il seguente: «Art. 35 (Misure di razionalizzazione in materia di organizzazione scolastica). (Omissis). 7. Ai fini dell'integrazione scolastica dei soggetti portatori di handicap si intendono destinatari delle attivita' di sostegno ai sensi dell'art. 3, comma 1, della legge 5 febbraio 1992, n. 104, gli alunni che presentano una minorazione fisica, psichica o sensoriale, stabilizzata o progressiva. L'attivazione di posti di sostegno in deroga al rapporto insegnanti/alunni in presenza di handicap particolarmente gravi, di cui all'art. 40 della legge 27 dicembre 1997, n. 449, e' autorizzata dal dirigente preposto all'ufficio scolastico regionale assicurando comunque le garanzie per gli alunni in situazione di handicap di cui al predetto art. 3 della legge 5 febbraio 1992, n. 104. All'individuazione dell'alunno come soggetto portatore di handicap provvedono le aziende sanitarie locali sulla base di accertamenti collegiali, con modalita' e criteri definiti con decreto del Presidente del Consiglio dei Ministri da emanare, d'intesa con la Conferenza unificata di cui all'art. 8 del decreto legislativo 28 agosto 1997, n. 281, e previo parere delle competenti Commissioni parlamentari, su proposta dei Ministri dell'istruzione, dell'universita' e della ricerca e della salute, entro sessanta giorni dalla data di entrata in vigore della presente legge. (Omissis)».
    - Il testo degli articoli 3, 12 e 13 della legge 5 febbraio 1002, n. 104 (Legge quadro per l'assistenza, l'integrazione sociale e i diritti delle persone handicappate), e' il seguente: «Art. 3 (Soggetti aventi diritto). - 1. E' persona handicappata colui che presenta una minorazione fisica, psichica o sensoriale, stabilizzata o progressiva, che e' causa di difficolta' di apprendimento, di relazione o di integrazione lavorativa e tale da determinare un processo di svantaggio sociale o di emarginazione. 2. La persona handicappata ha diritto alle prestazioni stabilite in suo favore in relazione alla natura e alla consistenza della minorazione, alla capacita' complessiva
    individuale residua e alla efficacia delle terapie riabilitative. 3. Qualora la minorazione, singola o plurima, abbia ridotto l'autonomia personale, correlata all'eta', in modo da rendere necessario un intervento assistenziale permanente, continuativo e globale nella sfera individuale o in quella di relazione, la situazione assume connotazione di gravita'. Le situazioni riconosciute di gravita' determinano
    priorita' nei programmi e negli interventi dei servizi pubblici. 4. La presente legge si applica anche agli stranieri e agli apolidi, residenti, domiciliati o aventi stabile dimora nel territorio nazionale. Le relative prestazioni sono corrisposte nei limiti ed alle condizioni previste dalla vigente legislazione o da accordi internazionali.». «Art. 12 (Diritto all'educazione e all'istruzione). - 1. Al bambino da 0 a 3 anni handicappato e' garantito l'inserimento negli asili nido. 2. E' garantito il diritto all'educazione e all'istruzione della persona handicappata nelle sezioni di scuola materna, nelle classi comuni delle istituzioni scolastiche di ogni ordine e grado e nelle istituzioni universitarie. 3. L'integrazione scolastica ha come obiettivo lo sviluppo delle potenzialita' della persona handicappata nell'apprendimento, nella comunicazione, nelle relazioni e nella socializzazione. 4. L'esercizio del diritto all'educazione e all'istruzione non puo' essere impedito da difficolta' di apprendimento ne' da altre difficolta' derivanti dalle disabilita' connesse all'handicap. 5. All'individuazione dell'alunno come persona handicappata ed all'acquisizione della documentazione risultante dalla diagnosi funzionale, fa seguito un profilo dinamico-funzionale ai fini della formulazione di un piano educativo individualizzato, alla cui definizione provvedono congiuntamente, con la collaborazione dei genitori della persona handicappata, gli operatori delle unita' sanitarie locali e, per ciascun grado di scuola, personale insegnante specializzato della scuola, con la partecipazione dell'insegnante operatore psico-pedagogico individuato secondo criteri stabiliti dal Ministro della pubblica istruzione. Il profilo indica le caratteristiche fisiche, psichiche e sociali ed affettive dell'alunno e pone in rilievo sia le difficolta' di apprendimento conseguenti alla situazione di handicap e le possibilita' di recupero, sia le capacita' possedute che devono essere sostenute, sollecitate e progressivamente rafforzate e sviluppate nel rispetto delle scelte culturali della persona handicappata. 6. Alla elaborazione del profilo dinamico-funzionale iniziale seguono, con il concorso degli operatori delle unita' sanitarie locali, della scuola e delle famiglie, verifiche per controllare gli effetti dei diversi interventi e l'influenza esercitata dall'ambiente scolastico. 7. I compiti attribuiti alle unita' sanitarie locali dai commi 5 e 6 sono svolti secondo le modalita' indicate con apposito atto di indirizzo e coordinamento emanato ai sensi dell'art. 5, primo comma, della legge 23 dicembre 1978, n. 833. 8. Il profilo dinamico-funzionale e' aggiornato a conclusione della scuola materna, della scuola elementare e della scuola media e durante il corso di istruzione secondaria superiore. 9. Ai minori handicappati soggetti all'obbligo scolastico, temporaneamente impediti per motivi di salute a frequentare la scuola, sono comunque garantite l'educazione e l'istruzione scolastica. A tal fine il provveditore agli
    studi, d'intesa con le unita' sanitarie locali e i centri di recupero e di riabilitazione, pubblici e privati, convenzionati con i Ministeri della sanita' e del lavoro e della previdenza sociale, provvede alla istituzione, per i minori ricoverati, di classi ordinarie quali sezioni staccate della scuola statale. A tali classi possono essere ammessi anche i minori ricoverati nei centri di degenza, che non versino in situazioni di handicap e per i quali sia accertata l'impossibilita' della frequenza della scuola dell'obbligo per un periodo non inferiore a trenta giorni di lezione. La frequenza di tali classi, attestata dall'autorita' scolastica mediante una relazione sulle attivita' svolte dai docenti in servizio presso il centro di degenza, e' equiparata ad ogni effetto alla frequenza delle classi alle quali i minori sono iscritti. 10. Negli ospedali, nelle cliniche e nelle divisioni pediatriche gli obiettivi di cui al presente articolo possono essere perseguiti anche mediante l'utilizzazione di personale in possesso di specifica formazione psico-pedagogica che abbia una esperienza acquisita presso i nosocomi o segua un periodo di tirocinio di un anno sotto la guida di personale esperto.». «Art. 13 (Integrazione scolastica). - 1. L'integrazione scolastica della persona handicappata nelle sezioni e nelle classi comuni delle scuole di ogni ordine e grado e nelle universita' si realizza, fermo restando quanto previsto dalle leggi 11 maggio 1976, n. 360, e 4 agosto 1977, n.
    517, e successive modificazioni, anche attraverso: a) la programmazione coordinata dei servizi scolastici con quelli sanitari, socio-assistenziali, culturali, ricreativi, sportivi e con altre attivita' sul territorio gestite da enti pubblici o privati. A tale scopo gli enti locali, gli organi scolastici e le unita' sanitarie locali, nell'ambito delle rispettive competenze, stipulano gli accordi di programma di cui all'art. 27 della legge 8 giugno 1990, n. 142. Entro tre mesi dalla data di entrata in vigore della presente legge, con decreto del Ministro della pubblica istruzione, d'intesa con i Ministri per gli affari sociali e della sanita', sono fissati gli indirizzi per la stipula degli accordi di programma. Tali accordi di programma sono finalizzati alla predisposizione, attuazione e verifica congiunta di progetti educativi,
    riabilitativi e di socializzazione individualizzati, nonche' a forme di integrazione tra attivita' scolastiche e attivita' integrative extrascolastiche. Negli accordi sono altresi' previsti i requisiti che devono essere posseduti dagli enti pubblici e privati ai fini della partecipazione alle attivita' di collaborazione coordinate; b) la dotazione alle scuole e alle universita' di attrezzature tecniche e di sussidi didattici nonche' di ogni altra forma di ausilio tecnico, ferma restando la dotazione individuale di ausili e presidi funzionali
    all'effettivo esercizio del diritto allo studio, anche mediante convenzioni con centri specializzati, aventi funzione di consulenza pedagogica, di produzione e adattamento di specifico materiale didattico; c) la programmazione da parte dell'universita' di interventi adeguati sia al bisogno della persona sia alla peculiarita' del piano di studio individuale; d) l'attribuzione, con decreto del Ministro
    dell'universita' e della ricerca scientifica e tecnologica, da emanare entro tre mesi dalla data di entrata in vigore della presente legge, di incarichi professionali ad interpreti da destinare alle universita', per facilitare la frequenza e l'apprendimento di studenti non udenti; e) la sperimentazione di cui al decreto del Presidente della Repubblica 31 maggio 1974, n. 419, da realizzare nelle classi frequentate da alunni con handicap. 2. Per le finalita' di cui al comma 1, gli enti locali e le unita' sanitarie locali possono altresi' prevedere l'adeguamento dell'organizzazione e del funzionamento degli asili nido alle esigenze dei bambini con handicap, al fine di avviarne precocemente il recupero, la socializzazione e l'integrazione, nonche' l'assegnazione di personale docente specializzato e di operatori ed assistenti specializzati. 3. Nelle scuole di ogni ordine e grado, fermo restando, ai sensi del decreto del Presidente della Repubblica
    24 luglio 1977, n. 616, e successive modificazioni, l'obbligo per gli enti locali di fornire l'assistenza per l'autonomia e la comunicazione personale degli alunni con handicap fisici o sensoriali, sono garantite attivita' di sostegno mediante l'assegnazione di docenti specializzati. 4. I posti di sostegno per la scuola secondaria di secondo grado sono determinati nell'ambito dell'organico del personale in servizio alla data di entrata in vigore della presente legge in modo da assicurare un rapporto almeno pari a quello previsto per gli altri gradi di istruzione e comunque entro i limiti delle disponibilita' finanziarie all'uopo preordinate dall'art. 42, comma 6,
    lettera h).
    - Il testo dell'art. 8, comma 6, della legge 5 giugno 2003, n. 131 (Disposizioni per l'adeguamento dell'ordinamento della Repubblica alla legge costituzionale 18 ottobre 2001, n. 3, pubblicata nella Gazzetta Ufficiale 10 giugno 2003, n. 132), e' il seguente: «6. Il Governo puo' promuovere la stipula di intese in sede di Conferenza Stato-Regioni o di Conferenza unificata, dirette a favorire l'armonizzazione delle rispettive legislazioni o il raggiungimento di posizioni unitarie o il conseguimento di obiettivi comuni; in tale caso e' esclusa l'applicazione dei commi 3 e 4 dell'art. 3 del decreto legislativo 28 agosto 1997, n. 281. Nelle materie di cui all'art. 117, terzo e quarto comma, della Costituzione non possono essere adottati gli atti di indirizzo e di coordinamento di cui all'art. 8 della legge 15 marzo 1997, n. 59, e all'art. 4 del decreto legislativo 31 marzo 1998, n. 112.». 5. Nella scuola secondaria di primo e secondo grado sono garantite attivita' didattiche di sostegno, con priorita' per le iniziative sperimentali di cui al comma 1, lettera e), realizzate con docenti di sostegno specializzati, nelle aree disciplinari individuate sulla base del profilo dinamico-funzionale e del conseguente piano educativo individualizzato. 6. Gli insegnanti di sostegno assumono la contitolarita' delle sezioni e delle classi in cui operano, partecipano alla programmazione educativa e didattica e alla elaborazione e verifica delle attivita' di competenza dei consigli di interclasse, dei consigli di classe e dei collegi dei docenti. 6-bis. Agli studenti handicappati iscritti all'universita' sono garantiti sussidi tecnici e didattici specifici, realizzati anche attraverso le convenzioni di cui alla lettera b) del comma 1, nonche' il supporto di appositi servizi di tutorato specializzato, istituiti dalle universita' nei limiti del proprio bilancio e delle risorse destinate alla copertura degli oneri di cui al presente
    comma, nonche' ai commi 5 e 5-bis dell'art. 16.».
    - Il decreto del Presidente della Repubblica 24 febbraio 1994 reca: «Atto di indirizzo e coordinamento relativo ai compiti delle unita' sanitarie locali in materia di alunni portatori di handicap.».
    - Il decreto-legge 3 luglio 2001, n. 255, convertito, con modificazioni, dalla legge 20 agosto 2001, n. 333 reca: «Disposizioni urgenti per assicurare l'ordinato avvio dell'anno scolastico 2001/2002.».
    - La legge 8 novembre 2000, n. 328 reca: «Legge quadro per la realizzazione del sistema integrato di interventi e servizi sociali.».
    - Il testo dell'art. 17, comma 3, della legge 23 agosto 1988, n. 400 (Disciplina dell'attivita' di Governo e ordinamento della Presidenza del Consiglio dei Ministri), e' il seguente: «3. Con decreto ministeriale possono essere adottati regolamenti nelle materie di competenza del Ministro o di autorita' sottordinate al Ministro, quando la legge espressamente conferisca tale potere. Tali regolamenti, per materie di competenza di piu' Ministri, possono essere adottati con decreti interministeriali, ferma restando la necessita' di apposita autorizzazione da parte della legge. I regolamenti ministeriali ed interministeriali non possono dettare norme contrarie a quelle dei regolamenti emanati dal Governo. Essi debbono essere comunicati al Presidente del Consiglio dei Ministri prima della loro emanazione.».
    - Il testo dell'art. 8 del decreto legislativo 28 agosto 1997, n. 281 (Definizione ed ampliamento delle attribuzioni della Conferenza permanente per i rapporti tra lo Stato, le regioni e le province autonome di Trento e Bolzano ed unificazione, per le materie ed i compiti di interesse comune delle regioni, delle province e dei comuni, con la Conferenza Stato-citta' ed autonomie locali, pubblicato nella Gazzetta Ufficiale 30 agosto 1997, n. 202), e' il seguente: «Art. 8 (Conferenza Stato-citta' ed autonomie locali e Conferenza unificata). - 1. La Conferenza Stato-citta' ed autonomie locali e' unificata per le materie ed i compiti di interesse comune delle regioni, delle province, dei comuni e delle comunita' montane, con la Conferenza Stato-regioni. 2. La Conferenza Stato-citta' ed autonomie locali e' presieduta dal Presidente del Consiglio dei ministri o, per sua delega, dal Ministro dell'interno o dal Ministro per gli affari regionali; ne fanno parte altresi' il Ministro del tesoro e del bilancio e della programmazione economica, il Ministro delle finanze, il Ministro dei lavori pubblici, il Ministro della sanita', il presidente dell'Associazione nazionale dei comuni d'Italia - ANCI, il presidente dell'Unione province d'Italia - UPI ed il presidente dell'Unione nazionale comuni, comunita' ed enti montani - UNCEM. Ne fanno parte inoltre quattordici sindaci designati dall'ANCI e sei presidenti di provincia designati dall'UPI. Dei quattordici sindaci designati dall'ANCI cinque rappresentano le citta' individuate dall'art. 17 della legge 8 giugno 1990, n. 142. Alle riunioni possono essere invitati altri membri del Governo, nonche' rappresentati di amministrazioni statali, locali o di enti pubblici. 3. La Conferenza Stato-citta' ed autonomie locali e' convocata almeno ogni tre mesi, e comunque in tutti i casi il presidente ne ravvisi la necessita' o qualora ne faccia richiesta il presidente dell'ANCI, dell'UPI o dell'UNCEM. 4. La Conferenza unificata di cui al comma 1 e' convocata dal Presidente del Consiglio dei Ministri. Le sedute sono presiedute dal Presidente del Consiglio dei Ministri o, su sua delega, dal Ministro per gli affari regionali o, se tale incarico non e' conferito, dal Ministro dell'interno.».

    Nota all'art. 1:
    - Per il testo dell'art. 35, comma 7 delle legge 27 dicembre 2002, n. 289 si vedano le note alle premesse.

    Note all'art. 2:
    - Per il testo degli articoli 12 e 13 della legge 5 febbraio 1992, n. 104 si vedano le note alle premesse.
    - Per il testo dell'art. 3 della legge 5 febbraio 1992, n. 104 si vedano le note alle premesse.
    - Il testo dell'art. 3, comma 2 del decreto del Presidente della Repubblica 24 febbraio 1994, e' il seguente: «2. Alla diagnosi funzionale provvede l'unita' multidisciplinare composta: dal medico specialista nella patologia segnalata, dallo specialista in neuropsichiatria infantile, dal terapista della riabilitazione, dagli operatori sociali in servizio presso la unita' sanitaria locale o in regime di convenzione con la medesima. La diagnosi funzionale deriva dall'acquisizione di elementi clinici e psico-sociali. Gli elementi clinici si acquisiscono tramite la visita medica diretta dell'alunno e l'acquisizione dell'eventuale documentazione medica preesistente. Gli elementi psico-sociali si acquisiscono attraverso specifica relazione in cui siano ricompresi: a) i dati anagrafici del soggetto; b) i dati relativi alle caratteristiche del nucleo familiare (composizione, stato di salute dei membri, tipo di lavoro svolto, contesto ambientale, ecc.).».

    Note all'art. 3:
    - Per il testo dell'art. 12 della legge 5 febbraio 1992, n. 104 si vedano le note alle premesse.
    - La legge 20 agosto 2001, n. 333 reca: «Conversione in legge, con modificazioni, del decreto-legge 3 luglio 2001, n. 255, recante disposizioni urgenti per assicurare l'ordinato avvio dell'anno scolastico 2001-2002.».
    - Il testo dell'art. 5, comma 2, del citato decreto del Presidente della Repubblica 24 febbraio 1994, e' il seguente: «2. Il P.E.I. e' redatto, ai sensi del comma 5 del predetto art. 12, congiuntamente dagli operatori sanitari individuati dalla USL e/o USSL e dal personale insegnante curriculare e di sostegno della scuola e, ove presente, con la partecipazione dell'insegnante operatore psico-pedagogico, in collaborazione con i genitori o gli esercenti la potesta' parentale dell'alunno.».
    - Il decreto-legge 3 luglio 2001, n. 255, convertito, con modificazioni, dalla legge 20 agosto 2001, n. 333 reca: «Disposizioni urgenti per assicurare l'ordinato avvio dell'anno scolastico 2001/2002.».
    - Il testo dell'art. 6 del citato decreto del Presidente della Repubblica 24 febbraio 1994, e' il seguente: «Art. 6 (Verifiche). - 1. Con frequenza, preferibilmente, correlata all'ordinaria ripartizione dell'anno scolastico o, se possibile, con frequenza trimestrale (entro ottobre-novembre, entro febbraio-marzo, entro maggio-giugno), i soggetti indicati al comma 6 dell'art. 12 della legge n. 104 del 1992, verificano gli effetti dei diversi interventi disposti e l'influenza esercitata dall'ambiente scolastico sull'alunno in
    situazione di handicap. 2. Le verifiche di cui al comma precedente sono finalizzate a che ogni intervento destinato all'alunno in
    situazione di handicap sia correlato alle effettive potenzialita' che l'alunno stesso dimostri di possedere nei vari livelli di apprendimento e di prestazioni educativo-riabilitative, nel rispetto della sua salute mentale. 3. Qualora vengano rilevate ulteriori difficolta' (momento di crisi specifica o situazioni impreviste relative all'apprendimento) nel quadro comportamentale o di relazione o relativo all'appuntamento del suddetto alunno, congiuntamente o da parte dei singoli soggetti di cui al comma 1, possono essere effettuate verifiche straordinarie, al di fuori del termine indicato dallo stesso comma 1. Gli esiti delle verifiche devono confluire nel P.E.I.».

    Nota all'art. 4:
    - Per il testo dell'art. 35, comma 7, della legge 27 dicembre 2002, n. 289 si vedano le note alle premesse.
