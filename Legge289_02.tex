\chapter{Legge 27 dicembre 2002, n. 289}
\label{cha:Legge_27_dicembre_2002_289}

Art. 35
(Misure di razionalizzazione in materia di organizzazione scolastica)

7. Ai fini dell'integrazione scolastica dei soggetti portatori di handicap si intendono destinatari delle attivita' di sostegno ai sensi dell'articolo 3, comma 1, della legge 5 febbraio 1992, n. 104, gli alunni che presentano una minorazione fisica, psichica o sensoriale, stabilizzata o progressiva. L'attivazione di posti di sostegno in deroga al rapporto insegnanti/ alunni in presenza di handicap particolarmente gravi, di cui all'articolo 40 della legge 27 dicembre 1997, n. 449, e' autorizzata dal dirigente preposto all'ufficio scolastico regionale assicurando comunque le garanzie per gli alunni in situazione di handicap di cui al predetto articolo 3 della legge 5 febbraio 1992, n. 104. All'individuazione dell'alunno come soggetto portatore di handicap provvedono le aziende sanitarie locali sulla base di accertamenti collegiali, con modalità e criteri definiti con decreto del Presidente del Consiglio dei ministri da emanare, d'intesa con la Conferenza unificata di cui all'articolo 8 del decreto legislativo 28 agosto 1997, n. 281, e previo parere delle competenti Commissioni parlamentari, su proposta dei Ministri dell'istruzione, dell'università' e della ricerca e della salute, entro sessanta giorni dalla data di entrata in vigore della presente legge.