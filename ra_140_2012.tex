\chapter[RA 140 del 25 luglio 2012]{Accordo tra Governo, Regioni e Province autonome di Trento e Bolzano su “Indicazioni per la diagnosi e la certificazione dei Disturbi specifici di apprendimento (DSA)”}
Accordo, ai sensi dell’articolo 4, del decreto legislativo 28 agosto 1997, n. 281.
Repertorio Atti n.140/CSR del 25 luglio 2012
\begin{center}
LA CONFERENZA PERMANENTE PER I RAPPORTI TRA LO STATO, LE REGIONI E LE PROVINCE AUTONOME DI TRENTO E BOLZANO,
\end{center}

Nella odierna seduta del 25 luglio 2012:

VISTI gli articoli 2, comma 1, lett. b) e 4, comma 1, del decreto legislativo 28 agosto 1997, n. 281, che affidano a questa Conferenza il compito di promuovere e sancire accordi tra Governo e Regioni, in attuazione del principio di leale collaborazione, al fine di coordinare l'esercizio delle rispettive competenze e svolgere attività di interesse comune;

VISTA la legge 8 ottobre n. 170 recante “Nuove norme in materia di disturbi specifici di apprendimento in ambito scolastico”;

VISTO in particolare l'articolo 3, comma 1, della legge n. 170 del 2010, il quale attribuisce alle Regioni nel cui territorio non sia possibile effettuare la diagnosi nell'ambito dei trattamenti specialistici erogati dal Servizio sanitario nazionale, la facoltà di prevedere che la medesima diagnosi sia effettuata da specialisti o strutture accreditati;

RITENUTO che la locuzione “specialisti o strutture accreditate” utilizzata dalla disposizione citata per l'individuazione dei soggetti che potranno affiancare il Servizio sanitario nazionale nell'attività diagnostica, debba essere interpretata come riferita a soggetti specificamente riconosciuti dalle regioni per il rilascio della certificazione di DSA;

RITENUTO necessario fornire criteri qualitativi utili all'individuazione di specialisti e strutture che offrano garanzie nello svolgimento dell'attività diagnostica, ai fini del riconoscimento da parte delle Regioni;

RITENUTO necessario, altresì, fornire criteri per lo svolgimento dell'attività diagnostica che contemperino le esigenze del Servizio sanitario nazionale e quelle delle istituzioni scolastiche in ordine alla tempestività della certificazione di DSA ed agli elementi conoscitivi che devono esservi riportati per consentire agli insegnanti di svolgere adeguatamente i compiti loro assegnati dalla legge n. 170 del 2010 ed agli alunni/studenti con DSA di fruire dei benefici e delle tutele che la stessa legge garantisce loro;

VISTO l'articolo 7, comma 3, della citata legge n. 170/2010 il quale prevede che, con decreto del Ministro dell'istruzione, dell'università e della ricerca, è istituito presso il Ministero stesso un Comitato tecnico-scientifico, composto da esperti di comprovata competenza sui DSA;

PRESO ATTO del risultato dell'attività svolta dal Comitato tecnico scientifico sui DSA istituito dal Ministro dell'istruzione, dell'università e della ricerca con decreto del 14 dicembre 2010 in attuazione dell'articolo 7, comma 3, della legge 8 ottobre 2010, n. 170;

VISTO il documento della Consensus Conference sui Disturbi specifici di apprendimento svoltasi presso l’Istituto Superiore di Sanità il 6 e 7 dicembre 2010 nell'ambito del Sistema Nazionale delle Linee Guida;

VISTE le note del 9 marzo e del 11 aprile 2012, con le quali il Ministero dell'istruzione, dell'università e della ricerca, ha provveduto a trasmettere una proposta di accordo concernente l'argomento indicato in oggetto, che è stata diramata, in data 11 aprile 2012, alle Regioni e alle Province autonome;

CONSIDERATO che, nella riunione, a livello tecnico, tenutasi il 19 giugno 2012, i rappresentanti delle Regioni hanno proposto alcune modifiche al testo, riferite agli articoli 1, commi 1, 2 3 e 4; all'articolo 2, commi 1 e 4 e all'articolo 3; inoltre, il rappresentante del coordinamento interregionale della salute ha rappresentato la necessità di rivedere il sistema di accreditamento delle strutture che possono diagnosticare i DSA.;

CONSIDERATO che, al riguardo, il Ministero autodistruzione, dell'università e della ricerca, con nota del 20 giugno 2012, a seguito della suindicata riunione tecnica, ha trasmesso una nuova formulazione, condivisa con il Ministero della salute, del provvedimento indicato in oggetto e relativo Modello di certificazione, diramati, in pari data, alle Regioni e alle Province autonome;

RILEVATO che il provvedimento è stato iscritto alla seduta di questa Conferenza del 21 giugno 2012, che non si è tenuta;

VISTA la nota pervenuta il 16 luglio e diramata il 17 luglio 2012, con la quale il Coordinamento interregionale salute ha inviato un documento di proposte emendative sul provvedimento in parola;

CONSIDERATO che al riguardo, ai fini dell'ulteriore esame del testo dell'accordo, è stata convocata una riunione tecnica il 23 luglio 2012 nella quale sono state esaminate le proposte emendative delle Regioni, nonché le osservazioni fatte pervenire dal Ministero della salute, in merito alle richieste delle Regioni;

CONSIDERATO altresì che, nella medesima sede tecnica, è stata accolta la richiesta del Ministero dell'economia e delle finanze di inserire alla fine del comma 4 dell'articolo 1, la clausola di salvaguardia finanziaria: “senza nuovi o maggiori oneri per la finanza pubblica”; 

CONSIDERATO altresì che, a conclusione dell'incontro, è stata condivisa la formulazione definitiva dell'accordo che il Ministero dell'istruzione ha fatto pervenire il 24 luglio 2012 e che è stato trasmesso, in pari data, alle Regioni ed alle Province autonome;

RILEVATO che nella odierna seduta di questa Conferenza le Regioni hanno espresso avviso favorevole al perfezionamento dell'accordo nella versione concordata nella riunione tecnica del 23 luglio 2012 e diramata in data 24 luglio 2012;

ACQUISITO, nel corso dell'odierna seduta di questa Conferenza, l'assenso del Governo, delle Regioni e delle Province autonome di Trento e Bolzano:
\begin{center}
SANCISCE ACCORDO
\end{center}
tra il Governo, le Regioni e le Province autonome di Trento e di Bolzano nei termini sotto indicati;
\begin{description}
	\item[Art. 1] Attivazione del percorso diagnostico
	\begin{enumerate}
		\item Il Ministero dell'istruzione, dell'università e della ricerca, il Ministero della salute, le Regioni e le Province autonome di Trento e di Bolzano convengono e raccomandano che la diagnosi di DSA debba essere tempestiva e che il percorso diagnostico debba essere attivato solo dopo la messa in atto da parte della scuola degli interventi educativo-didattici previsti dall'articolo 3, comma 2, della legge n. 170/2010, e in esito alle procedure di riconoscimento precoce, di cui al comma 3 del medesimo articolo 3.
		\item Le Regioni e le Aziende sanitarie si impegnano ad adottare le misure organizzative che consentono di attivare tempestivamente la consultazione per DSA.
		\item I servizi pubblici e i soggetti accreditati ai sensi dell'art. 8 quinquies del decreto legislativo n. 502 del 1992 e s.m.i. effettuano il percorso diagnostico e il rilascio delle certificazioni in coerenza con le indicazioni della Consensus Conference. La diagnosi di DSA deve essere prodotta in tempo utile per l'attivazione delle misure didattiche e delle modalità di valutazione previste, quindi, di norma, non oltre il 31 marzo per gli alunni che frequentano gli anni terminali di ciascun ciclo scolastico, in ragione degli adempimenti connessi agli esami di Stato. Fa eccezione la prima certificazione diagnostica, che è prodotta al momento della sua formulazione, indipendentemente dal periodo dell'anno in cui ciò avviene.
		\item Nel caso in cui i servizi pubblici o accreditati dal Servizio sanitario nazionale non siano in grado di garantire il rilascio delle certificazioni in tempi utili per l'attivazione delle misure didattiche e delle modalità di valutazione previste e, comunque, quando il tempo richiesto per il completamento dell'iter diagnostico superi sei mesi, con riferimento agli alunni del primo ciclo di istruzione, le Regioni, per garantire la necessaria tempestività, possono prevedere percorsi specifici per l'accreditamento di ulteriori soggetti privati ai fini dell'applicazione dell'art 3 comma 1 della legge n.170 del 2010, senza nuovi o maggiori oneri per la finanza pubblica.
		\end{enumerate}
		\item[Art. 2] Criteri qualitativi per l'individuazione dei soggetti accreditati per il rilascio della diagnosi
		\begin{enumerate}
			\item Ai soli fini del rilascio delle diagnosi di DSA, gli specialisti e le strutture per poter essere accreditati ai sensi dell'art. 3 della legge n. 170/2010, devono dimostrare il possesso dei seguenti requisiti:
			\begin{itemize}
				\item documentata esperienza nell'attività diagnostica dei DSA;
				\item disponibilità di un'equipe multidisciplinare costituita da neuropsichiatri infantili, psicologi, logopedisti eventualmente integrata da altri professionisti sanitari e modulabile in base alle fasce di età;
				\item dichiarazione di impegno a rispettare le Raccomandazioni per la pratica clinica dei DSA (2007-2009) e il suo aggiornamento, nonché i risultati della Consensus Conference dell'Istituto Superiore di Sanità, in merito:
				\begin{description}
					\item[a)] alle procedure diagnostiche utilizzate, e più precisamente: alla ricerca dei criteri di inclusione e di esclusione; alla adeguata misurazione delle competenze cognitive; alla rilevazione delle competenze specifiche e delle competenze accessorie necessarie alla formulazione del profilo del disturbo;
					\item [b)] alla formulazione della diagnosi. A questo fine, la diagnosi clinica deve essere corredata dagli elementi che consentano di verificare il rispetto delle raccomandazioni della Consensus Conference (2007-2009) e del suo aggiornamento, nonché della Consensus Conference dell’ISS;
					\item [c)]alla multidisciplinarietà.
				\end{description}
			\end{itemize}
			\item Le Regioni fissano le modalità per verificare nel tempo il mantenimento dei requisiti previsti nel presente articolo.
			\item Nelle more del completamento, da parte delle Regioni, delle procedure di accreditamento di ulteriori soggetti privati o di percorsi diagnostici, le Regioni individuano misure transitorie per ovviare ad eventuali carenze o ritardi da parte dei servizi pubblici o accreditati dal SSN, al fine di consentire agli alunni e studenti con DSA di usufruire delle misure previste dalla legge n. 170/2010. 
			\end{enumerate}
\item [Art. 3]Elementi della certificazione di DSA
\begin{enumerate}
	\item La certificazione di DSA deve evidenziare che il percorso diagnostico è stato effettuato secondo quanto previsto dalla Consensus Conference e deve essere articolata e formalmente chiara. E’ necessario il riferimento ai codici nosografici (attualmente, tutti quelli compresi nella categoria F81: Disturbi evolutivi Specifici delle Abilità Scolastiche dell'ICD-10) e alla dicitura esplicita del DSA in oggetto (della Lettura e/o della Scrittura e/o del Calcolo).
	\item La certificazione di DSA contiene le informazioni necessarie per stilare una programmazione educativa e didattica che tenga conto delle difficoltà del soggetto e preveda l'applicazione mirata delle misure previste dalla legge. La menzione della categoria diagnostica non è infatti sufficiente per la definizione di quali misure didattiche siano appropriate per il singolo soggetto. A tal fine è necessario che la certificazione di DSA contenga anche gli elementi per delineare un profilo di funzionamento (che definisce più precisamente le caratteristiche individuali con le aree di forza e di debolezza). Tale descrizione deve essere redatta in termini comprensibili e facilmente traducibile in indicazioni operative per la prassi didattica.
	\item Il profilo di funzionamento è di norma aggiornato:
	\begin{itemize}
		\item al passaggio da un ciclo scolastico all'altro e comunque, di norma, non prima di tre anni dal precedente;
		\item ogni qualvolta sia necessario modificare l'applicazione degli strumenti didattici e valutativi necessari, su segnalazione della scuola alla famiglia o su iniziativa della famiglia.
	\end{itemize}
	\item Al fine di semplificare l'iter procedurale della certificazione, con particolare riguardo alla fase di ricezione della documentazione da parte delle istituzioni scolastiche, nonché di rendere uniformi modalità e forme di attestazione della diagnosi su tutto il territorio nazionale, si fornisce, allegato al presente Accordo, un modello di certificazione ai fini disapplicazione delle misure previste dalla legge n. 170/2010, per essere utilizzato dalle strutture preposte
	\item La certificazione di DSA -- su richiesta della famiglia -- è trasmessa, ove possibile, per via telematica alla scuola, nel rispetto della normativa sulla privacy.
\end{enumerate}
\end{description}

\begin{tabular*}{\textwidth}%
	{@{\extracolsep{\fill}}cc}
Il Segretario	&Il Presidente\\
	Cons. Ermenegilda Siniscalchi&Dott. Piero Gnudi
\end{tabular*}

\includepdf[pages={6,7,8},pagecommand={\thispagestyle{plain}}]{DOC_037451_140}