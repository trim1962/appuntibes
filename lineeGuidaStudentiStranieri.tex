\chapter[Accoglienza e integrazione alunni stranieri]{Linee guida
	per l'accoglienza e l'integrazione degli alunni stranieri}
\label{cha:AccoglienzaStudentiStranieri}
8. La valutazione

La valutazione degli alunni stranieri, in particolare di coloro che si possono definire
neo-arrivati, pone diversi ordini di problemi, dalle modalità di valutazione a quelle di
certificazione, alla necessità di tener conto del singolo percorso di apprendimento. La pur
significativa normativa esistente sugli alunni con cittadinanza non italiana non fornisce
indicazioni specifiche a proposito della valutazione degli stessi.
Dall'emanazione della legge n. 517 del 4 agosto 1977 ad oggi, l'approccio alla
valutazione nella scuola è positivamente cambiato. Accanto alla funzione certificativa si è
andata sempre più affermando la funzione regolativa in grado di consentire, sulla base delle
informazioni via via raccolte, un continuo adeguamento delle proposte di formazione alle
reali esigenze degli alunni e ai traguardi programmati per il miglioramento dei processi e dei
risultati, sollecitando, altresì, la partecipazione degli alunni e delle famiglie al processo di
apprendimento. L'art. 4 del DPR n. 275/1999, relativo all'autonomia didattica delle
istituzioni scolastiche, assegna alle stesse la responsabilità di individuare le modalità e i
criteri di valutazione degli alunni, prevedendo altresì che esse operino “nel rispetto della
normativa nazionale”.
Il riferimento più congruo a questo tema lo si ritrova nell'art. 45, comma 4, del DPR
n 394 del 31 agosto 1999 che così recita “il collegio dei docenti definisce, in relazione al
livello di competenza dei singoli alunni stranieri, il necessario adattamento dei programmi
di insegnamento …”. Benché la norma non accenni alla valutazione, ne consegue che il
possibile adattamento dei programmi per i singoli alunni comporti un adattamento della
valutazione, anche in considerazione degli orientamenti generali su questo tema, espressi in
circolari e direttive, che sottolineano fortemente l'attenzione ai percorsi personali degli
alunni. Questa norma va ora inquadrata nel nuovo assetto ordinamentale ed educativo
esplicitato dalle “Indicazioni Nazionali per i piani di studio personalizzati” e con le finalità
del “Profilo educativo dello studente” che costituiscono il nuovo impianto pedagogico, didattico ed organizzativo della scuola italiana, basato sulla L 53/03, art. 3, relativi in particolare alla valutazione.
Per il consiglio di classe che deve valutare alunni stranieri inseriti nel corso dell'anno
scolastico – per i quali i piani individualizzati prevedono interventi di educazione
linguistica e di messa a punto curricolare - diventa fondamentale conoscere, per quanto
possibile, la storia scolastica precedente, gli esiti raggiunti, le caratteristiche delle scuole
frequentate, le abilità e le competenze essenziali acquisite. In questo contesto, che privilegia
la valutazione formativa rispetto a quella “certificativa” si prendono in considerazione il
percorso dell'alunno, i passi realizzati, gli obiettivi possibili, la motivazione e l'impegno e,
soprattutto, le potenzialità di apprendimento dimostrate. In particolare, nel momento in cui
si decide il passaggio o meno da una classe all'altra o da un grado scolastico al successivo,
occorre far riferimento a una pluralità di elementi fra cui non può mancare una previsione di
sviluppo dell'alunno. Emerge chiaramente come nell'attuale contesto normativo vengono
rafforzati il ruolo e la responsabilità delle istituzioni scolastiche autonome e dei docenti
nella valutazione degli alunni.