\chapter{Esempi di regolamento per i GLI}
\section*{Regolamento del GLI (Gruppo di lavoro per l'inclusione) d'Istituto}
\begin{description}
	\item[Art.1] -- Composizione 
	
	Presso il nostro Istituto\footcite{IISGGasparriniMelfi2013} viene costituito, conformemente all'art. 15 comma 2 della legge quadro
	5/02/1992 n.104 e alla restante normativa di riferimento, il Gruppo di Lavoro per l'inclusione, il cui
	compito, oltre a quello di collaborare all'interno dell'istituto alle iniziative educative e di
	integrazione che riguardano studenti con disabilità o con disturbi specifici di apprendimento (DSA),
	si estende alle problematiche relative a tutti i BES.
	
	Il GLI d'Istituto è composto da:
	\begin{enumerate}
		\item  il Dirigente scolastico, che lo presiede;
		\item  il Docente referente del GLH e dei DSA;
		\item  i coordinatori dei Consigli di classe in cui siano presenti alunni con disabilità (e con DSA);
		\item un docente curricolare;
		\item  i docenti specializzati per le attività di sostegno degli alunni con disabilità certificata;
		\item un rappresentante dei genitori di studenti con disabilità (e/o DSA)
		\item un rappresentante degli studenti con disabilità (e/o DSA)
		\item  un rappresentante degli studenti
		\item uno o più rappresentanti degli operatori sociali o sanitari che al di fuori dell'Istituto si occupano
		degli alunni BES.
	\end{enumerate}
	\item [Art.2] -- Convocazione e Riunioni
	
	Le riunioni sono convocate dal Dirigente Scolastico e presiedute dallo stesso o da un suo delegato.
	Le deliberazioni sono assunte a maggioranza dei componenti.
	Di ogni seduta deve essere redatto apposito verbale.
	Il GLI si può riunire in seduta plenaria (con la partecipazione di tutti i componenti), ristretta (con
	la sola presenza degli insegnanti), o dedicata (con la partecipazione delle persone che si occupano
	in particolare di un alunno). In quest'ultimo caso il GLI è detto operativo.
	Gli incontri di verifica con gli operatori sanitari sono equiparati a riunioni del GLI in seduta
	dedicata.
	\item [Art.3] -- Competenze
	
	Il GLI d'Istituto presiede alla programmazione generale dell'integrazione scolastica nella scuola ed
	ha il compito di collaborare alle iniziative educative e di integrazione previste dal piano educativo
	individualizzato dei singoli alunni attraverso l'attuazione di precoci interventi atti a prevenire il
	disadattamento e l'emarginazione e finalizzati alla piena realizzazione del diritto allo studio degli
	alunni con disabilità.
	
	In particolare il GLI svolge le seguenti funzioni:
	\begin{enumerate}
		\item rilevare i BES presenti nella scuola;
		\item elaborare una proposta di Piano Annuale per l'Inclusività riferito a tutti gli alunni con BES,
		da redigere al termine di ogni anno scolastico (entro il mese di giugno, discusso e deliberato
		in Collegio dei Docenti e inviato ai competenti Uffici degli UUSSRR, nonché ai GLIP e al
		GLIR);
		\item rilevare, monitorare e valutare il livello di inclusività della scuola;
		\item gestire e coordinare l'attività dell'Istituto in relazione agli alunni con disabilità al fine di
		ottimizzare le relative procedure e l'organizzazione scolastica;
		\item  analizzare la situazione complessiva dell'istituto (numero di alunni con disabilità, DSA,
		BSE, tipologia dello svantaggio, classi coinvolte);
		\item individuare i criteri per l'assegnazione degli alunni con disabilità alle classi;
		\item individuare i criteri per l'assegnazione dei docenti di sostegno alle classi, per la
		distribuzione delle ore delle relative aree e per l'utilizzo delle compresenze tra i docenti;
		\item definire le linee guida per le attività didattiche di sostegno agli alunni con disabilità
		dell'Istituto da inserire nel POF;
		\item seguire l'attività dei Consigli di classe e degli insegnanti specializzati per le attività di
		sostegno, verificando che siano attuate le procedure corrette e che sia sempre perseguito il
		massimo vantaggio per lo sviluppo formativo degli alunni nel rispetto della normativa;
		\item proporre l’acquisto di attrezzature, strumenti, sussidi, ausili tecnologici e materiali didattici
		destinati agli alunni con disabilità e DSA o ai docenti che se ne occupano;
		\item definire le modalità di accoglienza degli alunni con disabilità;
		\item analizzare casi critici e proposte di intervento per risolvere problematiche emerse nelle
		attività di integrazione;
		\item formulare proposte per la formazione e l'aggiornamento dei docenti.
	\end{enumerate}
	\item [Art.4] -- Competenze del referente del GLH
	
	Il Docente Referente del GLH si occupa di:
	\begin{enumerate}
		\item  convocare e presiedere, su delega del Dirigente Scolastico, le riunioni del GLH;
		\item predisporre gli atti necessari per le sedute del GLH;
		\item verbalizzare le sedute del GLH;
		\item curare la documentazione relativa agli alunni con disabilità, verificarne la regolarità e aggiornare
		i dati informativi (generalità, patologie, necessità assistenziali e pedagogiche, ecc.), sostenendone la
		sicurezza ai sensi del Documento programmatico sulla sicurezza dei dati personali e sensibili
		dell'Istituto;
		\item collaborare col Dirigente Scolastico all'elaborazione dell'orario degli insegnanti di sostegno,
		sulla base dei progetti formativi degli alunni e delle contingenti necessità didattico-organizzative;
		\item collaborare col Dirigente Scolastico alla elaborazione del quadro riassuntivo generale della
		richiesta di organico dei docenti di sostegno sulla base delle necessità formative degli alunni con
		disabilità desunte dai relativi PEI e dalle relazioni finali sulle attività di integrazione messe in atto
		dai rispettivi Consigli di classe;
		\item collaborare all'accoglienza dei docenti specializzati per le attività di sostegno;
		\item curare l'espletamento da parte dei Consigli di classe o dei singoli docenti di tutti gli atti dovuti
		secondo le norme vigenti;
		\item tenere i contatti con gli EE.LL. e con l'Unità multidisciplinare;
		\item curare l'informazione sulla normativa scolastica relativa all'integrazione degli alunni disabili;
		\item  curare, in collaborazione con l'Ufficio di Segreteria, le comunicazioni dovute alle famiglie e/o
		all'Ufficio Scolastico Territoriale di competenza.
	\end{enumerate}
	\item [Art.5] -- Competenze della Commissione per gli alunni con disabilità
	
	All'interno del Gruppo di lavoro sull'handicap i docenti di sostegno della scuola costituiscono una
	Commissione che si occupa degli aspetti che più strettamente riguardano le attività didattiche dei
	Consigli di Classe in cui sono presenti alunni con disabilità, ed in particolare di:
	\begin{enumerate}
		\item  analisi e revisione del materiale strutturato utile ai docenti per migliorare gli aspetti della
		programmazione (modello PDF, modello di PEI, relazione iniziale e finale, ecc..);
		\item sostegno, informazione e consulenza per i docenti riguardo le problematiche relative
		all'integrazione scolastica degli alunni con disabilità;
		\item individuazione di strategie didattiche rispondenti ai bisogni delle specifiche disabilità;
		\item collaborazione con gli specialisti che seguono periodicamente i ragazzi con disabilità;
		\item analisi dell'andamento didattico-disciplinare degli alunni con disabilità;
		\item segnalazione di casi critici e di esigenze di intervento rese necessarie da difficoltà emerse nelle
		attività di integrazione;
		\item sostegno alle famiglie;
		\item analisi degli elementi utili alla definizione della proposta per l’organico dei docenti di sostegno.
	\end{enumerate}
	\item [Art. 6] --  Competenze dei docenti specializzati per le attività di sostegno
	
	I docenti specializzati per le attività di sostegno devono inoltre:
	\begin{enumerate}
		\item informare gli altri membri del Consiglio di Classe sulle problematiche relative all'alunno
		con disabilità e sulle procedure previste dalla normativa;
		\item redigere il PDF e il PEI in versione definitiva;
		\item seguire l'attività educativa e didattica degli alunni con disabilità a loro affidati, secondo le
		indicazioni presenti nei relativi PEI;
		\item mediare, in collaborazione con il Coordinatore di classe, le relazioni tra il Consiglio di
		Classe e la famiglia dell'alunno con disabilità;
		\item relazionare sull'attività didattica svolta per gli alunni con disabilità e su qualsiasi problema
		che emerga rispetto all'integrazione scolastica.
	\end{enumerate}
	\item [Art. 7] -- Competenze dei Consigli di classe con alunni con disabilità
	
	I Consigli di Classe in cui siano inseriti alunni con disabilità, devono:
	\begin{enumerate}
		\item essere informati sulle problematiche relative all'alunno con disabilità per quanto è
		necessario all'espletamento dell'attività didattica;
		\item essere informati sulle procedure previste dalla normativa;
		\item discutere e approvare il percorso formativo più opportuno per l'alunno;
		\item definire e compilare la documentazione prevista (PDF; PEI) entro le date stabilite;
		\item  effettuare la verifica del PEI nei tempi e nelle modalità previsti, allo scopo di prevedere
		eventuali modificazioni e miglioramenti adeguati alle difficoltà riscontrate e valorizzare le
		pratiche di successo.
	\end{enumerate}
	\item [Art. 8] -- Competenze dei singoli docenti curricolari
	
	I singoli docenti che seguono alunni con disabilità, oltre a quanto descritto nell'art. 7, devono:
	\begin{enumerate}
		\item contribuire, in collaborazione con l'insegnante specializzato, all'elaborazione del P.E.I;
		\item seguire per gli alunni con disabilità le indicazioni presenti nei PEI relativi riguardo agli
		obiettivi, alle metodologie e attività e alle modalità di verifica e valutazione;
		\item segnalare al Coordinatore di classe, all'insegnante specializzato e al Referente del GLH
		qualsiasi problema inerente l'attività formativa che coinvolga gli alunni con disabilità;
		\item il docente coordinatore di Classe parteciperà agli incontri di verifica con gli operatori
		sanitari.
	\end{enumerate}
\end{description}

I singoli docenti oltre a quanto stabilito negli articoli precedenti, devono segnalare al Coordinatore
di classe, all'insegnante di sostegno o al Referente del GLH qualsiasi problema inerente l'attività
formativa che coinvolga alunni con disabilità certificate o disturbi specifici di apprendimento.
\section*{Regolamento del  Gruppo di Lavoro per l'Inclusione (GLI)}
Il Gruppo di Lavoro per l'Inclusione\footcite{ISISLinoZanussiPordenone2013} è istituito in conformità all'art.15 comma 2 della
L.104/92 per attuare un'efficace capacità di rilevazione e intervento relativi alle
problematiche degli allievi con bisogni educativi speciali (BES) in relazione alla Circolare
MIUR N. 8 del 6 Marzo 2013, prot. N. 561 \cit{strumenti d'intervento per alunni con Bisogni
Educativi Speciali}.
Il GLI dura in carica un anno e si riunisce in seduta plenaria o ristretta normalmente ogni
due mesi.
Il GLI è costituito da:
\begin{itemize}
	\item Dirigente Scolastico, che lo presiede
	\item Docente referente del GLI
	\item  Docenti di sostegno
	\item Docente referente DSA
	\item  Docenti curricolari
	\item Docente referente per l'intercultura
\end{itemize}
Possono partecipare alle riunioni assistenti alla comunicazione, coordinatori di classe,
genitori ed esperti istituzionali o esterni in regime di convenzionamento con la scuola.
\subsection*{COMPITI DEL GLI}
Il GLI ha il compito di:
\begin{itemize}
	\item Rilevare i BES presenti nella scuola
	\item  Raccogliere e documentare gli interventi didattico-educativi posti in essere o da
	progettare per gli allievi con bisogni educativi speciali
	\item Confrontarsi sui casi, dare consulenza e supporto ai colleghi sulle strategie e
	metodologie nella gestione dei BES all'interno di ogni classe
	\item Rilevare e monitorare il livello di inclusività dei BES all'interno dell'Istituto
	\item Accogliere eventuali proposte di lavoro del GLH operativo nella scuola
	\item Elaborare le proposte per il Piano Annuale per l'Inclusività riferito a tutti gli alunni
	con BES da redigere al termine di ogni anno scolastico (entro il mese di giugno)
\end{itemize}
Approvato dal Consiglio d'Istituto in data 12/04/2013