\chapter{Bisogni Educativi Speciali. Approfondimenti sulla redazione del piano annuale per l'inclusività }

Oggetto: \textbf{ Bisogni Educativi Speciali. Approfondimenti in ordine alla redazione del piano
annuale per l'incisività nell'ottica della personalizzazione dell'apprendimento.\footcite{USRperLEmiliaRomagna2013a}}

Materiali per la formazione dei docenti a.s. 2013-2014.

Nell'imminenza dell'avvio del prossimo anno scolastico 2013-2014 si richiama l'attenzione dei docenti
e dei Dirigenti Scolastici dell'Emilia-Romagna su alcuni punti essenziali del ricco dibattito culturale sul
tema dei bisogni educativi speciali, apertosi con la Direttiva 27 dicembre 2012, la Circolare Ministeriale
8/2013 e la nota del Capo Dipartimento Istruzione prot.1551 del 27 giugno 2013.
Nell'invitare ad una lettura coordinata dei diversi documenti, si ritiene utile riprenderne alcuni assunti,
anche in relazione a quanto già indicato da questo Ufficio con la nota prot.6721 del 29 maggio 2013
(reperibile nel settore BES del sito Internet di questa Direzione Generale unitamente ai documenti
ministeriali citati).
A questo Ufficio sono infatti pervenute sollecitazioni ad approfondire il tema della personalizzazione
dei percorsi di insegnamento/apprendimento e delle modalità di trascrizione di tali percorsi nel piano
annuale per l'inclusività , per fornire una base comune di indirizzo alle scuole della regione.
La nota ministeriale prot.1551/2013 sottolinea che il Piano annuale per l'inclusività non va
\cit{interpretato come un piano formativo per gli alunni con bisogni educativi speciali} ma come uno
\cit{strumento di progettazione} dell'offerta formativa delle scuole \cit{in senso inclusivo, è lo sfondo ed il
fondamento sul quale sviluppare una didattica attenta ai bisogni di ciascuno nel realizzare gli obiettivi
comuni}.
Viene inoltre confermato che la redazione del PAI non deve fornire l'occasione per categorizzare le
persone ma per individuare le situazioni problematiche e le strategie per farvi fronte, qualificando le
modalità di insegnamento.
In via preliminare va richiamata l'attenzione al linguaggio usato. In pochissimo tempo sta già entrando
nell'uso comune l'espressione “ragazzi BES”, non accettabile e non rispettosa. Coloro che lavorano
nella comunicazione educativa hanno il dovere di usare un linguaggio attento alle persone. Non è
questione di formalismo nominale, è questione sostanziale: \cit{Non esiste una cosa come il lettore
innocente. Le parole sono ricevute e collocate nel contesto interpretativo che noi costruiamo leggendo
la pagina. Questo processo è definito sia dal nostro background culturale, sia dalle esperienze sia dai nostri oggettivi limiti. Di conseguenza è necessario pensare attentamente al linguaggio che usiamo}
(Roger Slee, Inclusion in practice, Educational Review 2001 ).
Il rischio di catalogare persone anziché individuare problemi ed elaborare strategie di soluzione, non
riguarda soltanto il nostro Paese. Nel dibattito internazionale, infatti, da alcuni decenni si vanno
affrontando temi e prospettive di grande rilievo culturale e professionale, che conviene richiamare
all'attenzione del personale scolastico ed educativo.
I punti approfonditi nella presente nota sono i seguenti:
\begin{description}
	\item[A]] SPUNTI DAL DIBATTITO INTERNAZIONALE
	\begin{description}
		\item[1)] Bisogni speciali o diritti specifici?
		\item [2)] l'educazione inclusiva
		\item [3)] l'insegnante inclusivo
		\item [4)] Universal Design for Learning (UDL)
	\end{description} 
	\item [B]] IL PIANO ANNUALE PER L'INCLUSIVITA' (PAI)
	\begin{description}
		\item[5)] Scopi del PAI
		\item [6)] Punti essenziali da trattare tra POF e PAI
		\item [7)] Il PAI e le risorse professionali della scuola (o delle scuole in rete)
		\item [8)] il PAI e la razionalizzazione dell'uso delle risorse (anche economiche)
	\end{description}
	\item [C]] I PIANI PERSONALIZZATI E L'ADATTAMENTO DEI CURRICOLI
	\item [D]] AFFRONTARE LE SITUAZIONI DIFFICILI
	\begin{description}
		\item[9)] Evitare la riduzione in schede
		\item [10)] Esaminare i problemi
		\begin{description}
			\item[l0.1] il problema dei comportamenti-problema
			\item[l0.2] il problema dei comportamenti auto ed etero aggressivi
		  \item[l0.3] Il problema dei problemi \cit{invisibili}
		  \item[l0.4] il problema delle difficoltà di apprendimento
		\end{description}
		\item[l1)] Curare la formazione
	\end{description}
\end{description}
\begin{description}
	\item[A]] SPUNTI DAL DIBATTITO INTERNAZIONALE
	\begin{enumerate}
		\item Bisogni speciali o diritti specifici?
		
		Nel 2000, l'UNESCO, con il {\foreignlanguage{english} Dakar Framework for Action} ha definito il principio dell'Educazione per tutti {\foreignlanguage{english}	(Education For All - EFA)}
	 ponendolo come obiettivo dell'azione dei Governi, da raggiungere entro il
		2015. Anche se l'obiettivo è ben lontano dall'essere centrato, il fatto che sia stato posto costituisce un
		elemento culturale imprescindibile nel quadro educativo attuale.
		
		Il {\foreignlanguage{english} Dakar Framework for Action} ed i documenti UNESCO ad esso collegati, trattano il tema
		dell'acquisizione, da parte di ciascuna persona, degli elementi fondamentali dell'educazione. \cit{Ogni
		persona –- bambino, ragazzo e adulto -- deve poter fruire di opportunità educative specificamente
		strutturate per incontrare i propri basilari bisogni di educazione. Questi bisogni comprendono tanto i
		contenuti essenziali dell'apprendimento (dal linguaggio orale e scritto, alla matematica alla capacità 
		di risolvere i problemi) quanto gli strumenti della conoscenza, le competenze, i valori e lo sviluppo delle
		attitudini, cioè quanto richiesto ad un essere umano per sopravvivere, sviluppare in pieno le proprie
		capacità , vivere e lavorare dignitosamente, partecipare allo sviluppo, migliorare la qualità della propria
		vita, prendere decisioni informate, continuare ad apprendere} ({\foreignlanguage{english} Dakar Framework for Action}, Art.1).
	
		Nell'ottica delineata dal documento di Dakar (e da altri documenti internazionali) il termine inglese
			\cit{needs} corrisponde quindi più al concetto di \cit{diritti} che a quello di \cit{bisogni}. I \cit{{\foreignlanguage{english}basic educational needs}} del documento di Dakar potrebbero essere concettualmente tradotti con l'espressione italiana \cit{diritti educativi essenziali}. La riflessione dei delegati UNESCO ricorda che tali diritti non sono fissi
		ma variano in relazione ai contesti, alla storia, alla cultura, alle condizioni, al divenire dell'esperienza
		umana.
	
		\'{E} quindi compito delle comunità educanti individuare per ogni persona, in ciascuno specifico
		momento della vita e nelle condizioni in cui oggettivamente essa si trova, quali siano i diritti educativi
		essenziali, elaborando le più efficaci strategie per raggiungerli.
		
		Ecco quindi che si profila la possibilità di una ulteriore possibile sfumatura del lessico: tradurre \cit{{\foreignlanguage{english}Education For All}}
		 con l'espressione \cit{educazione per tutti} può indurre in errore. La scuola
		dell'obbligo, ad esempio, potrebbe già di per sé rispondere alla richiesta.
	
		Ma, nei propri documenti, l'UNESCO ammonisce in merito al fatto che l'espansione delle opportunità 
		educative non si traduce automaticamente in uno sviluppo significativo delle persone e della società .
		Occorre che l'educazione fornisca risultati efficaci e non soltanto che consenta ad un maggior numero
		di persone di frequentare contesti scolastici 
		\begin{otherlanguage}{english}
	(“ Whether or not expanded educational opportunities will
	translate into meaningful development - for an individual or for society - depends ultimately on whether
	people actually learn as a result of those opportunities, i.e., whether they incorporate useful knowledge,
	reasoning ability, skills, and values. The focus of basic education must, therefore, be on actual learning
	acquisition and outcome, rather than exclusively upon enrolment, continued participation in organized
	programmes and completion of certification requirements. Active and participatory approaches are
	particularly valuable in assuring learning acquisition and allowing learners to reach their fullest potential.
	It is, therefore, necessary to define acceptable levels of learning acquisition for educational programmes
	and to improve and apply systems of assessing learning achievement” . The Dakar Framework for
	Action, Art.4). 
		\end{otherlanguage}
		
		Pertanto l'UNESCO richiama il dovere degli Stati non soltanto di dichiarare l'assolvimento dei diritti
		educativi essenziali ma anche di assolverli effettivamente ed in modo efficace, fornendo risultati
		documentati attraverso idonee modalità di valutazione (assessment).
		Quindi la traduzione pedagogica della definizione dell'UNESCO potrebbe essere quella dell'educazione
		per ciascuno.
		
		Questo è l'aggancio, nella nostra legislazione, con il principio della personalizzazione introdotto con la
		Legge 53/2003, preceduta fin dal 1977 dalla Legge 517 che definì sia l'inclusione dei ragazzi con
		disabilità nella scuola comune sia il principio dell'individualizzazione dell'insegnamento con nuovi
		criteri di valutazione.
		
		Può essere interessante riflettere sul fatto che anche nel Regno Unito di Gran Bretagna sussistono
		concezioni diverse degli \cit{{\foreignlanguage{english}Special Educational Needs (SEN)}}; ad esempio, le disposizioni scozzesi 
		({\foreignlanguage{english}Scottish
		Executive} 2004, aggiornate nel 2009) denominate \cit{{\foreignlanguage{english}The Additional Support for Learning Act}}
		sostituiscono la dizione \cit{bisogni educativi speciali} con quella più ampia di \cit{supporti aggiuntivi
		all'apprendimento e di modalità diverse di insegnamento} 
	(\cit{\begin{otherlanguage}{english}
		provision wich is additional to, or
		otherwise different from, the educational provision made generally for children or … young persons of
		the same age in schools
	\end{otherlanguage}}).
		
		La norma scozzese ricorda che ogni ragazzo, in qualsiasi punto della sua carriera scolastica, per
		un'ampia varietà di ragioni, può avere bisogno di supporti aggiuntivi all'insegnamento (o di
		diversificazione dell'insegnamento) ed ha diritto a riceverli.
		Tuttavia anche questa accezione non corrisponde in pieno alla logica inclusiva, in quanto comunque
		identifica un range ritenuto comune, discostandosi dal quale si ha diritto a interventi specificamente
		mirati e intensificati (quindi fa riferimento ad un concetto di normalità definito dalla media delle
		risposte e di bisogno aggiuntivo come risposta ad un discostamento significativo da tale media).
		\item L'educazione inclusiva
		
		Il programma di Dakar è corredato da una serie di note che approfondiscono e specificano i temi
		succintamente definiti nei punti del programma stesso, riportando i contenuti della discussione
		preliminare. In tali note viene ricordato che l'{\foreignlanguage{english}Education For All} è un concetto inclusivo e che è pertanto urgente \cit{includere coloro che sono socialmente, culturalmente ed economicamente esclusi. Ciò
		richiede lo sviluppo … di approcci all'apprendimento diversi, flessibili e innovativi e di contesti che
		inducano al rispetto reciproco e alla fiducia} (punto 65).
		
		L'importanza che l'UNESCO assegna all'educazione inclusiva è ribadita in molti documenti, ad esempio
		nelle \begin{otherlanguage}{english}
		Conclusions ad Recommendations of the 48th Session of the International Conference on
		Education
		\end{otherlanguage}
		 Ginevra 2008: \cit{L'educazione inclusiva è un processo continuo che mira ad offrire educazione
		di qualità per tutti rispettando diversità e differenti bisogni e abilità , caratteristiche e aspettative
		educative degli studenti e delle comunità , eliminando ogni forma di discriminazione}.
		
		L'intenso dibattito internazionale sulla scuola inclusiva sta facendo emergere riflessioni per nulla
		scontate. Una delle più rilevanti è la differenza che esiste tra la concezione definita \cit{{\foreignlanguage{english}mainstreaming}}
		e quella \cit{inclusiva}. Con il termine \cit{{\foreignlanguage{english} mainstream}} nel mondo anglosassone si indicano le scuole che,
		adottando un curricolo specifico, accettano gli allievi con disabilità o in difficoltà soltanto se sono in
		grado di seguire il curricolo con una minima assistenza.
		
		Gli {\foreignlanguage{english}Special Educational Needs} (nelle varie accezioni) ampliano l'accoglienza scolastica mantenendo
		tuttavia la distinzione tra percorsi comuni (e quindi normali) e percorsi differenziati (e quindi speciali).
		I sostenitori dell'educazione inclusiva (in tutto il mondo) ritengono che queste non siano le risposte
		corrette perché soltanto nelle scuole inclusive gli insegnanti sono tenuti a modificare i loro stili di
		insegnamento per incontrare lo stile di apprendimento di ciascun allievo.\begin{otherlanguage}{english}
		\cit{School of the future need to
		ensure theat each student receives the individual attention, accomodations and supports theat will
		result in meaningful learning} (NVPE Nevada Partnership for inclusive education).
		\end{otherlanguage} 
		
		Le istituzioni scolastiche dell'Emilia-Romagna sono quindi invitate a riflettere sul fatto che la semplice
		presenza degli alunni disabili o con DSA o in difficoltà non basta a costruire una scuola inclusiva. Non
		bastano neppure i piani educativi individualizzati o personalizzati. Occorre che il modo di insegnare e
		di valutare cambi, per poter essere \cit{curvato} sulle diverse situazioni ed in relazione a diverse difficoltà .
		
		Ecco perché la nota ministeriale prot.1551/2013 definisce il PAI non come un \cit{documento} ma come
		uno \cit{strumento} che deve contribuire ad \cit{accrescere la consapevolezza dell'intera comunità educante
		sulla centralità e la trasversalità dei processi inclusivi in relazione alla qualità dei risultati educativi}.
		Senza questo passaggio qualitativo, qualunque riflessione resterebbe sterile.
		
		Quindi, come già ricordato nella precedente nota di questo Ufficio prot.6721/2013, i primi passi da
		effettuare sono quelli relativi all'effettiva inclusività della scuola e sulle capacità inclusive dei docenti.
		Non si tratta di ripetere le rituali parole dell'inclusione; si tratta di esplorare nel merito il modo di
		organizzare la scuola e di insegnare, scuola per scuola, docente per docente, alunno per alunno.
		
		Per approfondire i percorsi di auto-riflessione delle scuole sulla propria azione in termini di inclusione,
		si ricordano due importanti strumenti di lavoro, entrambi disponibili gratuitamente sul web.
		
		Il primo strumento è di derivazione inglese ma è disponibile in lingua italiana; si tratta dell'Index per
		l'inclusione di Toni Booth e Mel Ainscow.
	
		Il secondo strumento è il software Quadis elaborato dall'Ufficio Scolastico Regionale per la Lombardia.
		Per entrambi i link sono disponibili nel settore “Bisogni educativi speciali” del sito Internet di questa
		Direzione Generale \url{www.istruzioneer.it} cui si rimanda anche per materiali di approfondimento e di
		documentazione su vari argomenti.
		\item L'insegnante inclusivo
		
		Può essere interessante riprendere un documento elaborato dalla {\foreignlanguage{english}European Agency for Development} in {\foreignlanguage{english}Special Needs Education} \cit{Profilo dei docenti inclusivi} 2012, in cui tale profilo viene puntualizzato
		in quattro valori, ciascuno dei quali declinato in un interessante elenco di indicatori, sui quali le scuole
		potrebbero aprire una attenta riflessione, proprio in relazione alla stesura del PAI.
		
		I quattro valori di riferimento condivisi dai docenti inclusivi sono:
		\begin{description}
			\item[I] (Saper) valutare la diversità degli alunni – la differenza tra gli alunni è una risorsa e una
			ricchezza
			\item [II] Sostenere gli alunni – i docenti devono coltivare aspettative alte sul successo scolastico degli
			studenti
			\item [III] Lavorare con gli altri – la collaborazione e il lavoro di gruppo sono approcci essenziali per tutti
			i docenti
			\item [IV] Aggiornamento professionale continuo – l'insegnamento è una attività di apprendimento e i
			docenti hanno la responsabilità del proprio apprendimento permanente per tutto l'arco della
			vita.
		\end{description}
		 
		L'elenco degli indicatori proposti nella pubblicazione citata è molto lungo e dettagliato. Se ne
		segnalano qui soltanto alcuni tra quelli ritenuti più significativi:
		\begin{itemize}
			\item l'integrazione scolastica è una riforma sociale non negoziabile;
			\item l'accesso all'istruzione dell'obbligo in classi comuni non basta;
			\item partecipazione significa che gli alunni devono essere impegnati in attività di apprendimento
			utili ed importanti per loro;
			\item l'inclusione si delinea in termini di presenza (accesso all'istruzione), partecipazione (qualità 
			dell'esperienza di apprendimento) e conseguimento (dei risultati educativi e del successo
			scolastico) di tutti gli studenti;
			\item la classificazione e la catalogazione degli alunni può avere un impatto negativo sulle
			opportunità di apprendimento;
			\item i docenti devono capire i percorsi tipici e atipici della crescita;
			\item gli insegnanti capaci insegnano a tutti gli alunni;
			\item i metodi di valutazione devono incentrarsi sui punti di forza di un allievo.
		\end{itemize}
		
		Per una lettura più approfondita si rimanda alla pubblicazione citata, richiamando la necessità che
		ciascun docente e ciascun collegio si soffermino a riflettere sul profilo ivi tratteggiato. Anche il link di
		questo documento è reperibile nel sito Internet di questo Ufficio indicato al precedente punto.
		
		Tra le ricerche più interessanti a livello internazionale in merito alla personalizzazione
		dell'insegnamento (e quindi sostenere l'attuazione di una reale inclusività ), si ritiene particolarmente
		utile quella denominata \cit{{\foreignlanguage{english}Universal Design for Learning}}.
		\item Universal Design for Learning (UDL)
		
		L'espressione {\foreignlanguage{english}Universal Design for Learning (UDL)} indica una modalità di progettazione e di gestione
		della pratica educativa volta ad incontrare le diverse modalità di apprendimento e le diverse condizioni
		che possono presentarsi nei diversi contesti (principalmente scolastici).
		I suggerimenti concreti, i materiali e gli strumenti informatici messi a disposizione su Internet, i molti
		esempi concreti possono risultare di grande utilità anche agli insegnanti italiani, per cui si rimanda alla
		scheda di approfondimento allegata alla presente nota (Allegato 1).
		
	\end{enumerate}
	\item [B] IL PIANO ANNUALE PER L’INCLUSIVIT\'{A} (PAI)
	\begin{enumerate}
		\setcounter{enumi}{4}
		\item Scopi del PAI
		
		Appurato dunque che il PAI non è un documento burocratico ma uno strumento di auto riflessione
		delle scuole nell'ottica del raggiungimento del successo formativo degli allievi e del benessere
		psicologico nei contesti scolastici, si può cercare di suggerire alcune, più dettagliate, piste di lavoro.
		
		Il piano annuale per l'inclusività è il coronamento del lavoro svolto in ciascun anno scolastico e
		costituisce il fondamento per l'avvio del lavoro dell'anno scolastico successivo.
		
		La redazione del PAI e l'assunzione collegiale di responsabilità in relazione alla sua stesura,
		realizzazione e valutazione ha lo scopo di:
		\begin{description}
			\item[--] garantire l'unitarietà dell'approccio educativo e didattico dell'istituzione scolastica
			\item[--] garantire la continuità dell'azione educativa e didattica anche in caso di variazione dei docenti
			e del dirigente scolastico (continuità orizzontale e verticale)
			\item[--] consentire una riflessione collegiale sulle modalità educative e sui metodi di insegnamento
			adottati nella scuola, arrivando a scelte basate sull'efficacia dei risultati in termini di
			comportamento e di apprendimento di tutti gli alunni
			\item[--] individuare le modalità di personalizzazione risultate più efficaci in modo da assicurarne la
			diffusione tra gli insegnanti della scuola e tra scuole diverse
			\item[--] raccogliere i piani educativi individualizzati ed i piani didattici personalizzati in un unico
			contenitore digitale che ne conservi la memoria nel tempo come elemento essenziale della
			documentazione del lavoro scolastico, non più soggetta alle complessità di conservazione dei
			documenti cartacei
			\item[--] inquadrare ciascun percorso educativo e didattico in un quadro metodologico condiviso e
			strutturato, per evitare improvvisazioni, frammentazioni e contraddittorietà degli interventi
			dei singoli insegnanti (ed educatori)
			\item[--] evitare che scelte metodologiche improvvide, non documentate o non scientificamente
			supportate, effettuate da singoli insegnanti compromettano lo sviluppo delle capacità degli
			allievi; si ricorda che la libertà di insegnamento sancita dalla Costituzione non fornisce un via
			libera assoluto ed acritico verso qualsiasi scelta (o anche verso il nulla fare). La libertà di
			insegnamento va correttamente intesa come responsabilità di insegnamento: il docente è
			libero di scegliere tra le strategie più efficaci quelle ritenute idonee garantire il successo di
			ciascun allievo. Non si possono scegliere strade che non diano risultati efficaci e documentati
			\item[--] fornire criteri educativi condivisi con le famiglie (al di là della necessità di condividere ciascun
			PEI o PDP con le famiglie degli allievi cui si riferiscono, vi è la necessità di condividere con tutte
			le famiglie i criteri di intervento e di azione per la personalizzazione, proprio perché questa è
			una necessità che potrebbe presentarsi a qualunque allievo e che potrebbe richiedere la
			collaborazione attiva di tutta la comunità educante).
		\end{description}
		Procedere ad una \cit{conta} degli alunni problematici, non avviando contestualmente le azioni
		necessarie ad affrontare i problemi rilevati ed evitando riflessioni critiche sulle modalità educative e di
		insegnamento non portano alla redazione di un PAI (non è il titolo che fa il documento ma il contenuto).
		\item Punti essenziali da trattare tra POF e PAI
		
		\'{E} necessario che le scuole inclusive definiscano nei loro documenti di programmazione (POF e PAI)
		almeno i seguenti punti:
		\begin{description}
			\item[--] la definizione, su base scientificamente validata e collegialmente condivisa, delle modalità di
			identificazione delle necessità di personalizzazione dell'insegnamento
			\item[--] la definizione dei protocolli per la valutazione delle condizioni individuali e per il monitoraggio
			e la valutazione dell'efficacia degli interventi educativi e didattici
			\item[--] I criteri di stesura dei piani personalizzati, della loro valutazione e della modifica
			\item[--] la definizione del ruolo delle famiglie (dalla valutazione alla programmazione) e delle modalità 
			di mantenimento dei rapporti scuola/famiglia in ordine allo sviluppo delle attività 
			educative/didattiche personalizzate; una forte alleanza educativa con le famiglie è condizione
			essenziale per la riuscita dei percorsi di personalizzazione (così come dell'educazione e
			dell'insegnamento tout court)
			\item[--] la definizione delle responsabilità dei vari attori del processo (dirigente scolastico, docenti
			referenti delle varie tematiche, docenti di classe, docenti di sostegno, educatori, insegnanti
			tecnico-pratici e di laboratorio, personale ATA, …) e delle collaborazioni interistituzionali (ASL,
			Comune, Provincia, privato sociale, \ldots)
			\item[--] modalità di tutela della riservatezza e della privacy, ricordando comunque che fruire di percorsi
			personalizzati non è una vergogna da nascondere e che nella scuola inclusiva questa
			condizione dovrebbe essere prassi comune e non eccezione; se la personalizzazione fosse
			prassi comune, le famiglie porrebbero meno problemi di privacy in quanto non avrebbero
			ragione di temere lo stigma sociale della diversità .
		\end{description}
			\item Il PAI e le risorse professionali della scuola (o delle scuole in rete)
			
			I documenti ministeriali sui bisogni educativi speciali invitano le scuole alla valorizzazione delle risorse
			professionali di cui le scuole stesse dispongono (in termini di competenza, innanzi tutto) affinché
			possano essere adeguatamente valorizzate e messe a disposizione di tutto il corpo docente.
			Si indica come esempio da consolidare e diffondere il progetto \cit{Anagrafe professionale degli
			insegnanti di sostegno} avviato nell'a.s.2007-2008 su iniziativa del GLIP di Bologna e dell'Istituto
			Comprensivo di Ozzano Emilia (BO). Tale anagrafe raccoglie i dati volontariamente forniti dai docenti,
			i quali mettono a disposizione delle scuole del territorio le proprie competenze per consulenze
			didattiche. Le scuole possono avvalersi di tali docenti e delle loro competenze chiedendo
			direttamente il \cit{prestito professionale} attraverso il portale on-line dell'anagrafe
			\url{(http://www.icozzanoemilia.it/joomla/index.php?option=com_content&view=article&id=82&Itemid=103)}.
			Lo scambio di docenti può avvenire soltanto per via istituzionale, tra scuola e scuola; l'intervento
			dell'insegnante esterno avviene a supporto alla programmazione e non comporta alcun intervento
			diretto sugli allievi. Il tempo è aggiuntivo a quello di insegnamento ed è a carico finanziario della scuola
			che lo richiede.
			
			Questo esempio, attivato nella più vasta provincia dell'Emilia-Romagna, ha caratteristiche che lo
			rendono replicabile sia a livello di altre province sia di reti di scuole.
			\item Il PAI e la razionalizzazione dell'uso delle risorse (anche economiche)
			
			In momenti di ritrazione complessiva delle risorse economiche, anche per gli Enti Locali, è
			particolarmente opportuno che le istituzioni scolastiche si facciano promotrici di azioni di
			razionalizzazione di tali risorse attraverso la promozione ed il consolidamento dei contatti inter-
			istituzionali.
			
			Uno dei \cit{luoghi} inter-istituzionali in cui le Istituzioni Scolastiche dovrebbero essere presenti sono i tavoli dei cosiddetti \cit{Piani di zona per la salute e il benessere sociale}, strumento principale delle politiche sociali, volto a costruire un sistema integrato di interventi e servizi (Integrato, perché deve mettere in relazione servizi che si offrono in strutture, servizi domiciliari, servizi territoriali, misure economiche, prestazioni singole, iniziative non sistematiche, sia che siano rivolte alla singola persona
			sia alla famiglia; Integrato, perché deve coordinare politiche sociali, sanitarie, educative, formative,
			del lavoro, culturali, urbanistiche e abitative, e cioè: come, dove, e chi il sistema nel suo complesso
			assiste, si prende cura, riabilita, educa, forma, orienta e inserisce al lavoro, offre occasioni di cultura e
			di socialità, offre una città e un'abitazione vivibile e adeguata; Integrato, infine perché deve far
			collaborare e lavorare, in modo coordinato ed efficace per i cittadini, soggetti istituzionali e non,
			pubblici e privati
			\url{http://sociale.regione.emilia-romagna.it/entra-in-regione/piano-sociale-e-sanitario/faq-sui-piani-di-zona-1}).
			
			Per l'Emilia-Romagna il riferimento normativo è la Legge Regionale n.2/2003. I piani di zona elaborati
			nel 2011 sono reperibili al link sotto indicato.
			\cit{http://sociale.regione.emilia-romagna.it/entra-in-regione/piano-sociale-e-sanitario/link-piani-di-zona}
			
			Si ricorda inoltre che è opportuno che le scuole valutino le possibilità di accedere in rete ai
			finanziamenti derivanti dalle Leggi Regionali, in modo particolare la Legge 30 giugno 2003 n.12 \cit{Norme
			per l'uguaglianza delle possibilità di accesso al sapere per ognuno e per tutto l'arco della vita,
			attraverso il rafforzamento dell'istruzione e della formazione professionale, anche in integrazione tra
			loro} e la Legge 8 agosto 2001 n.26 \cit{Diritto allo studio ed all'apprendimento per tutta la vita}.
			
			Vale anche sottolineare il fatto che – nonostante le difficoltà economiche – nel territorio regionale
			sono attive importanti Fondazioni disponibili a supportare le scuole (soprattutto se in rete) a fronte di
			progetti circostanziati, specifici, ben documentati. Il supporto fornito alle scuole terremotate dai
			privati è stato un esempio formidabile di come l'intreccio tra un territorio eticamente sensibile e la
			dedizione del personale scolastico possa produrre effetti straordinari.
			
			Le Fondazioni possono fornire servizi, anziché denaro. Fra altre, accenniamo al ruolo importante svolto
			da tempo dalla Fondazione ASPHI (\url{www.asphi.it} Adattamento e Sviluppo di Progetti per ridurre
			l'handicap mediante l'Informatica). ASPHI sostiene la ricerca delle scuole (non soltanto in Emilia-
			Romagna) nel campo dell'informatica per la disabilità , per i disturbi specifici di apprendimento e per
			le strategie di insegnamento personalizzato. Ad esempio sono in corso una sperimentazione sulla
			didattica della matematica e sulla discalculia (Progetto Per-Contare) scientificamente coordinata
			dall'Università di Modena e Reggio Emilia e con il supporto della Compagnia di San Paolo di Torino, un
			progetto di simulazione di difficoltà in ambienti virtuali (Progetto Come-Se) e il Progetto Touch for
			Autism per la sperimentazione di tecnologie touch per supportare e sostenere persone con autismo.
			Si rimanda al sito di ASPHI per la visione complessiva dei progetti in corso.
	\end{enumerate}
	\item [C] I PIANI PERSONALIZZATI E L'ADATTAMENTO DEI CURRICOLI
	
	Nella scuola italiana dovrebbe esservi ormai una consolidata tradizione di stesura di piani
	personalizzati. Nella pratica effettiva, tuttavia, si riscontrano notevoli discrepanze tra scuola e scuola
	e tra piano e piano. Le difficoltà che le scuole stanno incontrando nella stesura di piani didattici
	
	personalizzati per gli allievi con disturbi specifici di apprendimento, sono una testimonianza
	indiscutibile del fatto che la capacità di lavorare sul singolo allievo e di descrivere puntualmente il
	processo educativo e didattico non è ancora acquisita dalla generalità degli insegnanti e dei consigli di
	classe.
	Ricordiamo quindi che una programmazione personalizzata contiene:
	\begin{description}
		\item[--] la descrizione accurata della situazione dell'allievo, partendo dai suoi punti di forza, dalle
		abilità e dalle capacità presenti. La descrizione deve essere sinottica, riassunta in tabelle (che
		non sono griglie) e poi eventualmente spiegata con maggiore dettaglio
		\item[--] la descrizione dello stile di apprendimento dell'allievo per adattarvi lo stile di insegnamento
		\item[--] l'individuazione delle aree di vocazionalità , cioè degli interessi e delle predisposizioni su cui si
		può fare leva per facilitare l'apprendimento
		\item[--] la segnalazione di eventuali difficoltà o problemi attraverso accurate descrizioni di
		comportamenti osservabili e dei contesti in cui si realizzano, anch'essi descritti con precisione;
		\item[--] la descrizione delle situazioni e delle condizioni che favoriscono le performance positive
		dell'allievo quanto quelle che ne condizionano negativamente i risultati
		\item[--] l'individuazione degli ambiti di lavoro per l'anno scolastico, degli obiettivi, dei contenuti e dei
		metodi per raggiungerli
		\item[--] le modalità di verifica e di valutazione dell'efficacia del lavoro svolto e l'eventuale modifica
		degli aspetti che non hanno fornito i risultati sperati (è essenziale comprendere che espressioni
		del tipo \cit{adeguato progresso} o altre generiche formulazioni non sono significative se non
		accompagnate da precise indicazioni sul cosa, sul quanto, sul come e sul perché e rispetto a
		quali standard previsti) 
	\end{description}

	Nella riflessione collegiale che gli insegnanti devono effettuare per la personalizzazione del curricolo è
	innanzi tutto necessario:
	
	\begin{description}
		\item[--] identificare i contenuti essenziali delle discipline per garantire la validità del corso di studi e
		del diploma rilasciato alla fine della scuola secondaria di II grado (ovviamente se non si tratta
		di piano differenziato di cui alla Legge 104/92);
		\item[--] scegliere obiettivi realistici (cioè che l'alunno possa effettivamente raggiungere);
		\item[--] scegliere obiettivi significativi (cioè che abbiano rilevanza per lui, anche in vista della vita
		adulta);
		\item[--] scegliere obiettivi razionali, di cui l'alunno possa comprendere e condividere il significato e la
		rilevanza;
		\item[--] definire un curricolo funzionale, cioè che miri ai diritti educativi essenziali, per la qualità della
		vita presente e futura dell'allievo
	\end{description}
 \item [D] AFFRONTARE LE SITUAZIONI DIFFICILI
 Questo Ufficio è consapevole che il personale scolastico necessita di concrete indicazioni su come
 impostare correttamente l'attività , in modo particolare su come procedere ad identificare le situazioni
 più complesse, sulle quali intervenire in via prioritaria con la programmazione della personalizzazione
 educativa e/o didattica.
 Si propongono pertanto all'attenzione e alla riflessione delle scuole alcuni spunti in ordine
 all'individuazione dei problemi sui quali teoricamente si potrebbe essere chiamati ad intervenire e ad
 alcune regole basilari da seguire.
 \begin{enumerate}
 	\setcounter{enumi}{8}
 	\item Evitare la riduzione in schede
 	
 	Nell'editoria e attraverso vari canali comunicativi si vanno diffondendo con grande velocità proposte
 	di \cit{schede} o \cit{cartelle} destinate alla catalogazione delle condizioni per cui un alunno può avere dei
 	bisogni educativi speciali. La caratteristica di questi prodotti è che iniziano sempre non con la
 	descrizione di un problema ma con i dati personali di un alunno, individuato per nome cognome classe
 	etc.
 	
 	Questo processo di riduzione delle persone a problemi è l'aspetto culturalmente ed educativamente
 	più insidioso che questo Ufficio sta rilevando.
 	
 	Ricordiamo quanto ben definito nella letteratura pedagogica mondiale: se si valutano le persone si
 	devono individuare i punti di forza, gli aspetti che possono fornire il fulcro della leva educativa con cui
 	\cit{sollevare} ciascun allievo ai massimi livelli di competenza per lui ipotizzabili. Non si producono lunghe
 	liste di problemi, di incapacità , di \cit{non sa fare}, che determinano effetti paralizzanti e non stimolanti
 	negli allievi e nelle loro famiglie e conducono a disistima di sé o –- al contrario –- ad una accentuazione
 	della conflittualità con la scuola. L'unico contesto in cui è lecito e necessario indicare tutti i
 	disfunzionamenti individuati quello di una segnalazione per la ASL (tramite la famiglia, ovviamente). In
 	questo caso è necessario che gli specialisti che ricevono il documento possano farsi un quadro
 	dettagliato delle condizioni che hanno messo in allarme la scuola e quindi possano tenerne conto per
 	l'avvio delle proprie procedure di valutazione. Questa necessità va ben spiegata alle famiglie nel
 	momento in cui si consegni loro un documento siffatto, affinché non si trovino di fronte ad una
 	prospettiva devastante sui propri figli.
 	
 	Questo Ufficio ha preso visione di alcuni tipi di schede pre-impostate, rilevando che esse spesso
 	presentano aspetti non scientifici di valutazione, con voci del tipo \cit{l'alunno non dimostra interesse}
 	senza precisare a cosa l'allievo non mostra interesse e a cosa invece si interessa (perché qualcosa
 	sicuramente ci sarà ) e quali possono essere gli aspetti su cui si potrebbe far leva per interessarlo;
 	magari la scuola che frequenta è davvero poco interessante (e ciò sarebbe problema della scuola e non dell'allievo; gli allievi molto dotati, ad esempio, possono essere poco interessati ad una scuola che non
 	propone sfide alla loro altezza).
 	
 	Si raccomanda alle scuole di evitare l'uso di strumenti siffatti. Con ciò non si vuole negare la necessità 
 	che i problemi vengano descritti, ma la descrizione deve essere basata su fatti e circostanziata. \'{E}
 	necessario che la descrizione delle difficoltà non sia effettuata in modo da apparire come un giudizio
 	su ciò che la persona è (un comportamento maleducato deve essere rilevato e sanzionato in quanto
 	comportamento; ma non significa che l'alunno possa essere apostrofato o descritto con espressioni
 	del tipo \cit{è maleducato}). La distinzione è importante: se una persona viene definita come maleducata
 	non ha altra possibilità continuare ad essere ciò che è; se la persona viene aiutata a comprendere che
 	un determinato comportamento non è adeguato può anche essere aiutata e motivata a cambiarlo.
 	\item Esaminare i problemi
 	
 	Le scuole ci chiedono: come definire i percorsi personalizzati partendo dai problemi e non dagli allievi?
 	
 	Il primo punto che si ritiene utile suggerire riguarda il fatto che i problemi o le difficoltà dentro la scuola
 	hanno sempre una matrice comunitaria e quindi vanno esaminati considerando non soltanto il
 	comportamento di un singolo allievo ma anche definendo le condizioni del contesto in cui il problema
 	si manifesta, ivi comprese le modalità che i vari insegnanti adottano per farvi fronte.
 	
 	Forniamo alcuni esempi di problemi che tutte le scuole si trovano oggi ad affrontare, proponendo
 	percorsi di riflessione e proposte di possibili interventi. Quanto di seguito indicato ha uno scopo
 	esclusivamente esemplificativo, poiché le modalità giuste di intervento in situazione possono essere
 	identificate soltanto da coloro che vi si trovano effettivamente e che ne portano la responsabilità.
 	\begin{description}
 		\item[10.1]Il problema dei comportamenti-problema
 		
 		Tutte le scuole segnalano situazioni in cui alcuni alunni possono “esplodere” in manifestazioni eclatanti
 		(urla, fughe o allontanamenti da qualsiasi contesto, distruzione di cose, ecc.).
 		
 		Questo tipo di manifestazioni può presentarsi per svariate ragioni o condizioni, sia di disabilità (quali
 		l'autismo o l'ADHD) sia per condizioni educative o sociali, quali ad esempio quelle di ragazzi non
 		abituati a ricevere dinieghi, che crescono in contesti educativi di tipo \cit{giustificazionista} per cui tutto
 		è permesso, che non conoscono limiti o confini al proprio volere e non accettano il ruolo degli adulti.
 		
 		Queste possono sembrare condizioni non assimilabili e non confrontabili, e non lo sono se considerate
 		dal punto di vista \cit{nosografico}.
 		
 		Tuttavia vi sono criteri di azione didattica comuni a tutte le condizioni citate (ovviamente con gli
 		opportuni adattamenti alle specifiche condizioni: trattare con un allievo con disabilità intellettive non è uguale a trattare con un ragazzo in grado di comprendere. Pure entrambi devono imparare a
 		comportarsi in modo adeguato).
 		
 		Il principale criterio da definire e da attuare è quello di una fortissima alleanza di tutto il mondo adulto,
 		che è chiamato ad individuare una linea comune da \cit{tenere} a qualsiasi costo. Anche i ragazzi autistici
 		con gravi disabilità cognitive hanno bisogno di comprendere cosa significa \cit{NO} e devono imparare ad
 		accettarlo, comprendendo che il NO rimarrà tale qualsiasi comportamento venga messo in atto.
 		
 		\'{E} fondamentale che tutto il mondo adulto condivida le scelte effettuate e che non sia possibile
 		all'allievo \cit{sfondare} con un adulto o con l'altro la \cit{linea educativa} definita, sia a casa, sia a scuola,
 		sia negli altri ambienti frequentati.
 		
 		Se queste modalità definiscono una linea di risposta efficace per ragazzi con disabilità , tanto più esse
 		risultano rilevanti per alunni che hanno soltanto problemi educativi, di mancanza del limite e di
 		onnipotenza del desiderio; occorre comprendere che la persistenza di questi comportamenti oltre i
 		primi anni di vita (cui appartengono come caratteristica evolutiva) è sempre determinata dal fatto che
 		il contesto ne conferma l'efficacia (magari senza saperlo). Occorre che il contesto sia strutturato in
 		modo da rendere non premianti questi comportamenti.
 		
 		Nel caso dell'autismo il comportamento-problema è spesso legato ad una comunicazione assente o
 		inefficace; pertanto oltre ad insegnare comportamenti adeguati è prioritario fornire strumenti di
 		comunicazione, almeno funzionale.
 		
 		Tuttavia non sono soltanto gli autistici privi di linguaggio che hanno problemi di comunicazione.
 		
 		Anche molti ragazzi che sanno parlare, purtroppo non conoscono il tipo di linguaggio che trascrive le
 		emozioni e i sentimenti e ne consente una corretta gestione (si è parlato per anni di analfabetismo
 		affettivo e dei pericoli che esso rappresentava: oggi i tanti casi di aggressività grave per futili motivi o
 		per un diniego dimostrano nei fatti quanto i segnali di allarme di un decennio fa fossero fondati).
 		
 		\'{E} quindi importante che il piano annuale per l'inclusività non sia formato da un elenco di tutti i ragazzi
 		che hanno comportamenti problematici ma dall'analisi dei diversi tipi di comportamenti e delle
 		condizioni che li rendono \cit{efficaci}, pur nella loro negatività . Quindi sulla definizione dei criteri generali
 		di intervento (che saranno poi dettagliati nei singoli percorsi personali).
 		\item[10.2]Il problema dei comportamenti auto ed etero aggressivi
 		
 		Vale innanzi tutto ricordare alle scuole di evitare il \cit{fai da te} nei casi in cui i comportamenti esplosivi
 		di alcuni allievi presentino anche aspetti di distruttività verso le cose e di aggressività verso le persone
 		(o verso lo stesso allievo).
 		
 		Lo scatenarsi di manifestazioni aggressive particolarmente intense (non soltanto fisiche ma anche
 		verbali) è sempre indice di problemi che vanno affrontati con consapevolezza e con modalità di
 		documentata efficacia. Anche le forme di aggressività a distanza, mediate dai social network (quali il
 		cyberbullismo) vanno prima affrontate come formazione professionale del personale docente,
 		sostenuta da personale qualificato ed esperto, e soltanto dopo come possibile analisi delle situazioni
 		personali degli allievi e dei climi relazionali delle classi.
 		
 		\'{E} quindi necessario che le scuole, come primo passo, concordino con le Neuropsichiatrie infantili o
 		con le cattedre universitarie, delle forme di supporto e di aggiornamento (anche per reti di scuole) per
 		comprendere la vastità e la complessità delle condizioni personali, familiari, sociali e di salute, che
 		possono essere alla base di comportamenti particolarmente aggressivi. Questo Ufficio ha sempre
 		riscontrato la massima disponibilità sia nei servizi socio sanitari sia nelle Università ed è comunque a
 		disposizione per fornire supporto e stabilire collegamenti.
 		
 		Il primo consiglio è che gli insegnanti affinino la propria \cit{capacità di vedere} i comportamenti e di
 		interpretare le modalità relazionali nelle classi; esiste uno sguardo pedagogico \cit{uno sguardo e un
 		ascolto attivi, loquaci, interroganti, comprendenti, partecipanti, non giudicanti, non intrusivi, non
 		\cit{captivi}, non capziosi, non repulsivi} (\url{www.pedagogia.it}) che va coltivato attraverso la formazione
 		continua dei docenti (che è dovere personale di ciascuno).
 		
 		Esistono diverse modalità di intervento di sperimentata efficacia per intervenire in situazioni
 		\cit{esplosive}; ovviamente l'individuazione della modalità più opportuna ed efficace può essere definita
 		soltanto nel contesto specifico (non soltanto in relazione all'alunno che manifesta il comportamento
 		problema ma anche dei compagni e degli insegnanti). Ricordiamo tra le modalità più efficaci quelle
 		basate su tecniche di scarico della tensione emotiva (ad esempio l'educazione fisica e la pratica
 		sportiva sono un ottimo canale per convogliare le tensioni in comportamenti socialmente accettabili e
 		costruttivi; per imparare a rispettare le regole e il ruolo degli adulti; così anche le tecniche legate
 		all'espressione teatrale e alla drammatizzazione che possono ottenere gli stessi risultati con modalità 
 		totalmente diverse). Gli aspetti positivi di questo tipo di tecniche consistono nel fatto che affermano e
 		creano comunità , non riguardando un solo allievo; che sono educative e didattiche e rientrano nei
 		curricoli scolastici. Vi sono inoltre tecniche di \cit{prevenzione della risposta}, cioè che aiutano a
 		comprendere quali aspetti o situazioni fungano da innesco dei comportamenti. Ciò consente da un lato
 		di aiutare tutti a comprendere come evitare le situazioni a rischio fornendo contestualmente chiavi di
 		lettura per razionalizzare e portare a consapevolezza. Non è ovviamente possibile in questo contesto
 		elencare tutte le possibili modalità di intervento. Gli esempi forniti sono da considerarsi soltanto come
 		esemplificativi di possibili piste di lavoro (e di formazione).
 		
 		Nella revisione dei Piani dell'Offerta Formativa e nella stesura dei Piani Annuali per l'inclusività è
 		indispensabile che i docenti e le famiglie condividano gli indicatori attraverso cui possono essere
 		analizzati i comportamenti aggressivi, anche chiedendo la collaborazione degli specialisti dei
 		poliambulatori del territorio. Anche le modalità di contenimento di eventuali atti aggressivi verso se
 		stessi o verso gli altri vanno concordate con gli psicologi o i neuropsichiatri ASL e con le famiglie, onde
 		evitare danni alle persone, forti turbamenti nei compagni, e aggravamento delle condizioni individuali
 		degli alunni.
 		
 		Ricordiamo che le diverse modalità di intervento condividono il fatto che affrontare i comportamenti
 		problema richiede anche la definizione di progetti sulla qualità della vita delle persone, di modifica
 		delle relazioni nei contesti e di acquisizione di consapevolezza sulle modalità relazionali di ciascuno, di
 		sostegno alle capacità adattive dei singoli, di comprensione dei significati delle \cit{crisi}, di prevenzione e di riduzione del danno.
 		\item[10.3] Il problema dei problemi \cit{invisibili}
 		
 		Occorre che la scuola faccia attenzione non soltanto ai comportamenti esplosivi o clamorosi ma anche
 		al loro contrario; parliamo di ragazzi troppo chiusi in se stessi, con scarsa socializzazione, che passano
 		troppo tempo al computer (anche di notte, a scapito del sonno), che stanno troppo chiusi da soli nella
 		loro camera, eccetera.
 		
 		\'{E} chiaro che molti di questi comportamenti possono essere evidenziati soltanto a casa; a scuola se ne
 		vedono le conseguenze (ad esempio della carenza di sonno). Per questo è necessario ribadire la forte
 		alleanza che deve stringersi tra scuola e famiglia.
 		
 		Sono infatti in aumento le condizioni di depressione, le fobie e le crisi di panico, in taluni casi di ritiro
 		dal mondo. Il profilo adolescenziale che i giapponesi chiamano hikikomori (giovani che non escono
 		dalle loro camere per anni e vivono esclusivamente attraverso Internet, si creano identità diverse e
 		vivono vite soltanto virtuali) è soltanto una delle punte estreme di questa tipologia di condizioni.
 		
 		L'approdo all'autoreclusione non è mai repentino ma è sempre preceduto da fasi di passaggio che il
 		mondo adulto deve essere capace di vedere e di affrontare per tempo.
 		
 		Tra i problemi che possono rimanere a lungo invisibili vi sono i comportamenti alimentari (anoressia e
 		bulimia); anch'essi richiedono attenzione e interventi ad alta specializzazione, in considerazione del
 		fatto che tali comportamenti possono comportare anche un serio rischio per la vita.
 		La scuola può –- da un lato –- aiutare ad individuare precocemente le condizioni di rischio e dall'altro
 		contribuire alla terapia seguendo scrupolosamente le indicazioni dei sanitari.
 		
 		Con il progetto \cit{Far scuola ma non a scuola} questo Ufficio finanzia progetti delle scuole volti a
 		\cit{mantenere il contatto} con i ragazzi che non escono di casa (o non riescono ad entrare a scuola).
 		
 		Le segnalazioni di questi ragazzi sono sempre effettuate da psicologi o neuropsichiatri che (unitamente
 		alla famiglia) espressamente richiedono l'intervento della scuola.
 		\item[10.4] Il problema delle difficoltà di apprendimento
 		Nelle scuole si danno situazioni di alunni i quali, pur non rientrando né nella Legge 104/92 né nella
 		Legge 170/2010, evidenziano rilevanti difficoltà di apprendimento, soprattutto quando si passa dagli
 		apprendimenti strumentali a più complessi livelli di elaborazione e di astrazione.
 		In alcuni casi questi alunni possiedono una diagnosi medica o una segnalazione psicologica che
 		evidenzia problemi cognitivi (come nel cosiddetto funzionamento intellettivo limite) e/o esecutivi
 		(come nel caso delle disprassie).
 		
 		\'{E} ovvio che, in presenza di una segnalazione specialistica, la scuola deve attivare i percorsi di
 		personalizzazione dell'insegnamento e delle modalità valutative utili ad incontrare le specifiche
 		necessità di questi allievi, pur non potendosi prevedere per loro, al momento, modifiche tali da
 		compromettere il valore legale del diploma al termine della scuola secondaria di II grado.
 		
 		In altri casi la scuola non ha documentazioni psicologiche o mediche che possano supportare
 		l'individuazione della natura delle difficoltà riscontrate.
 		
 		Ciò non elimina tuttavia la necessità che la scuola intervenga, e lo farà basandosi su valutazioni
 		pedagogico-didattiche.
 		
 		Come già più volte ricordato, le scuole devono evitare di stilare lunghi elenchi di competenze che
 		l'alunno non possiede. \'{E} necessario capire attraverso quali strade si possano ottenere risultati
 		favorevoli.
 		
 		Per tutti gli allievi con difficoltà intellettive vi sono criteri che facilitano l'apprendimento (criteri che
 		ovviamente vanno adattati all'età , alle condizioni individuali e a ciò che deve essere appreso).
 		
 		In linea molto generale ricordiamo che gli alunni con problemi di apprendimento:
 		\begin{description}
 			\item[--] sono facilitati dalla strutturazione dei contesti e della presentazione dei contenuti
 			dell'apprendimento affinché risultino chiaramente percepibili, scanditi in fasi, riassunti in
 			cartelloni, tabelle, mappe, schemi
 			\item[--] sono facilitati se la strutturazione delle presentazioni e dei compiti è tale da guidarli ad una
 			corretta percezione di ciò che veicola l'informazione principale
 			\item[--] sono facilitati dalla presentazione di pochi obiettivi per volta, presentati a piccole unità , di
 			frequente ripetute con leggere varianti
 			\item[--] devono essere guidati alla generalizzazione, all'astrazione e alla formazione dei concetti
 			attraverso passi ben calibrati e consolidati con cura
 			\item[--] sono facilitati se affrontano argomenti concreti che ben conoscono e che li interessano e di cui
 			possono fare esperienza concreta
 			\item[--] devono essere sostenuti nei passi per cui il supporto è necessario e soltanto per il tempo
 			necessario; la conquista dell'autonomia è il primo degli obiettivi
 			\item[--] devono essere elogiati per ogni piccola conquista, fortificando l'autostima e il ruolo sociale
 			anche nei confronti dei compagni
 			\item[--] devono essere sostenuti nell'uso del linguaggio, insegnando loro sia l'uso funzionale del
 			linguaggio nei vari contesti sia l'espressione dei sentimenti, delle emozioni, delle sensazioni e
 			delle riflessioni personali
 			\item[--] devono imparare a trasferire competenze tra campi affini, a stabilire relazioni d'ordine e di
 			equivalenza, a comprendere i rapporti di causa/effetto, le successioni temporali,
 			l'organizzazione spaziale e le sue rappresentazioni;
 			\item[--] devono essere guidati a potenziare le capacità di attenzione e di memoria a breve e lungo
 			termine;
 			\item[--] devono essere guidati a comprendere la complessità delle relazioni umane, di chi fidarsi, a chi
 			chiedere aiuto in caso di bisogno, come comportarsi in situazioni di rischio (gli adolescenti con
 			difficoltà intellettive sono più esposti ai rischi di abuso e di violenze)
 		\end{description}
 	\end{description}
	Si ritiene utile ricordare un metodo di provata efficacia per intervenire sugli alunni con difficoltà di
	apprendimento è il Metodo Feuerstein, che non lavora sui contenuti e non si occupa di ciò che gli
	alunni non sanno fare ma contiene sperimentati strumenti di valutazione che consentono di
	individuare i punti di forza e le modalità di apprendimento degli allievi, in modo tale da potenziare il
	loro sviluppo intellettivo con maggiore efficacia; lo scopo del Metodo Feuerstein consiste
	nell'individuare le risorse che la persona possiede, insegnando come attingervi, come potenziarle e
	come indirizzarle per imparare a imparare.
	
	Il metodo Feuerstein è stato oggetto di lunghi studi da parte dell'IRRE Emilia-Romagna, che ne ha
	curato anche la divulgazione attraverso convegni e incontri di formazione. IRRE Emilia-Romagna è stata
	la prima istituzione pubblica in Italia autorizzata alla formazione e alla pubblicizzazione del metodo
	
	(\url{http://kidslink.bo.cnr.it/irrsaeer/feuerstein/infogenerali2007.pdf}).
	
	La formazione per l'apprendimento di questo metodo è molto severa e attuata soltanto da centri
	autorizzati dalla struttura centrale israeliana. Vi sono in Emilia-Romagna centri e persone qualificati
	che potranno essere individuati con una semplice ricerca in rete. 
	\item Curare la formazione 
	
	Come punto conclusivo del presente documento si ritiene di dover ancora una volta richiamare la
	necessità che ogni docente, ogni dirigente scolastico, ogni collegio dei docenti, assegnino la priorità 
	alla formazione continua.
	Vale ricordare l'art. 282 del Testo Unico delle disposizioni normative in materia di istruzione (Decreto
	Legislativo 16 aprile 1994 n. 297) \cit{1. L'aggiornamento è un diritto-dovere fondamentale del personale
	ispettivo, direttivo e docente. Esso è inteso come adeguamento delle conoscenze allo sviluppo delle
	scienze per singole discipline e nelle connessioni interdisciplinari; come approfondimento della
	preparazione didattica; come partecipazione alla ricerca e alla innovazione didattico-pedagogica.
	2. L'aggiornamento si attua sulla base di programmi annuali nell'ambito del circolo didattico,
	dell'istituto, del distretto e con iniziative promosse sul piano regionale e nazionale\ldots
	
	3. I circoli didattici e gli istituti,…, favoriscono con l'organizzazione di idonee attrezzature e di servizi,
	l'autoaggiornamento e l'aggiornamento, anche in relazione alle esigenze risultanti dalla valutazione
	dell'andamento didattico del circolo o dell'istituto e di eventuali iniziative di sperimentazione}.

	Pur nella consapevolezza dei problemi finanziari legati all'attuale situazione economica, si ritiene
	indispensabile che le risorse a disposizione siano investite nella formazione iniziale e continua degli
	insegnanti. A volte le scuole investono in strutture impegnative (quali complesse attrezzature
	laboratoriali) che poi sono poco o malamente utilizzate in quanto gli insegnanti non sono pienamente
	formati al loro utilizzo.
	
	In ogni punto di questa nota si è richiamata la necessità di formazione continua, cui questo Ufficio
	dedica costante attenzione: nel sito Internet già richiamato all'inizio sono pubblicati (settore BES)
	svariati materiali di rilevante utilità per i docenti. Molte sono le iniziative di formazione organizzate
	dall'Ufficio Scolastico Regionale, anche in collaborazione con altre istituzioni, Fondazioni e
	organizzazioni di volontariato, cui i docenti e i dirigenti scolastici possono partecipare gratuitamente.
	Per tenersi costantemente aggiornati su queste iniziative è necessario consultare direttamente i siti
	Internet sia della Direzione Generale sia degli uffici per ambito territoriale provinciale.
	
 \end{enumerate}
\item Conclusione

 Non è ovviamente possibile, in un documento di indirizzo come la presente nota, approfondire
 ulteriormente gli esempi e le riflessioni.
 
 Si confida che quanto fornito con le note di questo Ufficio risulti utile per sostenere e indirizzare la
 riflessione sulla redazione dei piani educativi e del piano annuale per l'inclusività , così come le revisioni
 dei POF.
 
 In estrema sintesi – pur nella consapevolezza del rischio di banalizzazione di ogni riduzione in brevi
 frasi – si spera di aver chiaramente evidenziato che l'espressione Bisogni Educativi Speciali:
 \begin{description}
 	\item[--] non è una diagnosi
 	\item[--] non è una certificazione
 	\item[--] non è uno stigma
 \end{description}
 è il riconoscimento del fatto che alcuni alunni possono richiedere, nel corso della loro carriera
 scolastica, per tempi più o meno lunghi, una particolare accentuazione della personalizzazione
 didattica, che resta fondamentale per ciascuno.
 
 Questo Ufficio sta predisponendo e collaborando a diverse iniziative di formazione che verranno
 attuate nel prossimo anno scolastico. Si confida nella partecipazione attiva dei dirigenti scolastici e dei
 docenti a tali iniziative, che saranno tempestivamente annunciate nel sito Internet \url{www.istruzioneer.it}
 settore Bisogni Educativi Speciali, settore che riporta, come detto, numerosi materiali utili per
 l'autoformazione, cui si rimanda come supporto allo studio e alla riflessione individuale e collegiale dei
 docenti. Le iniziative degli Uffici per ambito territoriale sono tempestivamente segnalate nei relativi
 siti internet, che le segreterie e i docenti sono invitati a consultare quotidianamente.
\end{description}
	

Allegato:

- scheda di approfondimento su Universal Design for Learning\footcite{USRperLEmiliaRomagna2013c}
