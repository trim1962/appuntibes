\chapter{Tar Liguria – Sentenza n. 1178-2012}
\label{cha:TarLiguriaSentenza1178_2012}
N. 01178/2012 REG.PROV.COLL.
N. 00771/2012 REG.RIC.
\begin{center}
IN NOME DEL POPOLO ITALIANO
	
	Il Tribunale Amministrativo Regionale per la Liguria
	(Sezione Seconda)
	
	ha pronunciato la presente
	
	SENTENZA
\end{center}
sul ricorso numero di registro generale 771 del 2012, proposto dai signori [omissis],
nella qualità di genitori del minore - OMISSIS -, rappresentati e difesi dall'avvocato
Raffaella Rubino, con domicilio eletto presso di lei a Genova in via Carducci 3/6;
contro
Ministero dell'istruzione, dell'università e della ricerca in persona del ministro in
carica
Liceo statale
[omissis]
in persona del dirigente in carica rappresentati e difesi dall'avvocatura distrettuale dello Stato di Genova, con
domicilio presso l'ufficio; per l'annullamento
del provvedimento di non ammissione alla seconda classe del liceo per le scienze umane assunto dal consiglio di classe della 1 F
degli atti presupposti
Visti il ricorso e i relativi allegati;
visto l'atto di costituzione in giudizio dell'amministrazione scolastica
vista la memoria ed i documenti da questa depositati;
Viste le memorie difensive;
Visti tutti gli atti della causa;
Relatore nella camera di consiglio del giorno 20 settembre 2012 il dott. Paolo
Peruggia e uditi per le parti i difensori come specificato nel verbale;
Ritenuto e considerato in fatto e diritto quanto segue.
\begin{center}
FATTO e DIRITTO
\end{center}
rilevato che: i signori [omissis] si ritengono lesi dal giudizio di non ammissione alla seconda classe formulata dal consiglio della 1F del liceo [omissis] di Genova a proposito del minore figlio - OMISSIS -, per cui hanno notificato l'atto 22.8.2012, depositato il
6.9.2012, con cui denunciano censure in fatto e diritto, e propongono una domanda cautelare;
l'amministrazione scolastica si è costituita in giudizio con memoria, depositando documenti;
il collegio ritiene di poter decidere con sentenza brevemente motivata, vista la rituale instaurazione del contraddittorio, la proposizione della domanda cautelare e la sufficienza degli elementi di prova, resa nota alle parti la presente
determinazione; l'impugnazione ha riguardo al procedimento seguito dall'organo collegiale scolastico, che non avrebbe tenuto conto delle problematiche di apprendimento del giovane allievo, a proposito del quale è in atti la documentazione rilasciata dalla sezione ligure dell'AFA, centro REUL Onlus, centro di riabilitazione per udito, linguaggio e comunicazione;
dopo la prima (25.7.2011) certificazione di detto organismo risulta che il consiglio scolastico si pronunciò il 6.12.2011, formulando un piano didattico personalizzato per l'interessato, non ostante la documentazione clinica non provenisse da una struttura sanitaria pubblica; in tal senso deve ritenersi che l'amministrazione scolastica si fosse vincolata ad assistere l'apprendimento dello studente con attitudine personalizzata, avendo riguardo alle problematiche di dislessia e discalculia che erano state diagnostiche al giovane; le misure adottate dall'istituto sono documentate dalle copiose allegazioni in atti, e documentano che, in sostanza, il metodo per lo studio dell'interessato non differì oltremodo da quello approntato per gli studenti indenni dalla patologia denominata DSA (difficoltà specifiche di apprendimento);
le contrarie allegazioni di cui alla memoria 14.9.2012 dell'avvocatura dello Stato non appaiono convincenti, posto che non elidono la valenza delle circostanze già in atti, relative all'utilizzo della forma scritta delle verifiche a cui lo studente fu sottoposto; ciò si pone in contrasto con il vincolo assunto dalla Scuola nell'indicata occasione, posto che in quella sede l'istituzione non si era riservata la possibilità di sindacare l'attitudine dello studente alla frequenza del corso di studi, se non con le modalità convenute; in tal senso appare verificata la doglianza con cui si lamentano distinte violazione del dpr 122 del 2009 e della legge 170 del 2010, che dispongono appunto a proposito degli accorgimenti didattici che la scuola deve adottare per favorire l'apprendimento degli studenti affetti dall'indicata patologia; l'apprezzamento del consiglio di classe impugnato in questo giudizio è derivato dalla valutazione degli atti presupposti gravati -– i giudizi dei singoli insegnanti --,
per cui può ritenersi provata la censura con cui si lamenta che la motivazione delle valutazioni non risulta aver tenuto conto del piano personalizzato che la scuola s'era vincolata a seguire;
in tal senso il ricorso è fondato e va accolto, dovendosi rimettere all'istituto una nuova determinazione che tenga conto dei principi sopra esposti; alla condivisione delle censure esposte consegue l'accoglimento del ricorso;
le spese seguono la soccombenza e sono equamente liquidate nel dispositivo, tenendo conto della natura della lite e della qualità delle parti;
\begin{center}
P.Q.M.
\end{center}
Il Tribunale Amministrativo Regionale per la Liguria (Sezione Seconda) accoglie il ricorso ed annulla gli atti impugnati, nel senso di cui alla motivazione condanna l'amministrazione scolastica al pagamento delle spese di lite sostenute dai ricorrenti, che liquida in euro 2.000,00 (duemila/00), oltre ad iva, cpa e
contributo.

Ordina che la presente sentenza sia eseguita dall'autorità amministrativa.

Così deciso in Genova nella camera di consiglio del giorno 20 settembre 2012 con
l'intervento dei magistrati:

Santo Balba, Presidente

Roberto Pupilella, Consigliere

Paolo Peruggia, Consigliere, Estensore

DEPOSITATA IN SEGRETERIA

Il 20/09/2012

IL SEGRETARIO

(Art. 89, co. 3, cod. proc. amm.)

