\chapter{Quadro Sinottico BES}\footcite{Nocera2014a}
\label{char:QuadroSinotticoBES}
\begin{longtable}{p{0.2\textwidth}p{0.4\textwidth}p{0.2\textwidth}p{0.2\textwidth}}
\toprule
 &\textbf{Disabilità}&\textbf{DSA}&\textbf{BES}  \\
\midrule
\endfirsthead
\toprule
 &\textbf{Disabilità}&\textbf{DSA}&\textbf{BES}  \\
\midrule
\endhead
\textbf{Individuazione degli alunni}&Certificazione ai sensi legge 104/92\footcite{Legge_104_92} Art 3 commi 1 e 3 &Diagnosi ai senso legge 170/10\footcite{legge170}&Delibera del consiglio di classe ai sensi della direttiva ministeriale 271/12/2012\footcite{dir27Dic2012} e CM 8/13\footcite{cm8_2013} e Nota 22/11/2013\footcite{Nota_2563_2013}\\
\midrule
\textbf{Strumenti didattici}&\textbf{PEI:} con riduzione di talune discipline (art 16 Legge 194/92)\footcite{Legge_104_92} e prove equipollenti e tempi più lunghi (art. 16 comma 3 legge 104/92)\footcite{Legge_104_92}

 Insegnate per il sostegno e/o assistenti per l'autonomia e la comunicazione.&\textbf{PDP:} con strumenti compensativi e/o misure dispensative e tempi più lunghi. &\textbf{PDP:} (solo se prescrive strumenti compensativi e/o misure dispensative) \\
 \midrule
 \textbf{Effetti sulla valutazione del profitto}&
\textbf{PRIMO CICLO:}
 
 \textbf{1) Valutazione positiva} 
 
 (art. 16 commi 1 e 2 L. n 104/92): se si riscontrano miglioramenti rispetto ai livelli iniziali degli apprendimenti relativi ad un PEI formulato solo con riguardo alle effettive capacità dell'alunno.
 
 \textbf{2) Attestato con i crediti formativi:}
 
 eccezionalmente in caso di mancati o insufficienti progressi rispetto ai livelli iniziali degli apprendimenti. Rilasciato dalla Commissione d'esame e non dalla scuola. È comunque titolo idoneo all'iscrizione al secondo ciclo (O.M. n 90/01, art. 11 comma 12\footcite{OM_90_2001})
 
\textbf{SECONDO CICLO:}

\textbf{ 1) Programmazione semplificata:} diritto al diploma, se superato positivamente esame di Stato con prove equipollenti e tempi più lunghi

 \textbf{2) Programmazione differenziata:} diritto ad attestato certificante i crediti formativi (rilasciato sempre dalla commissione d'esame e non dalla scuola) 
 &
 \textbf{1) Dispensa scritto lingue straniere compensata da prova orale:} consente Diploma (Linee guida\footcite{LineGuida2011} 4.4 allegate a D.M. 12/07/2011, art. 6 comma 5\footcite{DM_122_2011}).
 
 \textbf{2) Esonero lingue straniere:} solo attestato con i crediti formativi (D.M. 12/07/2011 art. 6 comma 6).
 &Misure dispensative (ad eccezione della dispensa dallo scritto di lingue straniere e dell'esonero normativamente previste solo per DSA).
 
 Strumenti compensativi.
 
 Tempi più lunghi,
 
 Con possibile Diploma.
 
 Per gli stranieri c'è normativa specifica\footcite{CM_2_2010}\footcite{MIUR2007}\footcite{CM_24_2006}.\\
 \bottomrule
\end{longtable}