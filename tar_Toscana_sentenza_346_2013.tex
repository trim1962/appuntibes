\chapter{Tar Toscana – Sentenza n.-346-2013}
\label{cha:TarToscanaSentenzan346_2013}
%\includepdf[pages=-,pagecommand={\thispagestyle{plain}}]{SentenzaTARToscanan3462013} 
\begin{center}
	Repubblica Italiana
	
	IN NOME DEL POPOLO ITALIANO
	
	Il Tribunale amministrativo Regionale per la Toscana
	
	(Sezione Prima)
\end{center}

ha pronunciato la presente 
\begin{center}
	SENTENZA
\end{center}
sul ricorso numero di registro del 2012, proposto dai sigg.ri Massimo --OMISSIS-- ed Eugenia Carasale, quali esercenti la podestà genitoriale sul figlio minore Nicolò --OMISSIS-- rappresentati e difesi dall'avv. Marian Bonfà, presso il cui studio sono elettivamente 
\begin{center}
	contro
\end{center}
Istituto statale della SS. Annunziata di Firenze, Consiglio di Classe della I Sez. A dell'Istituto SS. Annunziata di Firenze, Ministero dell'Istruzione, dell'Università e della Ricerca, rappresentati e difesi per legge dall'Avvocatura dello Stato, presso la cui sede sono domiciliati in Firenze, via degli Arazzieri 4;
\begin{center}
	per l'annullamento 
\end{center}
\begin{description}
	\item[--] del giudizio di non ammissione alla classe successiva dell'alunno, adottato dal Consiglio della Classe I Sez. A dell'Istituto SS. Annunziata di Firenze, così come riportato nel documento di valutazione anno scolastico 2011--2012 datato 8.06.2012;
	\item [--] del verbale del Consiglio della Classe I Sez. A dell'Istituto SS. Annunziata di Firenze con il quale è stata deliberata la non ammissione alla classe successiva dell'alunno; 
	\item [--] delle valutazioni periodiche effettuate rispetto all'alunno nelle singole materie di studio così come riportate nel documento di valutazione anno scolastico 2011--2012 datato 8.06.2012;
	\item di ogni altro atto presupposto, connesso, e/o conseguente, ancorché incognito al ricorrente, comprese le comunicazioni dell'Istituto prot. 3250/fp del 4.05.2012 e prot. 3994/FP del 1.06.2012
\end{description}

Visti il ricorso e i relativi allegati;

Visti gli atti di costituzione in giudizio dell'Istituto SS. Annunziata di Firenze, del Consiglio di Classe della I Sez. A dell'Istituto SS. Annunziata di Firenze e del Ministero dell'Istruzione, dell'Università e della Ricerca;

Viste le memorie difensive;

Visti tutti gli atti della causa;

Relatore nell'udienza pubblica del giorno 19 dicembre 2012 il dott. Pierpaolo Grauso e uditi per le parti i difensori come specificato nel verbale;

Ritenuto e considerato in fatto e diritto quanto segue.
\begin{center}
	FATTO
\end{center} 
Con ricorso notificato il 4 e depositato il 10 luglio 2012, Massimo --OMISSIS-- ed Eugenia Carasale, nella qualità di esercenti la podestà genitoriale sul figlio minore Nicolò --OMISSIS-- iscritto nell'anno scolastico 2011--2012 alla classe I, sezione A, della scuola secondaria di primo grado presso l'Istituto statale SS. Annunziata di Firenze, proponevano impugnazione avverso il giudizio di non ammissione del minore alla classe successiva, risultante dal documento di valutazione scolastica dell'8 giugno 2012. I ricorrenti affidavano le proprie doglianze a quattro motivi in diritto e concludevano per l'annullamento dell'atto impugnato, nonchè di tutti gli atti ad esso presupposti, in epigrafe megli individuati.

Costituitisi l'Istituto scolastico e il Ministero dell'Istruzione, che resistevano al gravame, con ordinanza del 25 -- 26 luglio il collegio, in accoglimento della domanda cautelare contenuto nello stesso ricoro introduttivo del giudizio, disponeva l'ammissione dell'alunno Nicolò --OMISSIS-- alla classe II del corso di studi di istruzione secondaria di primo grado.

Nel merito, la causa veniva discussa e trattenuta per la decisione nella pubblica udienza del 19 dicembre 2012, preceduta dal deposito di documenti e memorie difensive.
\begin{center}
	DIRITTO
\end{center}
I ricorrenti --OMISSIS-- ed Eugenia Carasale agiscono in veste di esercenti la podestà genitoriale sul figlio minore Nicolò, impugnandone il giudizio di non ammissione alla classe II della scuola secondaria di primo grado adottato al termine dell'anno scolastico 2011 -- 2012 dall'Istituto SS. Annunziata di Firenze, presso il quale Nicolò aveva frequentato la classe I nella sezione A. In fatto i ricorrenti espongono che, nel dicembre del 2011, al minore era stata diagnosticata una patologia da DSA (Disturbo Specifico dell'Apprendimento) caratterizzata da dislessia disgrafia con disartografia, circostanza della quale essi avevano immediatamente informato l'istituto scolastico; nessuna misura o iniziativa era stata tuttavia assunta dalla scuola, che nel febbraio del 2012 si era limitata a riscontare il profilo insufficiente dell'alunno in diverse materie. Solo nel successivo mese di marzo, a seguito del formale deposito del certificato recante la diagnosi di DSA, l'istituto aveva elaborato uno specifico piano didattico educativo ove erano indicati gli strumenti dispensativi e compensativi, nonché i criteri di valutazione e verifica da applicare nei confronti di Nicolò. Detto piano era rimasto peraltro inattuato, unicamente agli obblighi di collaborazione e comunicazione assunti dalla scuola verso la famiglia dell'alunno, sino a quando, il 4 maggio 2012, l'istituto aveva rappresentato il persistere di una situazione di rendimento insufficiente e di comportamento connotato da attenzione non costante, da loro invitato a riformulare la valutazione la valutazione alla luce delle indicazioni contenute nel piano educativo personalizzato, l'istituto aveva quindi replicato di aver operato nel rispetto del piano e delle prescrizioni stabilite dalla legge, salvo, infine, comunicare la mancata ammissione di Nicolò alla classe successiva.

In diritto, con il primo motivo di gravame i ricorrenti, ricostruita la normativa vigente in materia di trattamento del disturbi dell'apprendimento in ambito scolastico, a partire dalla legge n. 170/2010 e dal DM 12 luglio 2011, n. 5669, addebitano al personale docente e dirigenziale dell'Istituto SS. Annunziata la mancata osservazione e riconoscimento dei sintomi, pur chiarissimi, che avrebbero indicato nel figlio Nicolò la presenza del disturbo poi diagnosticato, ed in ogni caso la mancata attivazione di qualsivoglia intervento anche dopo essere stati informati della diagnosi. L'inerzia dell'istituto sarebbe proseguita altresì dopo la formale presentazione delle certificazioni sanitarie attestanti il disturbo, come dimostrato dal fatto che la redazione del piano educativo personalizzato per Nicolò sarebbe seguita con ritardo.

Con il secondo motivo, è quindi denunciata la non conformità del piano personalizzato in questione alle linee guida allegate al citato D.M. 12 luglio 2011, trattandosi di documento del contenuto generico e privo dell'indicazione delle attività didattiche individualizzate e personalizzate richieste dalla normativa. Per altero verso è contestata la corretta applicazione delle misure dispensativa e compensative approvate per Nicolò, quali la prescrizione dei non assegnare un carico eccessivo di compiti a casa, la dispensa delle prove scritte e comunque l'utilizzo di criteri individualizzati di valutazione degli scritti, la programmazione delle interrogazioni.

Con il terzo motivo, i ricorrenti lamentano ulteriormente la violazione, da parte dell'istituto SS. Annunziata, delle previsioni di cui alla legge n. 170/2010 concernenti il continuo monitoraggio delle misure predisposte in favore dell'alunno --OMISSIS-- e l'obbligo di comunicare ai genitori il persistere delle difficoltà di apprendimento nonostante le attività di recupero poste in essere. Mentre, con il quarto motivo, evidenziano come l'impugnato giudizio di non ammissione non sa adeguatamente motivato sotto il profilo dell'incidenza del DSA sul rendimento scolastico di Nicolò.

Le censure, che saranno esaminate congiuntamente, sono fondate.

L'art. 5 della legge n. 170/2010(\cit{Nuove norme in materia di siturbi specifici di apprendimento in ambito scolatico}) stabilisce che li studenti con diagnosi di DSA hanno diritto a fruire di appositi provvedimenti dispensativi e compensativi di flessibilità didattica nel corso dei cicli di istruzione e formazione e negli istituti universitari, dovendo loro essere garantiti: l'uso di una didattica individualizzata e personalizzata, con forme efficaci e flessibili di lavoro scolastico che tengano conto anche di caratteristiche peculiari dei soggetti, quali il bilinguismo, adottando una metodologia e una strategia educativa adeguate; l'introduzione di strumenti compensativi, compresi i mezzi di apprendimento alternativi e le tecnologie informatiche, nonché misure dispensative da alcune prestazioni non essenziali ai fini della qualità dei concetti da apprendere; per l'insegnamento delle lingue straniere, l'uso di strumenti compensativi che favoriscano la comunicazione verbale e che assicurino ritmi graduali di apprendimento, prevedendo anche, ove risulti utile, la possibilità dell'esonero. Dette misure, da sottoporsi a periodico monitoraggio per valutarne l'efficacia e il raggiungimento degli obiettivi, debbono inoltre essere accompagnati da idonee forme di verifica e di valutazione.
 
 Le richiamate disposizioni di legge trovano attuazione ad opera del D.M. 12 luglio 2011, n. 5669, che per quanto qui interessa, e in sintesi, prevede che le istituzioni scolastiche, tenendo conto delle indicazioni contenute nelle allegate linee guida, provvedendo ad attuare i necessari interventi pedagogico-didattici per il successo formativo degli alunni e degli studenti con DSA, attivando percorsi di didattica individualizzata e personalizzata e ricorrendo a strumenti compensativi e misure dispensative (art. 4 co. 1); che le istituzioni scolastiche assicurano l'impiego degli opportuni strumenti compensativi, curando particolarmente l'acquisizione, da parte dell'alunno e dello studente con DSA delle competenze per un efficiente utilizzo degli stessi (art. 4 co.4); che l'adozione delle misure dispensative è finalizzata ad evitare situazioni di affaticamento e di disagio in compiti direttamente coinvolti dal disturbo, senza peraltro ridurre il livello degli obiettivi di apprendimento previsti nei percorsi didattici individualizzatati e personalizzati (art. 4 co. 5); che la scuola garantisce ed esplica, nei confronti di alunni e studenti con DSA, interventi didattici individualizzati e personalizzati, anche attraverso la redazione di un piano didattico personalizzato, con l'indicazione degli strumenti compensativi e delle misure dispensative adottate (art. 5); che la valutazione scolastica, periodica e finale, degli alunni e degli studenti co DSA deve essere coerente con gli interventi pedagogico- didattici approvati e che le istituzioni scolastiche adottano modalità valutative che consentono all'alunno o allo studente con DSA di dimostrare effettivamente il livello di apprendimento raggiunto, mediante l'applicazione di misure che determino le condizioni ottimali per l'espletamento della prestazione da valutare -- relativamente ai tempi di effettuazione e alle modalità di strutturazione delle prove -- riservando particolare attenzione alla padronanza dei contenuti disciplinari, a prescindere dagli aspetti legati all'abilità deficitaria (art. 6 co. 1 e 2).
 
 Le \cit{linee guida} allegate al decreto, dopo aver tratteggiato le diverse manifestazioni patologiche ascrivibili a DSA, nonché, quanto agli interventi educativi da porre in essere in presenza del disturbo, la distinzione fra didattica individualizzata e personalizzata, descrivono gli strumenti compensativi come strumenti didattici e tecnologici che sostituiscono o facilitano la prestazione richiesta nell'abilità deficitaria, sollevando così l'alunno o lo studente con DSA da una prestazione resa difficoltosa dal disturbo senza per questo facilitargli il compito dal punto di vista cognitivo. L'utilizzo di tali strumenti -- fra i quali la sinesi vocale, che trasforma un compito di lettura in compito di ascolto; il registratore, che consente all'alunno o allo studente di no scrivere gli appunti della lezione, i programmi di video scrittura con correttore orografico, che permettono la produzione di testi sufficientemente corretti senza l'affaticamento della rilettura e della contestuale correzione degli errori; la calcolatrice, che facilita le operazioni di calcolo; altri strumenti tecnologicamente meno evoluti quali tabelle, formulari, mappe concettuali -- non è immediato e i docenti debbono aver cura di sostenere l'uso da parte di alunni e studenti DSA.
 
 Rappresentano invece misure dispensative quegli interventi che consentono all'alunno o allo studente di non svolgere alcune prestazioni che, a causa del disturbo, risultano particolarmente difficoltose e che non migliorano l'apprendimento, ovvero di fruire di tempi di elaborazione più lunghi per lo svolgimento di una prova, tenuto conto che l'alunno affetto dal disturbo impiega più tempo dei propri compagni nella fase di decodifica del c.d. \textit{item} o quesiti della prova.
 
 Tanto premesso, deve innanzitutto rilevarsi come gli elementi a disposizione non permettano di imputare con certezza al personale docente dell'istituto SS. Annunziata il ritardo nella diagnosi di DSA a carico di Nicolò OMISSIS. La relazione del 24 luglio 2012 proveniente dall'Ambulatorio di neuropsicologia dell'A.O.U. \cit{Meyer} di Firenze, invocata dai ricorrenti, va letta in combinato disposto con le linee guida ministeriali, che indicano già nella scuola dell'infanzia e nella scuola primaria (quest'ultima in particolare) i segmenti formativi all'interno dei quali occorre operare per il riconoscimento dei un potenziale disturbo specifico dell'apprendimento; pertanto, nella specie, potrebbe al più rimproverarsi all'istituto resistente di aver atteso sino al secondo quadrimestre per l predisposizione del piano educativo personalizzato, comportamento che però si giustifica in ragione del fatto che la diagnosi definitiva del disturbo è stata formalizzata solo nel mese di febbraio 2012, e in termini tali (lieve dislessia e lieve disgrafia con disortografia ) da far presumere che all'osservazione degli insegnanti, la situazione di Nicolò potesse inconsapevolmente apparire non patologica, mentre non è dimostrato che la scuola ne fosse stata messa al corrente dai genitore, pur informalmente, in precedenza.
 
 Venendo ai rimanenti profili di gravame, si osserva come il piano didattico personalizzato redatto per Nicolò --OMISSIS-- nei primi giorni del mese di marzo 2012 prevedesse il ricorso a una cospicua serie di strumenti compensativi e dispensativi, nonché a criteri di verifica e valutazione implicanti, fra l'altro l'opportuno adeguamento delle \cit{griglie} valutative, e, infine l'\cit{accordo} con la famiglia circa le modalità di assegnazione dei compiti a casa, le dispense, le interrogazioni (modalità, contenuti richieste più importanti). Che tali strumenti didattici abbiano trovato trovato applicazione puntuale e sistematica non può tuttavia, dirsi adeguatamente dimostrato dall'istituto resistente, ove si consideri che a seguito dell'approvazione del piano solo uno dei docenti -- quello di spagnolo -- ha attestato nel proprio registro di aver sottoposto l'alunno ad interrogazioni programmata su argomenti preventivamente comunicati alla famiglia, mentre in altri casi le verifiche sono state precedute da comunicazioni incomplete, ovvero (le verifiche di storia del 31 maggio e quella di geografia del 1 giugno, comunicate rispettivamente il 28 e il 29 maggio) eccessivamente ridosso delle date stabilite per la prova, con il risultato di vanificare le stessa ragion d'essere del \cit{patto con la famiglia } sancito dal piano personalizzato: questo infatti, nel prevedere oltretutto il contenimento della mole dei compiti a casa, implicava il riconoscimento all'alunno della possibilità di programmare -- con l'aiuto della famiglia -- la distribuzione dello studio casalingo durante la settimana. 
 
 Altro aspetto critico, nel confronto con le previsioni del piano didattico dell'alunno, è poi quello che attiene alle modalità di svolgimento delle verifiche di italiano. Se, per un verso, non risulta che sia stata offerta all'alunno la possibilità di compensare con prove orali le ripetute insufficienze conseguite nelle prove scritte, le stesse modalità di correzione, nonostante i giudizi espressi dal docente tengano conto dei soli contenuti delle prove, non appaiono rispondenti alla concomitante prescrizione di non correggere tutti gli errori formali; prescrizione da collocarsi nell'ottica di non rischiare la demotivazione dell'alunno in presenza di un elevato numeri di errori e, nella specie, da tenere presente a maggior ragione per il fatto che le stesse metodologie alla base del piano prevedevano l'incentivazione della videoscrittura per la produzione testuale, ausilio che al figlio dei ricorrenti non risulta essere stato reso disponibile (l'utilizzo di videoscrittura con correttore automatico risponde, del resto, ad uno dei rimedi compensativi specificamente individuati dal paragrafo 4.3.2 delle linee-guida ministeriali per gli alunni affetti da disturbo di scrittura, al dichiarato scopo di ottenere all'origine testi più coretti). Ai vizi che affliggono alcuni profili dell'attività didattica e valutativa riconducibile ai singoli docenti, perché non adeguatamente calibrata sulle prescrizioni del piano educativo personalizzato, si aggiunge poi l'assenza di qualsivoglia controllo intermedio (monitoraggio) in ordine all'efficacia del piano stesso.
 
 Laddove il giudizio di ammissione di Nicolò --OMISSIS-- alla classe successiva presenta le più vistose ed insanabili divergenze dalla normativa di rango primario e secondari che regola la materia è peraltro, nell'omessa considerazione della peculiare -- patologica -- condizione dell'alunno in sede di scrutinio finale. L'inciso \cit{pur considerando la certificazione e il piano didattico personalizzato}, che si legge nel verbale di scrutinio del 5 giugno 2012, non dà invero alcun conto della misura e della qualità della qualità dell'apprezzamento concretamente riservato dal consiglio di classe all'incidenza del disturbo sul rendimento dell'alunno, né spiega per quale motivo il conclamato insuccesso degli ausili approvati con il piano didattico personalizzato debba, in definitiva essere esclusivamente imputato a scarso impegno dell'alunno medesimo, indipendente dal disturbo, piuttosto che ad altri fattori, in primo luogo ai dati obiettivi della (in)tempestività della diagnosi e dell'eventuale inadeguatezza degli interventi effettuati, oltretutto -- lo si ripete -- mai sottoposti a monitoraggio durante l'anno scolastico; e neppure consente di comprendere perché, in una prospettiva di lungo periodo, la ripetizione dell'anno scolastico avrebbe dovuto rappresentare la miglior soluzione possibile nell'interesse dell'alunno a fronte delle possibili conseguenze negative della bocciatura sull'equilibrio psichico e sulla vita di relazione del bambino, nonché sulla sua stessa capacità di apprendimento (si veda di nuovo la relazione sanitaria del 24 luglio 2012) e questo anche in relazione agli effetti positivi che la prosecuzione del trattamento didattico personalizzato avrebbe potuto o meno prevedibilmente produrre in futuro alla luce del grado di intensità del disturbo.
 
 Le riscontrate carenze motivazionali sono tanto più gravi, in quanto la delibera di non ammissione si fonda sul rilievo di un insieme di difficoltà dell'alunno -- difficoltà nella produzione scritta e in quella orale, difficoltà di astrazione, di concentrazione, mancato rispetto delle consegne in classe, assenza di autonomia e consapevolezza della propria condizione -- che, a ben vedere, coincidono proprio con i sintomi del disturbo diagnosticato a carico dello stesso, mentre non vi è alcuna considerazione circa la padronanza o meno dei contenuti disciplinari, in aperta violazione dell'art. 6 del D.M. 12 luglio 2012 sopra estensivamente citato, e, come già osservato in fase cautelare, tali carenze non sono suscettibili di venire colmate attraverso la nota dell'istituto in data 12 luglio 2012, trattandosi di attività eminentemente discrezionale che non ammette integrazioni postume della motivazione.
 
 In forza di tutto quanto precede, il ricorso deve essere accolto ai fini dell'annullamento degli atti e provvedimenti impugnati, nonché della definitiva conferma dell'ammissione di Nicolò -- OMISSIS -- alla classe II della scuola secondaria di primo grado, provvisoriamente disposta in via cautelare. 
 
 Le spese di lite seguono la soccombenza, e sono liquidate come in dispositivo.
 \begin{center}
 	P.Q.M.
 \end{center}
 Il Tribunale Amministrativo Regionale per la Toscana (sezione Prima), definitivamente pronunciando, accoglie il ricorso e per l'effetto, annullati gli atti e provvedimenti impugnati, conferma l'ammissione dell'alunno Nicolò -- OMISSIS-- alla classe II della scuola secondaria di primo grado.
 
 Condanna le amministrazioni resistenti alla rifusione delle spese processuali, che liquida in complessivi euro 2.000,00. oltre al rimborso del contributo unificato ed agli accessori di legge.
 
 Ordina che la presente sentenza sia eseguita dall'autorità amministrativa. 
 Così deciso in Firenze nella camera di consiglio del giorno 19 dicembre 2012 con l'intervento dei magistrati:
 
 Paolo Buonvino, Presidente
 
 Carlo Testori, Consigliare 
 
 Pierpaolo Grauso, Primo Referendario, Estensore
 \begin{center}
 	DEPOSITATA IN SEGRETERIA
 	
 	IL 28/02/2013
 	
 	(Art. 89, co 3, cod. proc. amm.)
 \end{center} 