\chapter{Cosa fa il Consiglio di classe}
\label{sec:Cosafaconsclass}
\section{Presa in carico}
\label{sub:presaincarico}
Iniziamo con una premessa generale a tutti i discorsi cioè dal DPR 275/1999\footcite{DPR_275_1999} sull'autonomia scolastica per capire cosa permette l'autonomia
\begin{quote}
\begin{description}
	\item[Art. 1] Natura e scopi dell'autonomia delle istituzioni scolastiche
	\begin{description}
		\item \mancatesto
		\item[2] L'autonomia delle istituzioni scolastiche è garanzia di libertà di insegnamento e di pluralismo culturale e si sostanzia nella progettazione e nella realizzazione di interventi di educazione, formazione e istruzione mirati allo sviluppo della persona umana, adeguati ai diversi contesti, alla domanda delle famiglie e alle caratteristiche specifiche dei soggetti coinvolti, al fine di garantire loro il successo formativo, coerentemente con le finalità e gli obiettivi generali del sistema di istruzione e con l'esigenza di migliorare l'efficacia del processo di insegnamento e di apprendimento
	\end{description}
	\item[Art. 4] Autonomia didattica
	\begin{description}
		\item[1] Le istituzioni scolastiche, nel rispetto della libertà di insegnamento, della libertà di scelta educativa delle famiglie e delle finalità generali del sistema, a norma dell'articolo 8 concretizzano gli obiettivi nazionali in percorsi formativi funzionali alla realizzazione del diritto ad apprendere e alla crescita educativa di tutti gli alunni, riconoscono e valorizzano le diversità, promuovono le potenzialità di ciascuno adottando tutte le iniziative utili al raggiungimento del successo formativo.
		\item [2] Nell'esercizio dell'autonomia didattica le istituzioni scolastiche regolano i tempi dell'insegnamento e dello svolgimento delle singole discipline e attività nel modo più adeguato al tipo di studi e ai ritmi di apprendimento degli alunni. A tal fine le istituzioni scolastiche possono adottare tutte le forme di flessibilità che ritengono opportune e tra l'altro:
		\begin{description}
			\item 	\mancatesto
			\item[c] l'attivazione di percorsi didattici individualizzati, nel rispetto del principio generale dell'integrazione degli alunni nella classe e nel gruppo, anche in relazione agli alunni in situazione di handicap secondo quanto previsto dalla legge 5 febbraio 1992, n. 104;
		\end{description} 
	\end{description}
\end{description}
\end{quote}
La presa in carico di un alunno in BES spetta a tutto il Consiglio, infatti lo Stato assegna un insegnante di sostegno solo agli alunni certificati in base alla legge 104/92, e ciò segue dall'Art. 35 comma 7~\footcite{Legge27dic00n289} legge 289/02 e dal DPCM 186/06~\footcite{DPCM22_02_06_N_185} e viene ribadito in una sentenza storica della Corte Costituzionale~\footcite{SCC_80_2010}. 

La Direttiva si pone lo scopo di estendere a tutti gli alunni in BES quanto previsto dalla legge 53 del 2003~\footcite{Legge_53_2003} e dalla legge 170 del 2010~\footcite{legge170} e le prassi per i DSA sono da modello per questa estensione. Da ricordare che non esiste una legge quadro su i BES come per i DSA, ciò che vincola il Consiglio di Classe sono una direttiva\footcite{dir27Dic2012}, una circolare\footcite{cm8_2013} ed una nota\footcite{Nota_1551_2013}. 

La presa in carico di un alunno da parte del Consiglio di classe crea un vincolo che non si estingue con la sola formale esecuzione della procedura. Il \cit{legame} che le varie leggi e circolari impongono è sostanziale e in caso di ricorso, la scuola soccombe molto facilmente. Ecco alcuni esempi:
TAR--Liguria\footnote{Sentenza in appendice}
\begin{quote}
	\mancatesto
	l'impugnazione ha riguardo al procedimento seguito dall'organo collegiale
	scolastico, che non avrebbe tenuto conto delle problematiche di apprendimento
	del giovane allievo, a proposito del quale è in atti la documentazione rilasciata\mancatesto formulando un piano didattico personalizzato
	per l'interessato, non ostante la documentazione clinica non provenisse da una
	struttura sanitaria pubblica;
	in tal senso deve ritenersi che l'amministrazione scolastica si fosse vincolata ad
	assistere l'apprendimento dello studente con attitudine personalizzata, avendo
	riguardo alle problematiche di dislessia e discalculia che erano state diagnostiche al
	giovane;
	le misure adottate dall'istituto sono documentate dalle copiose allegazioni in atti, e
	documentano che, in sostanza, il metodo per lo studio dell'interessato non differì
	oltremodo da quello approntato per gli studenti indenni dalla patologia
	denominata DSA (difficoltà specifiche di apprendimento);
	le contrarie allegazioni di cui alla memoria 14.9.2012 dell'avvocatura dello Stato
	non appaiono convincenti, posto che non elidono la valenza delle circostanze già
	in atti, relative all'utilizzo della forma scritta delle verifiche a cui lo studente fu
	sottoposto;
	ciò si pone in contrasto con il vincolo assunto dalla Scuola nell'indicata occasione,
	posto che in quella sede l'istituzione non si era riservata la possibilità di sindacare
	l'attitudine dello studente alla frequenza del corso di studi, se non con le modalità
	convenute;
	in tal senso appare verificata la doglianza con cui si lamentano distinte violazione
	del dpr 122 del 2009\footcite{DPR_122_2009} e della legge 170 del 2010\footcite{legge170}, che dispongono appunto a
	proposito degli accorgimenti didattici che la scuola deve adottare per favorire
	l'apprendimento degli studenti affetti dall'indicata patologia;
	l'apprezzamento del consiglio di classe impugnato in questo giudizio è derivato
	dalla valutazione degli atti presupposti gravati –- i giudizi dei singoli insegnanti --,
	per cui può ritenersi provata la censura con cui si lamenta che la motivazione delle valutazioni non risulta aver tenuto conto del piano personalizzato che la scuola
	si era vincolata a seguire;
	in tal senso il ricorso è fondato e va accolto, dovendosi rimettere all'istituto una
	nuova determinazione che tenga conto dei principi sopra esposti;
	alla condivisione delle censure esposte consegue l'accoglimento del ricorso;~\footcite{tarliguria1178}\mancatesto
\end{quote}
Il TAR-Toscana scrive\footnote{Sentenza in appendice}
 \begin{quote}
 \mancatesto
 la valutazione e la verifica degli apprendimenti, comprese quelle effettuate in sede
 di esame conclusivo dei cicli, devono tenere conto delle specifiche situazioni soggettive di tali
 alunni; a tali fini, nello svolgimento dell'attività didattica e delle prove di esame, sono adottati,
 nell'ambito delle risorse finanziarie disponibili a legislazione vigente, gli strumenti metodologico -
 didattici compensativi e dispensativi ritenuti più idonei. Se, alla luce della disciplina dianzi
 richiamata, la considerazione della condizione patologica dell'alunno rappresenta un elemento
 necessario non soltanto dell'iter didattico, ma anche del momento valutativo, nella specie il vizio
 dell'impugnato provvedimento di non ammissione consiste nel fatto che il giudizio formulato dal
 consiglio di classe, al di là di un fugace riferimento verbale, prescinde totalmente dal disturbo che
 affligge il ricorrente, le valutazioni negative non risultando essere state in alcun modo ponderate
 attraverso l'analisi delle possibili concause patologiche del cattivo rendimento dello studente nelle
 materie di indirizzo. In altri termini, nel mentre ha evidenziato l'atteggiamento \cit{non sempre
 collaborativo in tutte le discipline} del [omissis], ovvero il suo impegno \cit{non uniforme} e gli
 insoddisfacenti risultati delle attività di recupero e delle verifiche intermedie, il consiglio di classe
 non ha svolto alcuna analisi circa l'incidenza causale del DSA sul rendimento del ricorrente, non
 fosse altro per escluderla; di modo che il giudizio conclusivo manca di quella individualizzazione e
 personalizzazione che, richieste per ciascuno studente, lo sono a maggior ragione per quelli affetti
 da disturbi dell'apprendimento.
 Si aggiunga il difetto di qualsiasi verifica circa le ragioni della scarsa efficacia dimostrata dagli
 strumenti metodologici e didattici previsti dal PDP, la cui stessa attuazione non appare peraltro
 essere stata puntuale, con particolare riguardo alla prevista somministrazione di prove equipollenti\footcite{tartoscana346}\mancatesto
 \end{quote}
La sentenza del TAR-Umbria\footnote{Sentenza in appendice} è invece un contro esempio rispetto alle precedenti:
 \begin{quote}
 \begin{description}
\item 	\mancatesto
 In sintesi, la legge richiama la necessità che la scuola elabori e realizzi, in sede di insegnamento, verifica e valutazione, un percorso formativo personalizzato, che tenga conto delle esigenze e delle potenzialità specifiche di ciascun studente con DSA, ed indica a tal fine \cit{strumenti compensativi} (che si sostanziano nell'introduzione di mezzi di apprendimento alternativi e nell'uso di tecnologie informatiche) e \cit{misure dispensative} (che si sostanziano nella riduzione del programma o nell'esenzione dalle lingue straniere), che spetta ai docenti individuare ed attuare in concreto.
 \item[6] Ora, i ricorrenti hanno elencato gli strumenti compensativi - uso del registratore in classe, uso di tabelle della memoria, possibile utilizzo di computer con programmi di video scrittura con correttore ortografico –- e le possibili \cit{dispense} -- tempi più lunghi per le prove scritte e per lo studio, organizzazione di interrogazioni programmate, assegnazione di compiti a casa in misura ridotta, verifiche prevalentemente orali - che, a loro dire, avrebbero dovuto essere, ma non sono stati adottati.
L'Avvocatura dello Stato ha dato conto che l'Istituto ha predisposto a sostegno del ragazzo \cit{interventi individuali mirati e uno specifico Piano didattico individualizzato}, ed ha elencato (oltre che riassunto in un quadro sinottico) le misure compensative, le modalità di verifica ed i criteri di valutazione, elaborati per ciascuna materia dai docenti tenendo in considerazione le certificazioni della A.U.S.L., il dialogo con la famiglia, la conoscenza dello studente negli anni precedenti, i relativi giudizi osservativi e risultati di apprendimento
\item[7] Con la memoria finale, i ricorrenti, riconoscendo l'elaborazione del Piano didattico, ribattono che si tratta di un modello non personalizzato, bensì riprodotto di anno in anno per qualunque alunno sia interessato da DSA.
Ad avviso del Collegio, la utilizzazione di una sorta di \cit{modello} di intervento dedicato agli alunni affetti da DSA non comporta di per sé la non attuazione della legge 170/2010, anche tenuto conto che la norma si preoccupa di chiarire che gli interventi previsti sono sì garantiti, ma \cit{a valere sulle risorse specifiche e disponibili a legislazione vigente}, vale a dire nella misura in cui le scuole abbiano le risorse finanziarie, organizzative ed umane sufficienti a realizzarli. E possono considerarsi notorie le difficoltà in cui si dibattono gli istituti scolastici, in questi ultimi anni caratterizzati da una costante riduzione di dette risorse.
Può aggiungersi che la classe frequentata dal ragazzo è composta da altri 9 studenti, dei quali: 1 studente con handicap psico-fisico grave, 1 con handicap psichico grave, 1 con handicap psichico medio, 1 trasferito da altro istituto in corso d'anno con necessità di recupero in quasi tutte le materie, 1 trasferito da altro istituto in corso d'anno e seguito dai servizi sociali, 1 con grave situazione economico-familiare e deprivazione culturale, 2 in condizioni di normalità e 1 studente eccellente. Da tale particolare composizione (non è questa la sede per valutare quanto opportuna, non essendo peraltro tale profilo oggetto di censure), dall'esiguità del numero degli studenti, che consente con i docenti \cit{un lavoro 1 ad 1}, dalla circostanza che le materie di indirizzo previste dal piano di studi (disegno professionale, laboratorio fotografia, disegno dal vero, plastica) coprono circa il 70\% del monte ore totale e sono tali da non potersi svolgere se non in forma individuale in un contesto di natura laboratoriale ed operativo, la scuola fa (plausibilmente) discendere che l'individualizzazione e la personalizzazione dei percorsi formativi ed il sostegno non siano l'eccezione, bensì la regola quotidiana.
Ancora, i ricorrenti sottolineano che, per rendere credibile il piano didattico, non può essere invocata la continuità del corpo docente che ha seguito il ragazzo anche nelle prime due classi del corso, posto che gli insegnanti di italiano e di matematica sono invece stati sostituiti in quest'ultimo anno.
Al riguardo, occorre osservare che per tutte le altre materie i docenti sono rimasti gli stessi dall'inizio del corso.
Non sembra poi possa considerarsi sintomo di inadeguatezza la circostanza che il Piano sia datato 13 ottobre 2010, cioè ad un momento in cui, secondo i ricorrenti, non sarebbe stato possibile per i docenti avere già chiare le misure da intraprendere.
I ricorrenti censurano anche la concreta realizzazione di quanto previsto nel Piano, sostenendo che non sono mai state concordate e programmate le interrogazioni, né assegnati tempi più lunghi per le verifiche e per l'esame, né usati computer con videoscrittura, correttore ortografico, registratore, audio libri.
L'Amministrazione ha per contro precisato (cfr. verbali dei Consigli di Classe in data 1 febbraio, 19 maggio e 6 giugno 2011) che sono state attuate prove differenziate per tutte le materie previste (italiano, matematica, fisica); che i docenti hanno spiegato il programma utilizzando mappe concettuali, schematizzazioni, esemplificazioni, ripasso di consolidamento; che è stato consentito allo studente l'uso di mappe concettuali ed appunti al momento delle prove scritte e orali; che sono stati concessi tempi più lunghi per le verifiche; che è stata omessa la considerazione degli errori ortografici; che è stato limitato il carico dei compiti a casa.
Anche per quanto concerne l'esame di qualifica, risulta che siano stati concessi (a tutta la classe) tempi più lunghi per lo svolgimento delle prove scritte (6 ore per italiano, 4 per matematica, 6 per laboratorio – peraltro, risulta che il ragazzo sia uscito con largo anticipo rispetto all'orario consentito) e proposte prove più semplici rispetto a quelle effettuate durante l'anno; che siano stato fornito dal docente lo \cit{strumento d'aiuto} (calcolatrice) per la prova di matematica; che non siano stati considerati gli errori ortografici nella prova di italiano, dando nel contempo più importanza alla sostanza che alla forma; che siano stati limitati gli argomenti richiesti durante la prova orale, consentendo l'utilizzo di mappe concettuali e libro aperto.
In sostanza, l'unico intervento (che la stessa scuola riconosce) non attuato, tra quelli di cui i ricorrenti lamentano la mancanza, concerne la registrazione delle lezioni in classe. Sulla rilevanza di tale omissione i ricorrenti insistono, sempre con la memoria finale.
L'Avvocatura risponde sottolineando che si è trattato di una scelta della scuola dettata dall'intenzione di non assecondare la già spiccata negligenza dell'alunno (il quale si presentava a scuola privo spesso dei materiali didattici e dimostrava scarsa propensione allo studio pomeridiano) mediante l'utilizzo di uno strumento \cit{delegante}, che avrebbe potuto essere nocivo per il suo sviluppo cognitivo e lo stimolo delle sue capacità di concentrazione (lo stesso motivo ha condotto a mantenere nei limiti del necessario l'uso del PC e della videoscrittura).
Il Collegio (per quanto sia noto che molti specialisti di settore suggeriscono l'uso del registratore) ritiene di non poter sindacare tale specifica valutazione, che rientra nel novero delle possibili scelte tecnico discrezionali demandate alla scuola.
Più volte, del resto, risulta sottolineato (cfr. verbali del collegio dei docenti in data 9 novembre 2010, 1 febbraio, 30 marzo, 3 maggio, 19 maggio 2011) che le difficoltà di apprendimento e le carenze pregresse accumulate dallo studente non sono state colmate anche a causa del suo scarso impegno nello studio, non soltanto nelle materie \cit{teoriche} ma anche nella maggior parte di quelle di carattere operativo e laboratoriale, nelle quali l'incidenza dei fattori cognitivi è assai minore.
In sintesi, in base agli atti non sembra possibile imputare l'insuccesso scolastico ad omissioni e lacune significative del Piano didattico personalizzato e della sua attuazione.
\item [8] Quanto alla comunicazione con i genitori ricorrenti, risultano trasmessi il pagellino inter quadrimestrale (a novembre 2010) e la pagella del primo quadrimestre (a febbraio 2011), ed il pagellino inter quadrimestrale del secondo quadrimestre (ad aprile), con relative schede delle carenze registrate e lettera di convocazione presso il coordinatore di classe. La scuola sostiene anche che i ricorrenti avrebbero usufruito di numerosi colloqui con i docenti, e che più volte sarebbero state loro segnalate le assenze del figlio dall'istituto: in totale, 50 giorni, entità cospicua che tuttavia il consiglio di classe ha deciso di non considerare circostanza ostativa all'ammissione all'esame di qualifica. I ricorrenti ribattono che dal libretto delle giustificazioni risultano firmate giustificazioni per soli 10 giorni di assenza (l'ultima delle quali risalente al febbraio 2011), e lamentano quindi che non siano state segnalate le altre; la circostanza sembra effettivamente dimostrare una lacuna nella gestione del rapporto da parte della scuola, ma tale rilievo, da solo, non può evidentemente inficiare il provvedimento impugnato.
\item [9]Il ricorso non può pertanto essere accolto ~\footcite{tarumbria329}
 		 \mancatesto
 	\end{description}
 \end{quote}
 
 La circolare ministeriale 8 del 2013~\footcite{cm8_2013} precisa che il Consiglio predispone un PDP o altro percorso idoneo
\begin{quote}
sulla base di elementi oggettivi (come ad es. una segnalazione degli operatori dei sevizi sociali), ovvero di ben fondate considerazioni psicopedagogiche e didattiche
\mancatesto
Ove non sia presente certificazione clinica o diagnosi, il Consiglio di classe o il team dei docenti
motiveranno opportunamente, verbalizzandole, le decisioni assunte sulla base di considerazioni
pedagogiche e didattiche; ciò al fine di evitare contenzioso~\footcite{cm8_2013}
\end{quote}
Quindi il Consiglio di classe decide in base alla documentazione in suo possesso se deve definire un:
\begin{quote}
percorso individualizzato e personalizzato, redatto in un Piano
Didattico Personalizzato (PDP), che ha lo scopo di definire, monitorare e documentare –- secondo
un'elaborazione collegiale, corresponsabile e partecipata -- le strategie di intervento più idonee e i
criteri di valutazione degli apprendimenti~\footcite{cm8_2013}
\end{quote}
Mentre cosa si intende 
monitorare e documentare è chiaro dalla legge 170/10~\footcite{legge170}
\begin{quote}
	\begin{description}
		\item[Art. 15] Misure educative e didattiche di supporto
		\begin{description}
			\item \mancatesto
			\item[3] Le misure di cui al comma 2 devono essere sottoposte periodicamente a monitoraggio per valutarne l'efficacia e il raggiungimento degli obiettivi.
		\end{description}
	\end{description}
\end{quote}
e dalle linee guida del 2011\footcite{LineGuida2011} 
\begin{quote}
\begin{description}
	\item[3.1] Documentazione dei percorsi didattici
	
	 	Le attività di recupero individualizzato, le modalità didattiche personalizzate, nonché gli
	 	strumenti compensativi e le misure dispensative dovranno essere dalle istituzioni scolastiche
	 	esplicitate e formalizzate, al fine di assicurare uno strumento utile alla continuità didattica e alla
	 	condivisione con la famiglia delle iniziative intraprese.
	 	A questo riguardo, la scuola predispone, nelle forme ritenute idonee e in tempi che non
	 	superino il primo trimestre scolastico, un documento che dovrà contenere almeno le seguenti voci,
	 	articolato per le discipline coinvolte dal disturbo:
	 	\begin{itemize}
	 		\item dati anagrafici dell'alunno; 
	 		\item tipologia di disturbo;
	 		\item attività didattiche individualizzate;
	 		\item attività didattiche personalizzate;
	 		\item strumenti compensativi utilizzati;
	 		\item misure dispensative adottate;
	 		\item forme di verifica e valutazione personalizzate.
	 	\end{itemize}
	 	Nella predisposizione della documentazione in questione è fondamentale il raccordo con la famiglia, che può comunicare alla scuola eventuali osservazioni su esperienze sviluppate dallo studente anche autonomamente o attraverso percorsi extrascolastici. Sulla base di tale documentazione, nei limiti della normativa vigente, vengono predisposte le modalità delle prove e delle verifiche in corso d'anno o a fine Ciclo. Tale documentazione può acquisire la forma del Piano Didattico Personalizzato~\footcite{LineGuida2011}.
\end{description}
\end{quote}
Il colloquio con la famiglia è previsto dal DPR 122/09~\footcite{DPR_122_2009}
\begin{quote}
\begin{description}
	\item[Art 1] Oggetto del regolamento - finalità e caratteri della valutazione
	\begin{description}
		\item \mancatesto
		\item[7] Le istituzioni scolastiche assicurano alle famiglie una informazione tempestiva circa il processo di apprendimento e la valutazione degli alunni effettuata nei diversi momenti del percorso scolastico, avvalendosi, nel rispetto delle vigenti disposizioni in materia di riservatezza, anche degli strumenti offerti dalle moderne tecnologie.
	\end{description}
\end{description}
\end{quote}
Importante è comunicazione fra scuola e famiglia infatti la legge 170/2010 lo pone fra le sue finalità~\footcite{legge170}
\begin{description}
	\item[Art. 2] Finalità
	\begin{description}
		\item[g] incrementare la comunicazione e la collaborazione tra famiglia, scuola e servizi sanitari durante il percorso di istruzione e di formazione;
	\end{description}
\end{description}
Ribadito nelle Linee guida dove è scritto:
\begin{quote}
	\begin{description}
		\item[6.4] I Docenti
		
		In particolare, ogni docente, per sé e collegialmente:
		\begin{itemize}
			\item segnala alla famiglia la persistenza delle difficoltà nonostante gli interventi di recupero
			posti in essere;~\footcite{LineGuida2011}
		\end{itemize}
	\end{description}
\end{quote}
Redatto il PDP questo viene firmato dal dirigente o suo delegato, dai docenti del Consiglio di Classe e dalla Famiglia.

Uno schema di PDP è stato proposto dall'USR per il Piemonte\footcite{USRperilPiemonte2013a} in cui viene presentato un modello generale per i BES.

Nella formulazione del PEP, dato che sono trattati sicuramente dei dai sensibili, sarà cura della scuola allegare l'autorizzazione della famiglia al trattamento dei dati personali.
L'Autorità garante per la privacy si è espressa cosi nella relazione sul 2012
\begin{quote}
	un istituto scolastico pubblico ha chiesto all'Autorità se l'informazione
	relativa alla presenza di disturbi specifici di apprendimento debba considerarsi un dato
	sensibile, ai sensi dell'art. 4, comma 1, lett. d), del Codice.
	Al riguardo, l'Ufficio ha evidenziato che i disturbi specifici di apprendimento sono
	considerati, dalle ricerche più accreditate, disturbi di origine neurobiologica e, in base alla
	normativa di settore, devono essere diagnosticati dal Servizio sanitario nazionale, sicché le
	relative informazioni costituiscono dati sensibili in quanto idonei a rivelare lo stato di salute
	degli interessati, ai sensi del Codice (art. 3, l. 8 ottobre 2010, n. 170; \cit{Linee-guida per il
	diritto allo studio degli alunni e degli studenti con disturbi specifici di apprendimento}
	allegate al decreto del Ministro dell'istruzione dell'università e della ricerca n. 5669, del 12
	luglio 2011; art. 4, comma 1, lett. d), del Codice).
	Tali dati devono quindi essere trattati nel rispetto delle più stringenti regole poste dal
	Codice per tale categorie di informazioni e della specifica normativa di settore sopra
	richiamata (cfr. artt. 13, 20 e 22 del Codice; regolamento adottato dal Ministero della
	pubblica istruzione per i trattamenti dei dati sensibili e giudiziari da effettuarsi presso il
	medesimo Ministero, le istituzioni scolastiche ed educative e gli istituti regionali di ricerca
	educativa -si veda in particolare la scheda n. 4- d.m. 7 dicembre 2006, n. 305, sul quale il
	Garante ha espresso il parere di competenza in data 26 luglio 2006 [doc. web n. 1321703])
	(nota 23 gennaio 2013).\footcite{garantep2013}
\end{quote}

La CM 8/13~\footcite{cm8_2013} aggiunge che in caso di ritardo nel rilascio della certificazione, specialmente per i DSA, richiama la necessità di attivare comunque un PDP anche in attesa di essa (naturalmente verbalizzando). 

L'oggettività della certificazione appare del richiamo della R.A. n. 140 del 25 luglio 2012~\footcite{ra_140_2012}. L'accordo Stato Regioni è molto chiaro su il chi, il quando e il come deve essere una certificazione, basta leggere infatti quanto previsto:
\begin{quote}
\begin{description}
	\item[Art. 1] Attivazione del percorso diagnostico
	\begin{description}
		\item \mancatesto
		\item[3] I servizi pubblici e i soggetti accreditati ai sensi dell'art. 8 quinquies del decreto legislativo n. 502 del 1992 e s.m.i. effettuano il percorso diagnostico e il rilascio delle certificazioni in coerenza con le indicazioni della Consensus Conference. La diagnosi di DSA deve essere prodotta in tempo utile per l'attivazione delle misure didattiche e delle modalità di valutazione previste, quindi, di norma, non oltre il 31 marzo per gli alunni che frequentano gli anni terminali di ciascun ciclo scolastico, in ragione degli adempimenti connessi agli esami di Stato. Fa eccezione la prima certificazione diagnostica, che è prodotta al momento della sua formulazione, indipendentemente dal periodo dell'anno in cui ciò avviene.
		\item[4] Nel caso in cui i servizi pubblici o accreditati dal Servizio sanitario nazionale non siano in grado di garantire il rilascio delle certificazioni in tempi utili per l'attivazione delle misure didattiche e delle modalità di valutazione previste e, comunque, quando il tempo richiesto per il completamento dell'iter diagnostico superi sei mesi, con riferimento agli alunni del primo ciclo di istruzione, le Regioni, per garantire la necessaria tempestività, possono prevedere percorsi specifici per l'accreditamento di ulteriori soggetti privati ai fini dell'applicazione dell'art 3 comma 1 della legge n.170\footcite{legge170} del 2010, senza nuovi o maggiori oneri per la finanza pubblica.
	\end{description}
	\item[Art. 3]Elementi della certificazione di DSA
	\begin{description}
		\item[1] La certificazione di DSA deve evidenziare che il percorso diagnostico è stato effettuato secondo quanto previsto dalla Consensus Conference\footcite{Cons2011} e deve essere articolata e formalmente chiara. \'{E} necessario il riferimento ai codici nosografici (attualmente, tutti quelli compresi nella categoria F81: Disturbi evolutivi Specifici delle Abilità Scolastiche dell'ICD-10) e alla dicitura esplicita del DSA in oggetto (della Lettura e/o della Scrittura e/o del Calcolo).
		\item[2] La certificazione di DSA contiene le informazioni necessarie per stilare una programmazione educativa e didattica che tenga conto delle difficoltà del soggetto e preveda l'applicazione mirata delle misure previste dalla legge. La menzione della categoria diagnostica non è infatti sufficiente per la definizione di quali misure didattiche siano appropriate per il singolo soggetto. A tal fine è necessario che la certificazione di DSA contenga anche gli elementi per delineare un profilo di funzionamento (che definisce più precisamente le caratteristiche individuali con le aree di forza e di debolezza). Tale descrizione deve essere redatta in termini comprensibili e facilmente traducibile in indicazioni operative per la prassi didattica.
		\item[3]Il profilo di funzionamento è di norma aggiornato:
		\begin{description}
			\item[--]al passaggio da un ciclo scolastico all'altro e comunque, di norma, non prima di tre anni dal precedente;
			\item[--]ogni qualvolta sia necessario modificare l'applicazione degli strumenti didattici e valutativi necessari, su segnalazione della scuola alla famiglia o su iniziativa della famiglia.
		\end{description}
		\item[4] Al fine di semplificare l'iter procedurale della certificazione, con particolare riguardo alla fase di ricezione della documentazione da parte delle istituzioni scolastiche, nonché di rendere uniformi modalità e forme di attestazione della diagnosi su tutto il territorio nazionale, si fornisce, allegato al presente Accordo, un modello di certificazione ai fini disapplicazione delle misure previste dalla legge n. 170/2010~\footcite{legge170}, per essere utilizzato dalle strutture preposte~\footcite{ra_140_2012}
	\end{description}
\end{description}
\end{quote}
Ricapitolando: la conformità della documentazione al modello proposto dalla conferenza Stato Regioni aiuta il Consiglio a prendere decisioni corrette perché fornisce gli elementi necessari e va quindi letta con attenzione. Un altro fatto importante è che la certificazione ha una scadenza. %Tempi per il rinnovo standard sono Il primo (cambio di ordine di scuola), il terzo (tre anni) e probabilmente il quinto visto che la certificazione deve essere allegata al fascicolo per l'Esame di Stato, ricordando inoltre, che deve pervenire prima del 31 marzo essendo classe terminale. 

Per gli studenti con difficoltà linguistiche e culturali vale la CM n. 24/2006~\footcite{lin_1_marzo_2006} e la CM n. 2/2010~\footcite{CM_2_2010} che precisa quali iniziative intraprendere 
\begin{quote}
	\begin{description}
		\item[d] Competenze linguistiche degli alunni stranieri
		
		Per assicurare agli studenti di nazionalità non italiana, soprattutto se di recente immigrazione e di ingresso nella scuola in corso d'anno, la possibilità di seguire un efficace
		processo di insegnamento-apprendimento -– e quindi una loro effettiva integrazione -– le scuole
		attivano dal prossimo anno 2010/2011 iniziative di alfabetizzazione linguistica anche
		utilizzando le risorse che saranno messe a disposizione dalla legge 440/97~\footcite{Legge_440_97} e con opportune
		scelte di priorità nella finalizzazione delle disponibilità finanziarie relative alle aree a forte
		processo migratorio.
		In merito, sempre nel rispetto autonomia delle scuole, si suggeriscono le seguenti
		misure, peraltro già richiamate dalla normativa vigente~\footcite[Art. 45 comma 4 \caporali{Il collegio dei docenti definisce, in relazione al livello di competenza dei singoli alunni stranieri, il necessario
		adattamento dei programmi di insegnamento; allo scopo possono essere adottati specifici interventi individualizzati o per gruppi di alunni, per facilitare l'apprendimento della lingua italiana, utilizzando, ove possibile, le risorse
		professionali della scuola. Il consolidamento della conoscenza e della pratica della lingua italiana può essere realizzata altresì mediante l'attivazione di corsi intensivi di lingua italiana sulla base di specifici progetti, anche nell'ambito delle
		attività aggiuntive di insegnamento per l'arricchimento dell'offerta formativa.}]{DPR_394_1999}:
		\begin{itemize}
			\item 	attivazione di moduli intensivi, laboratori linguistici, percorsi personalizzati di
			lingua italiana per gruppi di livello sia in orario curricolare (anche in ore di
			insegnamento di altre discipline) sia in corsi pomeridiani realizzati grazie all'arricchimento dell'offerta formativa);
			\item utilizzo della quota di flessibilità del 20 per cento, destinato per corsi di lingua
			italiana di diverso livello (di progressiva alfabetizzazione per gli allievi stranieri privi
			delle necessarie competenze di base; di recupero, mantenimento e potenziamento per tutti gli altri, stranieri e non);
			\item partecipazione a progetti mirati all'insegnamento della lingua italiana come lingua
			seconda, utilizzando eventualmente risorse professionali interne o di rete, offerti e/o
			organizzati dal territorio;
			\item 	possibilità per gli allievi stranieri neo arrivati in corso d'anno di essere inseriti nella
			scuola - se ritenuto utile e/o necessario anche in una classe non corrispondente
			all'età anagrafica – per attività finalizzate a un rapporto iniziale sia con la lingua
			italiana, sia con le pratiche e le abitudini della vita scolastica ovvero di frequentare
			un corso intensivo propedeutico all'ingresso nella classe di pertinenza (anche in periodi – giugno/luglio/inizio settembre in cui non si tiene la normale attività scolastica).
		\end{itemize}
	\mancatesto
		La scuola potrà infine favorire, anche d'intesa con soggetti del privato sociale, situazioni
		di relazioni, di socializzazioni, di esperienze extracurricolari in cui gli alunni stranieri potranno
		sviluppare in ambiente non formale e con coetanei la conoscenza e l'uso della lingua italiana.
	\end{description}
\end{quote}
In caso di Istruzione domiciliare o ospedaliera si può far riferimento al vademecum del 2003~\footcite{Vad_ist_dom} o alla circolare 24 del 2011~\footcite{cm_24_11} che da istruzioni operative su come operare. Concludendo, il Consiglio di classe: 
\begin{enumerate}
	\item in presenza di elementi oggettivi e verbalizzabili
	\item in concerto con la scuola e la famiglia
	\item definisce un \glslink{pdpa}{PDP} se necessario
	\item che può avere una scadenza
	\item che contiene le strategie più adeguate
	\item il PDP prevede un sistema di monitoraggio
	\item il PDP prevede un sistema di documentazione
\end{enumerate}
\section{Didattica}
\label{sub:Didattic}
%Lo strumento principe ma non esclusivo è il PDP. Nel redigerlo bisogna ricordare la direttiva del 27 dicembre 2012~\footcite{dir27Dic2012} e la CM 8/2013~\footcite{cm8_2013}, suggeriscono di utilizzare come modello ciò che è previsto per i DSA ovviamente adattandolo alla situazione.
 La didattica deve essere individualizzata e personalizzata. 
Cosa si intende per individualizzato e personalizzato è chiarito nelle Linee guida per il diritto allo studio degli alunni e degli studenti con disturbi specifici di apprendimento:
\begin{quote}
\begin{description}
	\item[3] La didattica individualizzata e personalizzata.
	strumenti compensativi e misure dispensative.
	
		\mancatesto I termini individualizzata e personalizzata non sono da considerarsi sinonimi. In letteratura, la discussione in merito è molto ampia e articolata. Ai fini di questo documento, è possibile individuare alcune definizioni che, senza essere definitive, possono consentire di ragionare con un vocabolario comune.
		E’ comunque preliminarmente opportuno osservare che la Legge 170/2010~\footcite{legge170} insiste più volte
		sul tema della didattica individualizzata e personalizzata come strumento di garanzia del diritto allo
		studio, con ciò lasciando intendere la centralità delle metodologie didattiche, e non solo degli
		strumenti compensativi e delle misure dispensative, per il raggiungimento del successo formativo
		degli alunni con DSA.
		\cit{Individualizzato} è l'intervento calibrato sul singolo, anziché sull'intera classe o sul piccolo
		gruppo, che diviene \cit{personalizzato} quando è rivolto ad un particolare discente.
		Più in generale - contestualizzandola nella situazione didattica dell'insegnamento in classe -
		l'azione formativa individualizzata pone obiettivi comuni per tutti i componenti del gruppo - classe,
		ma è concepita adattando le metodologie in funzione delle caratteristiche individuali dei discenti,
		con l'obiettivo di assicurare a tutti il conseguimento delle competenze fondamentali del curricolo,
		comportando quindi attenzione alle differenze individuali in rapporto ad una pluralità di dimensioni.
		L'azione formativa personalizzata ha, in più, l'obiettivo di dare a ciascun alunno l'opportunità
		di sviluppare al meglio le proprie potenzialità e, quindi, può porsi obiettivi diversi per ciascun
		discente, essendo strettamente legata a quella specifica ed unica persona dello studente a cui ci
		rivolgiamo. Si possono quindi proporre le seguenti definizioni.
		La didattica individualizzata consiste nelle attività di recupero individuale che può svolgere
		l'alunno per potenziare determinate abilità o per acquisire specifiche competenze, anche nell'ambito
		delle strategie compensative e del metodo di studio; tali attività individualizzate possono essere
		realizzate nelle fasi di lavoro individuale in classe o in momenti ad esse dedicati, secondo tutte le
		forme di flessibilità del lavoro scolastico consentite dalla normativa vigente.
		La didattica personalizzata, invece, anche sulla base di quanto indicato nella Legge 53/2003~\footcite{Legge_53_2003} e
		nel Decreto legislativo 59/2004~\footcite{DL_59_2004}, calibra l'offerta didattica, e le modalità relazionali, sulla specificità
		ed unicità a livello personale dei bisogni educativi che caratterizzano gli alunni della classe,
		considerando le differenze individuali soprattutto sotto il profilo qualitativo; si può favorire, così,
		l'accrescimento dei punti di forza di ciascun alunno, lo sviluppo consapevole delle sue 'preferenze'
		e del suo talento. Nel rispetto degli obiettivi generali e specifici di apprendimento, la didattica
		personalizzata si sostanzia attraverso l'impiego di una varietà di metodologie e strategie didattiche,
		tali da promuovere le potenzialità e il successo formativo in ogni alunno: l'uso dei mediatori
		didattici (schemi, mappe concettuali, etc.), l'attenzione agli stili di apprendimento, la calibrazione
		degli interventi sulla base dei livelli raggiunti, nell'ottica di promuovere un apprendimento
		significativo.
		La sinergia fra didattica individualizzata e personalizzata determina dunque, per l'alunno e lo
		studente con DSA, le condizioni più favorevoli per il raggiungimento degli obiettivi di
		apprendimento\mancatesto
\end{description}
\end{quote}
Bisogna provvedere a strumenti compensativi che nelle Linee Guida vengono definiti:
\begin{quote}
	\begin{description}[style=sameline]
		\item\mancatesto
		
		Gli strumenti compensativi sono strumenti didattici e tecnologici che sostituiscono o facilitano
		la prestazione richiesta nell'abilità deficitaria
		\mancatesto
				
		Tali strumenti sollevano l'alunno o lo studente con DSA da una prestazione resa difficoltosa
		dal disturbo, senza peraltro facilitargli il compito dal punto di vista cognitivo. L'utilizzo di tali
		strumenti non è immediato e i docenti - anche sulla base delle indicazioni del referente di istituto -
		avranno cura di sostenerne l'uso da parte di alunni e studenti con DSA\mancatesto
	\end{description}
\end{quote} 
e a misure dispensative 
\begin{quote}
	\begin{description}[style=sameline]
\item
\mancatesto
	Le misure dispensative sono invece interventi che consentono all'alunno o allo studente di non
	svolgere alcune prestazioni che, a causa del disturbo, risultano particolarmente difficoltose e che
	non migliorano l'apprendimento.
\mancatesto
	L'adozione delle misure dispensative, al fine di non creare percorsi immotivatamente
	facilitati, che non mirano al successo formativo degli alunni e degli studenti con DSA, dovrà essere
	sempre valutata sulla base dell'effettiva incidenza del disturbo sulle prestazioni richieste, in modo
	tale, comunque, da non differenziare, in ordine agli obiettivi, il percorso di apprendimento
	dell'alunno o dello studente in questione.\mancatesto
	\end{description}
\end{quote}
per i tempi le linee guida consigliano 
\begin{quote}
		\begin{description}[style=sameline]
			\item %[3] La didattica individualizzata e personalizzata. strumenti compensativi e misure dispensative
			
			\mancatesto Consentire all'alunno o allo studente con DSA di usufruire di maggior tempo per
			lo svolgimento di una prova, o di poter svolgere la stessa su un contenuto comunque
			disciplinarmente significativo ma ridotto, trova la sua ragion d'essere nel fatto che il disturbo li
			impegna per più tempo dei propri compagni nella fase di decodifica degli items della prova. A
			questo riguardo, gli studi disponibili in materia consigliano di stimare, tenendo conto degli indici di
			prestazione dell'allievo, in che misura la specifica difficoltà lo penalizzi di fronte ai compagni e di
			calibrare di conseguenza un tempo aggiuntivo o la riduzione del materiale di lavoro. In assenza di
			indici più precisi, una quota del 30\% in più appare un ragionevole tempo aggiuntivo~\footcite{LineGuida2011}\mancatesto.
		\end{description}
\end{quote}
\section{Valutazione}
\label{sub:valutazione}
Riportiamo di seguito alcune norme che, pur non parlando esplicitamente di BES, riguardano soggetti che ricadono in questa catalogazione.
\subsection{Valutazione alunni 104}
Per quanto riguarda gli alunni certificati 104/92 vale la pena di ricordare quanto riportato nelle \caporali{Linee guida sull'integrazione scolastica degli alunni con disabilità}~\footcite{LineGuida2009}
parlando di valutazione viene riportato
\begin{quote}
	\begin{description}
		\item[2.4] La valutazione
		
		La valutazione in decimi va rapportata al P.E.I., che costituisce il punto di riferimento per le attività educative a favore dell'alunno con disabilità. Si rammenta
		inoltre che la valutazione in questione dovrà essere sempre considerata come valutazione dei processi e non solo come valutazione della performance.
	\end{description}
\end{quote}
% % % % % % % % % % % % % % % % % % % % % % % % % 
L'Ordinanza ministeriale 90/2001~\footcite{OM_90_2001} precisa 
	\begin{quote}
		\begin{description}
			\item[Art. 15] 	Valutazione degli alunni in situazione di handicap
			\begin{enumerate}
				\item Nei confronti degli alunni con minorazioni fisiche e sensoriali non si procede, di norma, ad alcuna
				valutazione differenziata; è consentito, tuttavia, l'uso di particolari strumenti didattici
				appositamente individuati dai docenti, al fine di accertare il livello di apprendimento non
				evidenziabile attraverso un colloquio o prove scritte tradizionali.
				\item Per gli alunni in situazione di handicap psichico la valutazione, per il suo carattere formativo ed
				educativo e per l'azione di stimolo che esercita nei confronti dell'allievo, deve comunque aver
				luogo. Il Consiglio di classe, in sede di valutazione periodica e finale, sulla scorta del Piano
				Educativo Individualizzato a suo tempo predisposto con la partecipazione dei genitori nei modi e
				nei tempi previsti dalla C. M. 258/83\footcite{CM_258_1983}, esamina gli elementi di giudizio forniti da ciascun insegnante
				sui livelli di apprendimento raggiunti, anche attraverso l'attività di integrazione e di sostegno,
				verifica i risultati complessivi rispetto agli obiettivi prefissati dal Piano Educativo
				Individualizzato.
				\item Ove il Consiglio di classe riscontri che l'allievo abbia raggiunto un livello di preparazione conforme
				agli obiettivi didattici previsti dai programmi ministeriali o, comunque, ad essi globalmente
				corrispondenti, decide in conformità dei precedenti artt.12 e 13.
				\item Qualora, al fine di assicurare il diritto allo studio ad alunni in situazione di handicap psichico e,
				eccezionalmente, fisico e sensoriale, il piano educativo individualizzato sia diversificato in funzione
				di obiettivi didattici e formativi non riconducibili ai programmi ministeriali, il Consiglio di classe,
				fermo restando l'obbligo della relazione di cui al paragrafo 8 della Circolare ministeriale n. 262 del
				22 settembre 1988\footcite{cm_262_1988}, valuta i risultati dell'apprendimento, con l'attribuzione di voti relativi unicamente
				allo svolgimento del citato piano educativo individualizzato e non ai programmi ministeriali. Tali voti
				hanno, pertanto, valore legale solo ai fini della prosecuzione degli studi per il perseguimento degli
				obiettivi del piano educativo individualizzato. I predetti alunni possono, di conseguenza, essere
				ammessi alla frequenza dell'anno successivo o dichiarati ripetenti anche per tre volte in forza del
				disposto di cui all’art.316 del D.Lvo 16.4.1994, n.297\footcite{dl_297_1994}. In calce alla pagella degli alunni medesimi,
				deve essere apposta l'annotazione secondo la quale la votazione è riferita al P.E.I e non ai
				programmi ministeriali ed è adottata ai sensi dell'art.14 della presente Ordinanza. Gli alunni
				valutati in modo differenziato come sopra possono partecipare agli esami di qualifica professionale
				e di licenza di maestro d'arte, svolgendo prove differenziate, omogenee al percorso svolto,
				finalizzate all'attestazione delle competenze e delle abilità acquisite. Tale attestazione può
				costituire, in particolare quando il piano educativo personalizzato preveda esperienze di
				orientamento, di tirocinio, di stage, di inserimento lavorativo, un credito formativo spendibile nella
				frequenza di corsi di formazione professionale nell'ambito delle intese con le Regioni e gli Enti
				locali. In caso di ripetenza, il Consiglio di classe riduce ulteriormente gli obiettivi didattici del piano
				educativo individualizzato. Non può, comunque, essere preclusa ad un alunno in situazione di
				handicap fisico, psichico o sensoriale, anche se abbia sostenuto gli esami di qualifica o di licenza
				di maestro d'arte, conseguendo l'attestato di cui sopra, l'iscrizione e la frequenza anche per la
				terza volta alla stessa classe. Qualora durante il successivo anno scolastico vengano accertati
				livelli di apprendimento corrispondenti agli obiettivi previsti dai programmi ministeriali, il Consiglio di
				classe delibera in conformità dei precedenti artt 12 e 13,senza necessità di prove di idoneità
				relative alle discipline dell'anno o degli anni precedenti, tenuto conto che il Consiglio medesimo
				possiede già tutti gli elementi di valutazione. Gli alunni in situazione di handicap che svolgono piani educativi individualizzati differenziati, in possesso dell'attestato di credito formativo, possono
				iscriversi e frequentare, nel quadro dei principi generali stabiliti dall’art.312 e seguenti del D.Lvo n.297/1994\footcite{dl_297_1994}, le classi successive, sulla base di un progetto – che può prevedere anche percorsi
				integrati di istruzione e formazione professionale, con la conseguente acquisizione del relativo
				credito formativo in attuazione del diritto allo studio costituzionalmente garantito. Per gli alunni
				medesimi, che al termine della frequenza dell'ultimo anno di corso, essendo in possesso di crediti
				formativi, possono sostenere l'esame di Stato sulla base di prove differenziate coerenti con il
				percorso svolto e finalizzate solo al rilascio dell'attestazione di cui all'art.13 del Regolamento, si fa
				rinvio a quanto previsto dall'art.17, comma 4, dell'O.M. n.29/2001\footcite{OM_29_2001}.
			\item Qualora un Consiglio di classe intenda adottare la valutazione differenziata di cui sopra, deve
			darne immediata notizia alla famiglia fissandole un termine per manifestare un formale assenso, in
			mancanza del quale la modalità valutativa proposta si intende accettata. In caso di diniego
			espresso, l'alunno non può essere considerato in situazione di handicap ai soli fini della
			valutazione, che viene effettuata ai sensi dei precedenti artt.12 e 13.
			\item Per gli alunni che seguono un Piano educativo Individualizzato differenziato, ai voti riportati nello
			scrutinio finale e ai punteggi assegnati in esito agli esami si aggiunge, nelle certificazioni rilasciate,
			l'indicazione che la votazione è riferita al P.E.I e non ai programmi ministeriali.
			\end{enumerate}
		\end{description}
		 \mancatesto
	\end{quote}
ribadito nell'articolo 9 del D.P.R. 22/6/2009, n.122~\footcite{DPR_122_2009}
\begin{quote}
\begin{description}\item [Art. 9] Valutazione degli alunni con disabilità
	\begin{description}
		\item[1]La valutazione degli alunni con disabilità certificata nelle
		forme e con le modalità previste dalle disposizioni in vigore è riferita al comportamento, alle discipline e alle attività svolte sulla base del piano educativo individualizzato previsto
		dall'articolo 314, comma 4, del testo unico di cui al ed è espressa con voto in decimi decreto
		legislativo n. 297 del 1994\footcite{dl_297_1994},
		secondo le modalità e condizioni indicate nei precedenti articoli.
	\end{description}
\end{description}
\end{quote}
Interessante è il passo della nota Prot. n. 1075/C27 dell'USR della Liguria del 21.2.2011 che ha per oggetto \caporali{La continuità educativa a favore degli alunni disabili}:~\footcite{Prot_1075}
\begin{quote}
	\mancatesto
	 Nel caso di alunni con esigenze educative particolari, si ricorda che nulla vieta che il PEI possa prevedere un percorso fortemente individualizzato, senza che questo comporti la necessità di rallentare o posticipare l'avvio del percorso scolastico. Analoga attenzione deve essere posta alla regolarità e fluidità del percorso scolastico, che deve consentire, anche agli alunni disabili, di poter usufruire di tutte le opportunità che il sistema scolastico e formativo offrono. Con ciò non si esclude la possibilità di ripetenza, ma pare opportuno ricordare che la promozione o meno dell'alunno, sia pure disabile, è competenza esclusiva degli organi collegiali nella sola componente docente.  L'alunno sarà valutato in riferimento non ad obiettivi standard, ma agli obiettivi didattici previsti espressamente per lui nel PEI. Non si ritiene che l'alunno possa essere respinto qualora nella definizione degli obiettivi del PEI siano state fissate mete non raggiungibili per l'alunno stesso.  La valutazione, e quindi l'esito scolastico, non può essere condizionato da considerazioni e pregiudizi rispetto all'idoneità o meno della struttura di futura frequenza.
\end{quote}
\subsection{Valutazione DSA}
Per i DSA vale, l'articolo 10 del D.P.R. 22/6/2009, n.122~\footcite{DPR_122_2009}
\begin{quote}
\begin{description}
	\item[Art. 10] Valutazione degli alunni con difficoltà specifica di apprendimento (DSA)
	\begin{description}
		\item[1] Per gli alunni con difficoltà specifiche di apprendimento (DSA) adeguatamente certificate, la valutazione e la verifica degli apprendimenti, comprese quelle effettuate in sede di esame conclusivo dei cicli, devono tenere conto delle specifiche situazioni soggettive di tali alunni; a tali fini, nello svolgimento dell'attività didattica e delle prove di esame, sono adottati, nell'ambito delle risorse finanziarie disponibili a legislazione vigente, gli strumenti metodologico-didattici compensativi e dispensativi ritenuti più idonei.
	\end{description}
\end{description}
\end{quote}
 la legge 170/2010~\footcite{legge170} pone l'accento sull'adeguatezza della valutazione:
\begin{quote}
\begin{description}
	\item[Art 2] Finalità
	\begin{description}
		\item[d] adottare forme di verifica e di valutazione adeguate alle necessità formative degli studenti;
	\end{description}
	\item [Art 5] Misure educative e didattiche di supporto
	\begin{description}
		\item[4] Agli studenti con DSA sono garantite, durante il percorso di istruzione e di formazione scolastica e
		universitaria, adeguate forme di verifica e di valutazione, anche per quanto concerne gli esami di
		Stato e di ammissione all'università nonché gli esami universitari.
	\end{description}
\end{description}
\end{quote}
Importante è quanto riportato nel DM 5669/2011~\footcite{decreto5669_2011}
\begin{quote}
	\begin{description}
		\item[Art. 6] Forme di verifica e di valutazione
		\begin{description}
		\item[1] La valutazione scolastica, periodica e finale, degli alunni e degli studenti con DSA deve
		essere coerente con gli interventi pedagogico-didattici di cui ai precedenti articoli.
		\item[2] Le Istituzioni scolastiche adottano modalità valutative che consentono all'alunno o allo
		studente con DSA di dimostrare effettivamente il livello di apprendimento raggiunto,
		mediante l'applicazione di misure che determinino le condizioni ottimali per l'espletamento
		della prestazione da valutare - relativamente ai tempi di effettuazione e alle modalità di
		strutturazione delle prove - riservando particolare attenzione alla padronanza dei contenuti
		disciplinari, a prescindere dagli aspetti legati all'abilità deficitaria.
		\item[4] Le Istituzioni scolastiche attuano ogni strategia didattica per consentire ad alunni e studenti
		con DSA l'apprendimento delle lingue straniere. A tal fine valorizzano le modalità attraverso
		cui il discente meglio può esprimere le sue competenze, privilegiando l'espressione orale,
		nonché ricorrendo agli strumenti compensativi e alle misure dispensative più opportune.
		Le prove scritte di lingua straniera sono progettate, presentate e valutate secondo modalità
		compatibili con le difficoltà connesse ai DSA.
		\item[5] Fatto salvo quanto definito nel comma precedente, si possono dispensare alunni e studenti
		dalle prestazioni scritte in lingua straniera in corso d'anno scolastico e in sede di esami di
		Stato, nel caso in cui ricorrano tutte le condizioni di seguito elencate:
		\begin{itemize}
				\item certificazione di DSA attestante la gravità del disturbo e recante esplicita richiesta di
				dispensa dalle prove scritte;
				\item richiesta di dispensa dalle prove scritte di lingua straniera presentata dalla famiglia o
				dall'allievo se maggiorenne;
				\item approvazione da parte del consiglio di classe che confermi la dispensa in forma temporanea
				o permanente, tenendo conto delle valutazioni diagnostiche e sulla base delle risultanze
				degli interventi di natura pedagogico-didattica, con particolare attenzione ai percorsi di
				studio in cui l'insegnamento della lingua straniera risulti caratterizzante (liceo linguistico,
				istituto tecnico per il turismo, ecc.).
			\end{itemize}
			In sede di esami di Stato, conclusivi del primo e del secondo ciclo di istruzione, modalità e
			contenuti delle prove orali – sostitutive delle prove scritte – sono stabiliti dalle Commissioni,
			sulla base della documentazione fornita dai consigli di classe.
			I candidati con DSA che superano l'esame di Stato conseguono il titolo valido per l'iscrizione
			alla scuola secondaria di secondo grado ovvero all'università
			\item[6]Solo in casi di particolari gravità del disturbo di apprendimento, anche in comorbilità con
			altri disturbi o patologie, risultanti dal certificato diagnostico, l'alunno o lo studente possono –
			su richiesta delle famiglie e conseguente approvazione del consiglio di classe - essere esonerati
			dall'insegnamento delle lingue straniere e seguire un percorso didattico differenziato.
			In sede di esami di Stato, i candidati con DSA che hanno seguito un percorso didattico
			differenziato e sono stati valutati dal consiglio di classe con l'attribuzione di voti e di un
			credito scolastico relativi unicamente allo svolgimento di tale piano, possono sostenere prove
			differenziate, coerenti con il percorso svolto, finalizzate solo al rilascio dell'attestazione di cui
			all'art.13 del D.P.R. n.323/1998~\footcite{DPR_323_1998}.
		\end{description}
	\end{description}
\end{quote}
Continuano le linee guida~\footcite{LineGuida2011} che sono parte integrante del DM 5669/2011 che pongono nella premessa 
\begin{quote}
	\begin{description}
		\item[] Premessa
		
		\mancatesto assegnando al sistema nazionale di
		 istruzione e agli atenei il compito di individuare le forme didattiche e le modalità di valutazione più adeguate affinché alunni e studenti con DSA possano raggiungere il successo formativo.
	\end{description}
\end{quote} 
sempre nelle linee guida parlando del PDP
\begin{quote}
	\begin{description}
		\item[3.1]Documentazione dei percorsi didattici
		
		forme di verifica e valutazione personalizzate
	\end{description}
\end{quote}
e inoltre
\begin{quote}
\begin{description}
	\item[4.3.1] Disturbo di lettura
	
	In fase di verifica e di valutazione, lo studente con dislessia può usufruire di tempi aggiuntivi
	per l'espletamento delle prove o, in alternativa e comunque nell'ambito degli obiettivi disciplinari
	previsti per la classe, di verifiche con minori richieste.
	Nella valutazione delle prove orali e in ordine alle modalità di interrogazione si dovrà tenere
	conto delle capacità lessicali ed espressive proprie dello studente.
	\item[4.3.2] Disturbo di scrittura
	
	In via generale, comunque, la valutazione si soffermerà soprattutto sul contenuto disciplinare piuttosto che sulla forma ortografica e sintattica.
	
	\mancatesto Per quanto concerne le misure dispensative, oltre a tempi più lunghi per le verifiche scritte o a
	una quantità minore di esercizi, gli alunni con disgrafia e disortografia sono dispensati dalla
	valutazione della correttezza della scrittura e, anche sulla base della gravità del disturbo, possono
	accompagnare o integrare la prova scritta con una prova orale attinente ai medesimi contenuti.
	
	\item[4.4] Didattica per le lingue straniere
	
	In relazione alle forme di valutazione, per quanto riguarda la comprensione (orale o scritta), sarà valorizzata la capacità di cogliere il senso generale del messaggio; in fase di produzione sarà dato più rilievo all'efficacia comunicativa, ossia alla capacità di farsi comprendere in modo chiaro,
	anche se non del tutto corretto grammaticalmente.
	\item[7.1] I contenuti della formazione
	
	 Forme adeguate di verifica e di valutazione.
		
	La valutazione deve concretizzarsi in una prassi che espliciti concretamente le modalità di
			differenziazione a seconda della disciplina e del tipo di compito, discriminando fra ciò che è
			espressione diretta del disturbo e ciò che esprime l'impegno dell'allievo e le conoscenze
			effettivamente acquisite.

\end{description}
\end{quote}
\subsection{Valutazione difficoltà linguistiche }
 Le linee guida per l'inserimento degli studenti stranieri~\footcite{lin_1_marzo_2006} dedicano alla valutazione il seguente paragrafo: 
 \begin{quote}
 	\begin{description}
 		\item[8] La valutazione
 		
 		La valutazione degli alunni stranieri, in particolare di coloro che si possono definire
 		neo-arrivati, pone diversi ordini di problemi, dalle modalità di valutazione a quelle di
 			certificazione, alla necessità di tener conto del singolo percorso di apprendimento. La pur
 			significativa normativa esistente sugli alunni con cittadinanza non italiana non fornisce
 			indicazioni specifiche a proposito della valutazione degli stessi.
 			Dall'emanazione della legge n. 517 del 4 agosto 1977 ad oggi, l'approccio alla
 			valutazione nella scuola è positivamente cambiato. Accanto alla funzione certificativa si è
 			andata sempre più affermando la funzione regolativa in grado di consentire, sulla base delle
 			informazioni via via raccolte, un continuo adeguamento delle proposte di formazione alle
 			reali esigenze degli alunni e ai traguardi programmati per il miglioramento dei processi e dei
 			risultati, sollecitando, altresì, la partecipazione degli alunni e delle famiglie al processo di
 			apprendimento. L'art. 4 del DPR n. 275/1999~\footcite{DPR_275_1999}, relativo all'autonomia didattica delle
 			istituzioni scolastiche, assegna alle stesse la responsabilità di individuare le modalità e i
 			criteri di valutazione degli alunni, prevedendo altresì che esse operino \cit{nel rispetto della
 normativa nazionale}.
 			Il riferimento più congruo a questo tema lo si ritrova nell'art. 45, comma 4, del DPR
 			n 394 del 31 agosto 1999~\footcite{DPR_394_1999} che così recita \cit{il collegio dei docenti definisce, in relazione al
 			livello di competenza dei singoli alunni stranieri, il necessario adattamento dei programmi
 			di insegnamento \dots}. Benché la norma non accenni alla valutazione, ne consegue che il
 			possibile adattamento dei programmi per i singoli alunni comporti un adattamento della
 			valutazione, anche in considerazione degli orientamenti generali su questo tema, espressi in
 			circolari e direttive, che sottolineano fortemente l'attenzione ai percorsi personali degli
 			alunni. Questa norma va ora inquadrata nel nuovo assetto ordinamentale ed educativo
 			esplicitato dalle \cit{Indicazioni Nazionali per i piani di studio personalizzati} e con le finalità
 			del \cit{Profilo educativo dello studente} che costituiscono il nuovo impianto pedagogico, didattico ed organizzativo della scuola italiana, basato sulla L 53/03, art. 3~\footcite{Legge_53_2003}, relativi in particolare alla valutazione.
 			Per il consiglio di classe che deve valutare alunni stranieri inseriti nel corso dell'anno
 			scolastico – per i quali i piani individualizzati prevedono interventi di educazione
 			linguistica e di messa a punto curricolare - diventa fondamentale conoscere, per quanto
 			possibile, la storia scolastica precedente, gli esiti raggiunti, le caratteristiche delle scuole
 			frequentate, le abilità e le competenze essenziali acquisite. In questo contesto, che privilegia
 			la valutazione formativa rispetto a quella \cit{certificativa} si prendono in considerazione il percorso dell'alunno, i passi realizzati, gli obiettivi possibili, la motivazione e l'impegno e,
 			soprattutto, le potenzialità di apprendimento dimostrate. In particolare, nel momento in cui
 			si decide il passaggio o meno da una classe all'altra o da un grado scolastico al successivo,
 			occorre far riferimento a una pluralità di elementi fra cui non può mancare una previsione di
 			sviluppo dell'alunno. Emerge chiaramente come nell'attuale contesto normativo vengono
 			rafforzati il ruolo e la responsabilità delle istituzioni scolastiche autonome e dei docenti
 			nella valutazione degli alunni.
 	\end{description}
 \end{quote}
\subsection{Monitoraggio valutazione}
 La valutazione va adattata alla situazione. Anche le griglie di valutazione di conseguenza devono essere adeguate, di volta in volta, alle necessità dell'alunno e questo non solo per un uso contingente delle medesime ma anche in funzione del loro uso durante l' Esame di Stato. Infatti l'OM 13/2013~\footcite{OM_13_13} dedica un articolo sulla valutazione dei DSA che al Art. 18 recita
\begin{quote}
	Art.18 ESAME DEI CANDIDATI CON DSA
	\begin{description}
		\item[1] La Commissione d'esame – sulla base di quanto previsto dall'articolo 10 del D.P.R. 22/6/2009, n.122~\footcite{DPR_122_2009} e dal relativo DM n.5669 12 luglio 2011~\footcite{decreto5669_2011} di attuazione della Legge 8 ottobre 2010, n. 170~\footcite{legge170}, recante Nuove norme in materia di disturbi specifici di apprendimento in ambito scolastico – nonché dalle Linee Guida allegate al citato DM n. 5669/2011~\footcite{decreto5669_2011}, – considerati gli elementi forniti dal Consiglio di classe, terrà in debita considerazione le specifiche situazioni soggettive, adeguatamente certificate, relative ai candidati con disturbi specifici di apprendimento (DSA), in particolare, le modalità didattiche e le forme di valutazione individuate nell'ambito dei percorsi didattici individualizzati e personalizzati. A tal fine il Consiglio di classe inserisce nel documento del 15 maggio di cui al DPR n.323/1998~\footcite{DPR_323_1998} il Piano Didattico Personalizzato o altra documentazione predisposta ai sensi dell'art.5 del DM n. 5669 del 12 luglio 2011~\footcite{decreto5669_2011}. Sulla base di tale documentazione e di tutti gli elementi forniti dal Consiglio di classe, le Commissioni predispongono adeguate modalità di svolgimento delle prove scritte e orali. 
		\mancatesto
		\item[4]Per altre situazioni di alunni con difficoltà di apprendimento di varia natura, formalmente individuati dal Consiglio di classe, devono essere fornite dal medesimo Organo utili e opportune indicazioni per consentire a tali alunni di sostenere adeguatamente l'esame di Stato.
	\end{description}
\end{quote}
Importante è anche il comma 4 che sembra un timido accenno alla normativa su i BES.