\chapter{Cosa fa la scuola}
\label{sub:gruppolavoroinclusione}
\section{Gruppo di lavoro per l'inclusione}
La legge 104/1992 prevede a livello di singola scuola:~\footcite[Art. 15 comma 2]{Legge_104_92}
\begin{quote}
	\begin{description}
		\item[Art. 15] Gruppi di lavoro per l'integrazione scolastica. 
		\begin{description}
			\item[2] Presso ogni circolo didattico ed istituto di scuola secondaria di primo e secondo grado sono
			costituiti gruppi di studio e di lavoro composti da insegnanti, operatori dei servizi, familiari e
			studenti con il compito di collaborare alle iniziative educative e di integrazione predisposte dal
			piano educativo.
		\end{description}
	\end{description}
\end{quote}
il cosiddetto GLHI. La circolare 8/2013~\footcite{cm8_2013} rinomina questo gruppo di lavoro in GLI ossia in Gruppo di Lavoro per l'Inclusione e lo integra con: 
\begin{quote}
	funzioni strumentali, insegnanti per il sostegno, \glslink{aeca}{AEC}, assistenti alla
	comunicazione, docenti \cit{disciplinari} con esperienza e/o formazione specifica o con compiti di
	coordinamento delle classi, genitori ed esperti istituzionali o esterni in regime di
	convenzionamento con la scuola.
\end{quote}
Il GLI svolge le seguenti funzioni 
\begin{quote}
	\begin{itemize}
		\item rilevazione dei BES presenti nella scuola;
		\item raccolta e documentazione degli interventi didattico-educativi posti in essere anche in
		funzione di azioni di apprendimento organizzativo in rete tra scuole e/o in rapporto con
		azioni strategiche dell'Amministrazione;
		\item focus/confronto sui casi, consulenza e supporto ai colleghi sulle strategie/metodologie di
		gestione delle classi;
		\item rilevazione, monitoraggio e valutazione del livello di inclusività della scuola;
		\item raccolta e coordinamento delle proposte formulate dai singoli GLH Operativi sulla base
		delle effettive esigenze, ai sensi dell’art. 1, c. 605, lettera b, della legge 296/2006~\footcite[b) il perseguimento della sostituzione del criterio previsto dall'articolo 40, comma 3, della legge 27 dicembre 1997, n. 449, con l'individuazione di organici corrispondenti alle effettive esigenze rilevate, tramite una stretta collaborazione tra regioni, uffici scolastici regionali, aziende sanitarie locali e istituzioni scolastiche, attraverso certificazioni idonee a definire appropriati interventi formativi;]{Legge_296_2006}, tradotte
		in sede di definizione del PEI come stabilito dall'art. 10 comma 5 della Legge 30 luglio
		2010 n. 122~\footcite[I soggetti di cui all'articolo 12, comma 5, della legge 5 febbraio 1992, n. 104 (GLH), in sede di formulazione del piano educativo individualizzato, elaborano proposte relative all'individuazione delle risorse necessarie, ivi compresa l'indicazione del numero delle ore di sostegno, che devono essere esclusivamente finalizzate all'educazione e all'istruzione, restando a carico degli altri soggetti istituzionali la fornitura delle altre risorse professionali e materiali necessarie per l'integrazione e l'assistenza dell'alunno disabile richieste dal piano educativo individualizzato]{Legge_122_2010};
		\item elaborazione di una proposta di Piano Annuale per l'Inclusività riferito a tutti gli alunni con BES, da redigere al termine di ogni anno scolastico (entro il mese di Giugno).
		
		A tale scopo, il Gruppo procederà ad un'analisi delle criticità e dei punti di forza degli interventi di inclusione scolastica operati nell'anno appena trascorso e formulerà un'ipotesi globale di utilizzo funzionale delle risorse specifiche, istituzionali e non, per incrementare il livello di inclusività generale della scuola nell'anno successivo. Il Piano sarà quindi discusso e deliberato in Collegio dei Docenti e inviato ai competenti Uffici degli UUSSRR, nonché ai GLIP e al GLIR, per la richiesta di organico di sostegno, e alle altre istituzioni territoriali come proposta di assegnazione delle risorse di competenza, considerando anche gli Accordi di Programma in vigore o altre specifiche intese sull'integrazione scolastica sottoscritte con gli Enti Locali. A seguito di ciò, gli
		Uffici Scolastici regionali assegnano alle singole scuole globalmente le risorse di sostegno secondo quanto stabilito dall'art 19 comma 11 della Legge n. 111/2011\footcite{legge_111_2011}.
		
		Nel mese di settembre, in relazione alle risorse effettivamente assegnate alla scuola – ovvero, secondo la previsione dell'art. 50 della L.35/2012\footcite{Legge_35_2012}, alle reti di scuole -, il Gruppo provvederà ad un adattamento del Piano, sulla base del quale il Dirigente scolastico procederà all'assegnazione definitiva delle risorse, sempre in termini \cit{funzionali}.
		
		A tal punto i singoli \glslink{glhoa}{GLHO} completeranno la redazione del PEI per gli alunni con disabilità di ciascuna classe, tenendo conto di quanto indicato nelle Linee guida del 4 agosto 2009~\footcite{LineGuida2009}.
		\item Inoltre il Gruppo di lavoro per l'inclusione costituisce l'interfaccia della rete dei CTS e
		dei servizi sociali e sanitari territoriali per l'implementazione di azioni di sistema
		(formazione, tutoraggio, progetti di prevenzione, monitoraggio, ecc.).
	\end{itemize}
\end{quote}
In appendice vi sono esempi di regolamento di GLI da inserire nel POF della scuola. 

Per capire cosa è un PAI e a che serve ci vien in aiuto la nota ministeriale n. 1515 del 27 giugno 2013 che riporta:
\begin{quote}
\mancatesto	scopo del Piano annuale per
	l'Inclusività (PAI) è fornire un elemento di riflessione nella predisposizione del POF, di cui il
	PAI è parte integrante. Il PAI, infatti, non va inteso come un ulteriore adempimento
	burocratico, bensì come uno strumento che possa contribuire ad accrescere la consapevolezza
	dell'intera comunità educante sulla centralità e la trasversalità dei processi inclusivi in relazione
	alla qualità dei \cit{risultati} educativi, per creare un contesto educante dove realizzare concretamente
	la scuola \cit{per tutti e per ciascuno}. Esso è prima di tutto un atto interno della scuola autonoma,
	finalizzato all'auto-conoscenza e alla pianificazione, da sviluppare in un processo responsabile e
	attivo di crescita e partecipazione.
	
	In questa ottica di sviluppo e monitoraggio delle capacità inclusive della scuola -– nel rispetto
	delle prerogative dell'autonomia scolastica -- il PAI non va dunque interpretato come un \cit{piano
	formativo per gli alunni con bisogni educativi speciali}, ad integrazione del P.O.F. (in questo caso
	più che di un \cit{piano per l'inclusione} si tratterebbe di un \cit{piano per gli inclusi}). Il PAI non è
	quindi un \cit{documento} per chi ha bisogni educativi speciali, ma è lo strumento per una
	progettazione della propria offerta formativa in senso inclusivo, è lo sfondo ed il fondamento sul
	quale sviluppare una didattica attenta ai bisogni di ciascuno nel realizzare gli obiettivi comuni, le
	linee guida per un concreto impegno programmatico per l'inclusione,\textit{basato su una attenta lettura del grado di inclusività della scuola e su obiettivi di miglioramento, da perseguire nel senso
	della trasversalità delle prassi di inclusione negli ambiti dell'insegnamento curricolare, della
	gestione delle classi, dell'organizzazione dei tempi e degli spazi scolastici, delle relazioni tra
	docenti, alunni e famiglie.}\footcite{Nota_1551_2013}
	
\end{quote}
Su come è strutturato un PAI sono reperibili molti modelli in rete. Un Piano che va per la maggiore è questo reperibile nel sito dell'USR Marche~\footcite{pai13b}.

Il GLI si riunisce:
\begin{quote}
	con una cadenza -- ove possibile -- almeno mensile, nei tempi e nei modi che maggiormente si confanno alla complessità interna della scuola, ossia in orario di servizio ovvero in orari aggiuntivi o funzionali (come previsto dagli art. 28\footnote{Attività di insegnamento} e 29\footnote{Attività funzionali all'insegnamento} del CCNL 2006/2009)~\footcite{ccnl_2006_09}, potendo far rientrare la partecipazione alle attività del gruppo nei compensi già pattuiti per i docenti in sede di contrattazione integrativa di istituto.
\end{quote}
\section{POF}
\begin{quote}
	Nel POF della scuola occorre che trovino esplicitazione:
	\begin{itemize}
		\item un concreto impegno programmatico per l'inclusione, basato su una attenta lettura del
		grado di inclusività della scuola e su obiettivi di miglioramento, da perseguire nel senso
		della trasversalità delle prassi di inclusione negli ambiti dell'insegnamento curricolare,
		della gestione delle classi, dell'organizzazione dei tempi e degli spazi scolastici, delle
		relazioni tra docenti, alunni e famiglie;
		\item criteri e procedure di utilizzo \cit{funzionale} delle risorse professionali presenti,
		privilegiando, rispetto a una logica meramente quantitativa di distribuzione degli
		organici, una logica \cit{qualitativa}, sulla base di un progetto di inclusione condiviso con
		famiglie e servizi sociosanitari che recuperi l'aspetto \cit{pedagogico} del percorso di
		apprendimento e l'ambito specifico di competenza della scuola;
		\item l'impegno a partecipare ad azioni di formazione e/o di prevenzione concordate a livello
		territoriale.~\footcite{cm8_2013}
	\end{itemize}
\end{quote}
Praticamente il POF dovrebbe avere (come suggerisce l'USR  in una sua nota) 
\begin{quote}
	Il Piano Annuale per l'inclusività, richiamato nelle indicazioni ministeriali citate, va innanzi tutto inteso come parte essenziale del Piano dell'Offerta Formativa che ciascuna Istituzione Scolastica è tenuta ad elaborare in base al DPR 8 marzo 1999 n. 275\footcite{DPR_275_1999} (regolamento autonomia delle istituzioni scolastiche). Si Suggerisce quindi di approfondire nei POF le diverse linee di azione che la scuola (intesa in dimensione sia individuale sia collegiale) ha individuato come più efficaci per la presa in carico degli alunni. Le linee di azione, concretamente delineate (cioè con l'indicazione di chi fa che cosa quando e come) consentiranno poi ai docenti, anche di nuovo arrivo, di essere guidati nel momento in cui si presentino problemi particolari e di difficile soluzione. 
	
	Le Istituzioni Scolastiche devono organizzarsi in modo tale da non lasciare gli insegnanti abbandonati a se stessi; una delle condizioni che oggi i docenti maggiormente rappresentano è quello della solitudine, del non aver appoggi, consigli, indicazioni, subito quando servono.
	
	Ad Esempio, è necessario che siano definite, concordate e rese pubbliche le modalità di comportamento che un docente deve seguire se sospetta che un alunno sia maltrattato, o subisca abusi, o sia costretto al lavoro, patisca la fame o il freddo, o non riceva le cure adeguate in caso di malattie importanti, e così via.
	
	Decidere di agire in queste situazioni, e farlo in modo casuale e impulsivo, può comportare il rischio di aggravare le situazioni; gli insegnanti hanno necessità di sapere come e dove trovare, una guida esperta ed autorevole, come ricevere consigli e come individuare le modalità giuste di azione~\footcite{usrER2013}.
\end{quote}
\section[L'inclusività della scuola]{La rilevazione, il monitoraggio e la valutazione del grado di inclusività della scuola} La circolare prevede che la scuola si munisca di un sistema di autovalutazione e propone tre strumenti l'Index per l'inclusione~\footcite{Indexforinclusio} o il progetto Quadis~\footcite{quadis} o utilizzare ICF. 