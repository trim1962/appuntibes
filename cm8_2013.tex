\chapter[CM n. 8 2013]{CIRCOLARE MINISTERIALE n. 8 Roma,\xheadbreak 6 marzo 2013}
\label{cha:cm8_2013}
\begin{center}
Ministero dell'Istruzione, dell'Università e della Ricerca
Dipartimento per l'Istruzione
\end{center}
\begin{tabular*}{\textwidth}%
	{@{\extracolsep{\fill}}lr}
	CIRCOLARE MINISTERIALE n. 8&Roma, 6 marzo 2013\\
	Prot. 561&
\end{tabular*}
\begin{flushright}
Ai Direttori Generali degli Uffici Scolastici Regionali
LORO SEDI

Ai Dirigenti Scolastici

LORO SEDI

Ai Referenti Regionali per la Disabilità / per i DSA

LORO SEDI

Alle Associazioni componenti
l'Osservatorio permanente per l'Integrazione degli alunni con disabilità

LORO SEDI

Alle Associazioni del FONAGS

LORO SEDI

Alle Associazioni del Forum Nazionale degli Studenti

LORO SEDI

Ai Presidenti delle Consulte Provinciali degli Studenti

LORO SEDI
\end{flushright}
Oggetto: Direttiva Ministeriale 27 dicembre 2012 “Strumenti d'intervento per alunni con bisogni educativi speciali e organizzazione territoriale per l'inclusione scolastica”. Indicazioni operative

Il 27 dicembre scorso è stata firmata dall'On.le Ministro l'unita Direttiva recante Strumenti
d'intervento per alunni con bisogni educativi speciali e organizzazione territoriale per l'inclusione
scolastica, che delinea e precisa la strategia inclusiva della scuola italiana al fine di realizzare
appieno il diritto all'apprendimento per tutti gli alunni e gli studenti in situazione di difficoltà. La
Direttiva ridefinisce e completa il tradizionale approccio all'integrazione scolastica, basato sulla
certificazione della disabilità, estendendo il campo di intervento e di responsabilità di tutta la comunità educante all'intera area dei Bisogni Educativi Speciali (BES), comprendente: “svantaggio
sociale e culturale, disturbi specifici di apprendimento e/o disturbi evolutivi specifici, difficoltà
derivanti dalla non conoscenza della cultura e della lingua italiana perché appartenenti a culture
diverse”.
La Direttiva estende pertanto a tutti gli studenti in difficoltà il diritto alla personalizzazione
dell'apprendimento, richiamandosi espressamente ai principi enunciati dalla Legge 53/2003.
Fermo restando l'obbligo di presentazione delle certificazioni per l'esercizio dei diritti
conseguenti alle situazioni di disabilità e di DSA, è compito doveroso dei Consigli di classe o dei
teams dei docenti nelle scuole primarie indicare in quali altri casi sia opportuna e necessaria
l'adozione di una personalizzazione della didattica ed eventualmente di misure compensative o
dispensative, nella prospettiva di una presa in carico globale ed inclusiva di tutti gli alunni.
Strumento privilegiato è il percorso individualizzato e personalizzato, redatto in un Piano
Didattico Personalizzato (PDP), che ha lo scopo di definire, monitorare e documentare – secondo
un'elaborazione collegiale, corresponsabile e partecipata - le strategie di intervento più idonee e i
criteri di valutazione degli apprendimenti.
In questa nuova e più ampia ottica, il Piano Didattico Personalizzato non può più essere inteso
come mera esplicitazione di strumenti compensativi e dispensativi per gli alunni con DSA; esso è
bensì lo strumento in cui si potranno, ad esempio, includere progettazioni didattico-educative
calibrate sui livelli minimi attesi per le competenze in uscita (di cui moltissimi alunni con BES,
privi di qualsivoglia certificazione diagnostica, abbisognano), strumenti programmatici utili in
maggior misura rispetto a compensazioni o dispense, a carattere squisitamente didattico-
strumentale.
La Direttiva ben chiarisce come la presa in carico dei BES debba essere al centro dell'attenzione
e dello sforzo congiunto della scuola e della famiglia.
È necessario che l'attivazione di un percorso individualizzato e personalizzato per un alunno
con Bisogni Educativi Speciali sia deliberata in Consiglio di classe - ovvero, nelle scuole primarie,
da tutti i componenti del team docenti - dando luogo al PDP, firmato dal Dirigente scolastico (o da
un docente da questi specificamente delegato), dai docenti e dalla famiglia. Nel caso in cui sia
necessario trattare dati sensibili per finalità istituzionali, si avrà cura di includere nel PDP apposita
autorizzazione da parte della famiglia.
A titolo esemplificativo, sul sito del MIUR saranno pubblicati alcuni modelli di PDP (Cfr.
\url{http://hubmiur.pubblica.istruzione.it/web/istruzione/dsa}) .
Ove non sia presente certificazione clinica o diagnosi, il Consiglio di classe o il team dei docenti
motiveranno opportunamente, verbalizzandole, le decisioni assunte sulla base di considerazioni
pedagogiche e didattiche; ciò al fine di evitare contenzioso.
\section*{Alunni con DSA e disturbi evolutivi specifici}
Per quanto riguarda gli alunni in possesso di una diagnosi di DSA rilasciata da una struttura
privata, si raccomanda - nelle more del rilascio della certificazione da parte di strutture sanitarie
pubbliche o accreditate – di adottare preventivamente le misure previste dalla Legge 170/2010,
qualora il Consiglio di classe o il team dei docenti della scuola primaria ravvisino e riscontrino, sulla  base di considerazioni psicopedagogiche e didattiche, carenze fondatamente riconducibili al disturbo.
Pervengono infatti numerose segnalazioni relative ad alunni (già sottoposti ad accertamenti
diagnostici nei primi mesi di scuola) che, riuscendo soltanto verso la fine dell'anno scolastico ad
ottenere la certificazione, permangono senza le tutele cui sostanzialmente avrebbero diritto. Si
evidenzia pertanto la necessità di superare e risolvere le difficoltà legate ai tempi di rilascio delle
certificazioni (in molti casi superiori ai sei mesi) adottando comunque un piano didattico
individualizzato e personalizzato nonché tutte le misure che le esigenze educative riscontrate
richiedono. Negli anni terminali di ciascun ciclo scolastico, in ragione degli adempimenti connessi
agli esami di Stato, le certificazioni dovranno essere presentate entro il termine del 31 marzo, come
previsto all'art.1 dell'Accordo sancito in Conferenza Stato-Regioni sulle certificazioni per i DSA
(R.A. n. 140 del 25 luglio 2012).
\section*{Area dello svantaggio socioeconomico, linguistico e culturale}
Si vuole inoltre richiamare ulteriormente l'attenzione su quell'area dei BES che interessa lo
svantaggio socioeconomico, linguistico, culturale. La Direttiva, a tale proposito, ricorda che “ogni
alunno, con continuità o per determinati periodi, può manifestare Bisogni Educativi Speciali: o per
motivi fisici, biologici, fisiologici o anche per motivi psicologici, sociali, rispetto ai quali è
necessario che le scuole offrano adeguata e personalizzata risposta”. Tali tipologie di BES dovranno
essere individuate sulla base di elementi oggettivi (come ad es. una segnalazione degli operatori dei
servizi sociali), ovvero di ben fondate considerazioni psicopedagogiche e didattiche.
Per questi alunni, e in particolare per coloro che sperimentano difficoltà derivanti dalla non
conoscenza della lingua italiana - per esempio alunni di origine straniera di recente immigrazione
e, in specie, coloro che sono entrati nel nostro sistema scolastico nell'ultimo anno - è parimenti
possibile attivare percorsi individualizzati e personalizzati, oltre che adottare strumenti
compensativi e misure dispensative (ad esempio la dispensa dalla lettura ad alta voce e le attività
ove la lettura è valutata, la scrittura veloce sotto dettatura, ecc.), con le stesse modalità sopra
indicate.
In tal caso si avrà cura di monitorare l'efficacia degli interventi affinché siano messi in atto per
il tempo strettamente necessario. Pertanto, a differenza delle situazioni di disturbo documentate da
diagnosi, le misure dispensative, nei casi sopra richiamati, avranno carattere transitorio e attinente
aspetti didattici, privilegiando dunque le strategie educative e didattiche attraverso percorsi
personalizzati, più che strumenti compensativi e misure dispensative.
In ogni caso, non si potrà accedere alla dispensa dalle prove scritte di lingua straniera se non in
presenza di uno specifico disturbo clinicamente diagnosticato, secondo quanto previsto dall'art. 6
del DM n. 5669 del 12 luglio 2011 e dalle allegate Linee guida.
Si rammenta, infine, che, ai sensi dell'articolo 5 del DPR n. 89/2009, le 2 ore di insegnamento
della seconda lingua comunitaria nella scuola secondaria di primo grado possono essere utilizzate
anche per potenziare l'insegnamento della lingua italiana per gli alunni stranieri non in possesso
delle necessarie conoscenze e competenze nella medesima lingua italiana, nel rispetto
dell'autonomia delle istituzioni scolastiche.
Eventuali disposizioni in merito allo svolgimento degli esami di Stato o delle rilevazioni annuali
degli apprendimenti verranno fornite successivamente.
\section*{AZIONI A LIVELLO DI SINGOLA ISTITUZIONE SCOLASTICA}
Per perseguire tale “politica per l'inclusione”, la Direttiva fornisce indicazioni alle istituzioni
scolastiche, che dovrebbero esplicitarsi, a livello di singole scuole, in alcune azioni strategiche di
seguito sintetizzate.
\begin{enumerate}
	\item Fermo restando quanto previsto dall'art. 15 comma 2 della L. 104/92, i compiti del Gruppo di
	lavoro e di studio d'Istituto (GLHI) si estendono alle problematiche relative a tutti i BES. A
	tale scopo i suoi componenti sono integrati da tutte le risorse specifiche e di coordinamento
	presenti nella scuola (funzioni strumentali, insegnanti per il sostegno, AEC, assistenti alla
	comunicazione, docenti “disciplinari” con esperienza e/o formazione specifica o con compiti di
	coordinamento delle classi, genitori ed esperti istituzionali o esterni in regime di
	convenzionamento con la scuola), in modo da assicurare all'interno del corpo docente il
	trasferimento capillare delle azioni di miglioramento intraprese e un'efficace capacità di
	rilevazione e intervento sulle criticità all'interno delle classi.
	
	Tale Gruppo di lavoro assume la denominazione di Gruppo di lavoro per l'inclusione (in sigla
	GLI) e svolge le seguenti funzioni:
	\begin{itemize}
		\item rilevazione dei BES presenti nella scuola;
		\item raccolta e documentazione degli interventi didattico-educativi posti in essere anche in
		funzione di azioni di apprendimento organizzativo in rete tra scuole e/o in rapporto con
		azioni strategiche dell'Amministrazione;
		\item focus/confronto sui casi, consulenza e supporto ai colleghi sulle strategie/metodologie di
		gestione delle classi;
		\item rilevazione, monitoraggio e valutazione del livello di inclusività della scuola;
		\item raccolta e coordinamento delle proposte formulate dai singoli GLH Operativi sulla base
		delle effettive esigenze, ai sensi dell'art. 1, c. 605, lettera b, della legge 296/2006, tradotte
		in sede di definizione del PEI come stabilito dall'art. 10 comma 5 della Legge 30 luglio
		2010 n. 122 ;
		\item elaborazione di una proposta di Piano Annuale per l'Inclusività riferito a tutti gli
		alunni con BES, da redigere al termine di ogni anno scolastico (entro il mese di Giugno).
		
		A tale scopo, il Gruppo procederà ad un'analisi delle criticità e dei punti di forza degli
		interventi di inclusione scolastica operati nell'anno appena trascorso e formulerà
		un'ipotesi globale di utilizzo funzionale delle risorse specifiche, istituzionali e non, per
		incrementare il livello di inclusività generale della scuola nell'anno successivo. Il Piano
		sarà quindi discusso e deliberato in Collegio dei Docenti e inviato ai competenti Uffici
		degli UUSSRR, nonché ai GLIP e al GLIR, per la richiesta di organico di sostegno, e
		alle altre istituzioni territoriali come proposta di assegnazione delle risorse di
		competenza, considerando anche gli Accordi di Programma in vigore o altre specifiche
		intese sull'integrazione scolastica sottoscritte con gli Enti Locali. A seguito di ciò, gli
		Uffici Scolastici regionali assegnano alle singole scuole globalmente le risorse di
		sostegno secondo quanto stabilito dall’ art 19 comma 11 della Legge n. 111/2011.
		
		Nel mese di settembre, in relazione alle risorse effettivamente assegnate alla scuola –
		ovvero, secondo la previsione dell'art. 50 della L.35/2012, alle reti di scuole -, il Gruppo provvederà ad un adattamento del Piano, sulla base del quale il Dirigente scolastico
		procederà all'assegnazione definitiva delle risorse, sempre in termini “funzionali”.
		
		A tal punto i singoli GLHO completeranno la redazione del PEI per gli alunni con
		disabilità di ciascuna classe, tenendo conto di quanto indicato nelle Linee guida del 4
		agosto 2009.
		\item Inoltre il Gruppo di lavoro per l'inclusione costituisce l'interfaccia della rete dei CTS e
		dei servizi sociali e sanitari territoriali per l'implementazione di azioni di sistema
		(formazione, tutoraggio, progetti di prevenzione, monitoraggio, ecc.).
	\end{itemize}
	Dal punto di vista organizzativo, pur nel rispetto delle autonome scelte delle scuole, si suggerisce
	che il gruppo svolga la propria attività riunendosi (per quanto riguarda le risorse specifiche
	presenti: insegnanti per il sostegno, AEC, assistenti alla comunicazione, funzioni strumentali,
	ecc.), con una cadenza - ove possibile - almeno mensile, nei tempi e nei modi che maggiormente
	si confanno alla complessità interna della scuola, ossia in orario di servizio ovvero in orari
	aggiuntivi o funzionali (come previsto dagli artt. 28 e 29 del CCNL 2006/2009), potendo far
	rientrare la partecipazione alle attività del gruppo nei compensi già pattuiti per i docenti in sede di
	contrattazione integrativa di istituto. Il Gruppo, coordinato dal Dirigente scolastico o da un suo
	delegato, potrà avvalersi della consulenza e/o supervisione di esperti esterni o interni, anche
	attraverso accordi con soggetti istituzionali o del privato sociale e, a seconda delle necessità (ad
	esempio, in caso di istituto comprensivo od onnicomprensivo), articolarsi anche per gradi
	scolastici.
	All'inizio di ogni anno scolastico il Gruppo propone al Collegio dei Docenti una
	programmazione degli obiettivi da perseguire e delle attività da porre in essere, che confluisce
	nel Piano annuale per l'Inclusività; al termine dell'anno scolastico, il Collegio procede alla
	verifica dei risultati raggiunti.
	\item Nel P.O.F. della scuola occorre che trovino esplicitazione:
	\begin{itemize}
		\item un concreto impegno programmatico per l'inclusione, basato su una attenta lettura del
		grado di inclusività della scuola e su obiettivi di miglioramento, da perseguire nel senso
		della trasversalità delle prassi di inclusione negli ambiti dell'insegnamento curricolare,
		della gestione delle classi, dell'organizzazione dei tempi e degli spazi scolastici, delle
		relazioni tra docenti, alunni e famiglie;
		\item criteri e procedure di utilizzo “funzionale” delle risorse professionali presenti,
		privilegiando, rispetto a una logica meramente quantitativa di distribuzione degli
		organici, una logica “qualitativa”, sulla base di un progetto di inclusione condiviso con
		famiglie e servizi sociosanitari che recuperi l'aspetto “pedagogico” del percorso di
		apprendimento e l'ambito specifico di competenza della scuola;
		\item l'impegno a partecipare ad azioni di formazione e/o di prevenzione concordate a livello
		territoriale.
	\end{itemize}
	\item  La rilevazione, il monitoraggio e la valutazione del grado di inclusività della scuola sono
	finalizzate ad accrescere la consapevolezza dell'intera comunità educante sulla centralità e la
	trasversalità dei processi inclusivi in relazione alla qualità dei “risultati” educativi. Da tali azioni
	si potranno inoltre desumere indicatori realistici sui quali fondare piani di miglioramento
	organizzativo e culturale. A tal fine possono essere adottati sia strumenti strutturati reperibili in
	rete [come l’”Index per l'inclusione” o il progetto “Quadis” (\url{http://www.quadis.it/jm/})], sia
	concordati a livello territoriale. Ci si potrà inoltre avvalere dell'approccio fondato sul modello
	ICF dell'OMS e dei relativi concetti di barriere e facilitatori.
\end{enumerate}
\section*{AZIONI A LIVELLO TERRITORIALE}
La direttiva affida un ruolo fondamentale ai CTS - Centri Territoriali di Supporto, quale
interfaccia fra l'Amministrazione e le scuole, e tra le scuole stesse nonché quale rete di supporto al
processo di integrazione, allo sviluppo professionale dei docenti e alla diffusione delle migliori
pratiche.

Le scuole dovranno poi impegnarsi a perseguire, anche attraverso le reti scolastiche, accordi e
intese con i servizi sociosanitari territoriali (ASL, Servizi sociali e scolastici comunali e provinciali,
enti del privato sociale e del volontariato, Prefetture, ecc.) finalizzati all'integrazione dei servizi “alla
persona” in ambito scolastico, con funzione preventiva e sussidiaria, in ottemperanza a quanto
previsto dalla Legge 328/2000. Tali accordi dovranno prevedere l'esplicitazione di procedure
condivise di accesso ai diversi servizi in relazione agli alunni con BES presenti nella scuola.

Si precisa inoltre che, fermi restando compiti e composizione dei GLIP di cui all'art. 15 commi 1,
3 e 4 della L. 104/92, le loro funzioni si estendono anche a tutti i BES, stante l'indicazione contenuta
nella stessa L. 104/92 secondo cui essi debbono occuparsi dell'integrazione scolastica degli alunni
con disabilità, “nonché per qualsiasi altra attività inerente all'integrazione degli alunni in difficoltà di
apprendimento.”

In ogni caso, i CTS dovranno strettamente collaborare con i GLIP ovvero con i GLIR, la cui
costituzione viene raccomandata nelle Linee guida del 4 agosto 2009.
\section*{CTI - Centri Territoriali per l'Inclusione}
Il ruolo dei nuovi CTI (Centri Territoriali per l'Inclusione), che potranno essere individuati a
livello di rete territoriale - e che dovranno collegarsi o assorbire i preesistenti Centri Territoriali per
l'integrazione Scolastica degli alunni con disabilità, i Centri di Documentazione per l'integrazione
scolastica degli alunni con disabilità (CDH) ed i Centri Territoriali di Risorse per l'integrazione
scolastica degli alunni con disabilità (CTRH) - risulta strategico anche per creare i presupposti per
l'attuazione dell'art. 50 del DL 9.2.2012, n.5, così come modificato dalla Legge 4.4.2012, n 35, là
dove si prevede (comma b) la “definizione, per ciascuna istituzione scolastica, di un organico
dell'autonomia, funzionale all'ordinaria attività didattica, educativa, amministrativa, tecnica e
ausiliaria, alle esigenze di sviluppo delle eccellenze, di recupero, di integrazione e sostegno agli
alunni con bisogni educativi speciali e di programmazione dei fabbisogni di personale scolastico,
anche ai fini di una estensione del tempo scuola” e ancora (comma c) la “costituzione […] di reti
territoriali tra istituzioni scolastiche, al fine di conseguire la gestione ottimale delle risorse umane,
strumentali e finanziarie” e ancora (comma d) la “definizione di un organico di rete per le finalità
di cui alla lettera c) nonché per l'integrazione degli alunni con bisogni educativi speciali, la
formazione permanente, la prevenzione dell'abbandono e il contrasto dell'insuccesso scolastico e
formativo e dei fenomeni di bullismo, specialmente per le aree di massima corrispondenza tra
povertà e dispersione scolastica” e infine (comma e) la “costituzione degli organici di cui alle
lettere b) e d) […] sulla base dei posti corrispondenti a fabbisogni con carattere di stabilità per
almeno un triennio sulla singola scuola, sulle reti di scuole e sugli ambiti provinciali, anche per i
posti di sostegno, fatte salve le esigenze che ne determinano la rimodulazione annuale.”

Laddove, per ragioni legate alla complessità territoriale, i CTI non potessero essere istituiti o
risultassero poco funzionali, le singole scuole cureranno, attraverso il Gruppo di Lavoro per
l'Inclusione, il contatto con i CTS di riferimento.

Si precisa che il gruppo di docenti operatori del CTS o anche del CTI dovrà essere in possesso
di specifiche competenze, al fine di poter supportare concretamente le scuole e i colleghi con
interventi di consulenza e di formazione mirata. È quindi richiesta una “specializzazione” – nel
senso di una approfondita competenza – nelle tematiche relative ai BES. Per quanto riguarda l'area
della disabilità, si tratterà in primis di docenti specializzati nelle attività di sostegno, ma anche di
docenti curricolari esperti nelle nuove tecnologie per l'inclusione. Per l'area dei disturbi evolutivi
specifici, potranno essere individuati docenti che abbiano frequentato master e/o corsi di
perfezionamento in “Didattica e psicopedagogia per i DSA”, ovvero che abbiano maturato
documentata e comprovata esperienza nel campo, a partire da incarichi assunti nel progetto NTD
(Nuove Tecnologie e Disabilità) attivato sin dal 2006. Anche in questo secondo caso è auspicabile
che il docente operatore dei CTS o dei CTI sia in possesso di adeguate competenze nel campo delle
nuove tecnologie, che potranno essere impiegate anche in progetti per il recupero dello svantaggio
linguistico e culturale ivi compresa l'attivazione di percorsi mirati.

Le istituzioni scolastiche che volessero istituire un CTI possono presentare la propria
candidatura direttamente all'Ufficio Scolastico regionale competente per territorio.

Nel rinviare all'unita Direttiva per una riflessione da portare anche all'interno del Collegio dei
Docenti o loro articolazioni, si invitano le SS.LL. a dare la massima diffusione alla presente
Circolare che viene pubblicata sul sito Internet del Ministero e sulla rete Intranet.

Confidando nella sensibilità e nell'attenzione degli uffici dell'Amministrazione e di tutti coloro
cui la presente circolare è indirizzata, si ringrazia per la collaborazione.

\begin{tabular*}{\textwidth}%
	{@{\extracolsep{\fill}}lc}
	&IL CAPO DIPARTIMENTO\\
	&f.to Lucrezia Stellacci
\end{tabular*}



