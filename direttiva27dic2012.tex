\chapter{Direttiva ministeriale 27 dicembre 2012}
\label{cha:Direttivaministeriale27dicembre2012}
\begin{center}
Il Ministro dell'Istruzione, dell'Università e della Ricerca

STRUMENTI D'INTERVENTO PER ALUNNI CON BISOGNI EDUCATIVI SPECIALI
E ORGANIZZAZIONE TERRITORIALE PER L'INCLUSIONE SCOLASTICA
\end{center}
\section*{Premessa}
I principi che sono alla base del nostro modello di integrazione scolastica - assunto a punto di riferimento per
le politiche di inclusione in Europa e non solo - hanno contribuito a fare del sistema di istruzione italiano un
luogo di conoscenza, sviluppo e socializzazione per tutti, sottolineandone gli aspetti inclusivi piuttosto che
quelli selettivi.
Forte di questa esperienza, il nostro Paese è ora in grado, passati più di trentanni dalla legge n.517 del 1977,
che diede avvio all'integrazione scolastica, di considerare le criticità emerse e di valutare, con maggiore
cognizione, la necessità di ripensare alcuni aspetti dell'intero sistema.
Gli alunni con disabilità si trovano inseriti all'interno di un contesto sempre più variegato, dove la
discriminante tradizionale - alunni con disabilità / alunni senza disabilità - non rispecchia pienamente la
complessa realtà delle nostre classi. Anzi, è opportuno assumere un approccio decisamente educativo, per il
quale l'identificazione degli alunni con disabilità non avviene sulla base della eventuale certificazione, che
certamente mantiene utilità per una serie di benefici e di garanzie, ma allo stesso tempo rischia di chiuderli in
una cornice ristretta. A questo riguardo è rilevante l'apporto, anche sul piano culturale, del modello
diagnostico ICF (International Classification of Functioning) dell'OMS, che considera la persona nella sua
totalità, in una prospettiva bio-psico-sociale. Fondandosi sul profilo di funzionamento e sull'analisi del
contesto, il modello ICF consente di individuare i Bisogni Educativi Speciali (BES) dell'alunno
prescindendo da preclusive tipizzazioni.
In questo senso, ogni alunno, con continuità o per determinati periodi, può manifestare Bisogni Educativi
Speciali: o per motivi fisici, biologici, fisiologici o anche per motivi psicologici, sociali, rispetto ai quali è
necessario che le scuole offrano adeguata e personalizzata risposta.
Va quindi potenziata la cultura dell'inclusione, e ciò anche mediante un approfondimento delle relative
competenze degli insegnanti curricolari, finalizzata ad una più stretta interazione tra tutte le componenti della
comunità educante.
In tale ottica, assumono un valore strategico i Centri Territoriali di Supporto, che rappresentano l'interfaccia
fra l'Amministrazione e le scuole e tra le scuole stesse in relazione ai Bisogni Educativi Speciali. Essi
pertanto integrano le proprie funzioni - come già chiarito dal D.M. 12 luglio 2011 per quanto concerne i
disturbi specifici di apprendimento - e collaborano con le altre risorse territoriali nella definizione di una rete
di supporto al processo di integrazione, con particolare riferimento, secondo la loro originaria vocazione, al
potenziamento del contesto scolastico mediante le nuove tecnologie, ma anche offrendo un ausilio ai docenti
secondo un modello cooperativo di intervento.
Considerato, pertanto, il ruolo che nel nuovo modello organizzativo dell'integrazione è dato ai Centri
Territoriali di Supporto, la presente direttiva definisce nella seconda parte le modalità di organizzazione degli
stessi, le loro funzioni, nonché la composizione del personale che vi opera.
Nella prima parte sono fornite indicazioni alle scuole per la presa in carico di alunni e studenti con Bisogni
Educativi Speciali.
\section*{1.1 Bisogni Educativi Speciali (BES)}
L'area dello svantaggio scolastico è molto più ampia di quella riferibile esplicitamente alla presenza di
deficit. In ogni classe ci sono alunni che presentano una richiesta di speciale attenzione per una varietà di
ragioni: svantaggio sociale e culturale, disturbi specifici di apprendimento e/o disturbi evolutivi specifici,
difficoltà derivanti dalla non conoscenza della cultura e della lingua italiana perché appartenenti a culture
diverse. Nel variegato panorama delle nostre scuole la complessità delle classi diviene sempre più evidente.
Quest'area dello svantaggio scolastico, che ricomprende problematiche diverse, viene indicata come area dei
Bisogni Educativi Speciali (in altri paesi europei: Special Educational Needs). Vi sono comprese tre grandi
sotto-categorie: quella della disabilità; quella dei disturbi evolutivi specifici e quella dello svantaggio socio-
economico, linguistico, culturale.
Per “disturbi evolutivi specifici” intendiamo, oltre i disturbi specifici dell'apprendimento, anche i deficit del
linguaggio, delle abilità non verbali, della coordinazione motoria, ricomprendendo – per la comune origine
nell'età evolutiva – anche quelli dell'attenzione e dell'iperattività, mentre il funzionamento intellettivo limite
può essere considerato un caso di confine fra la disabilità e il disturbo specifico. Per molti di questi profili i
relativi codici nosografici sono ricompresi nelle stesse categorie dei principali Manuali Diagnostici e, in
particolare, del manuale diagnostico ICD-10, che include la classificazione internazionale delle malattie e dei
problemi correlati, stilata dall'Organizzazione mondiale della sanità (OMS) e utilizzata dai Servizi Sociosanitari
pubblici italiani.
Tutte queste differenti problematiche, ricomprese nei disturbi evolutivi specifici, non vengono o possono non
venir certificate ai sensi della legge 104/92, non dando conseguentemente diritto alle provvidenze ed alle
misure previste dalla stessa legge quadro, e tra queste, all'insegnante per il sostegno.
La legge 170/2010, a tal punto, rappresenta un punto di svolta poiché apre un diverso canale di cura
educativa, concretizzando i principi di personalizzazione dei percorsi di studio enunciati nella legge 53/2003,
nella prospettiva della “presa in carico” dell'alunno con BES da parte di ciascun docente curricolare e di
tutto il team di docenti coinvolto, non solo dall'insegnante per il sostegno.
\section*{1.2 Alunni con disturbi specifici}
Gli alunni con competenze intellettive nella norma o anche elevate, che – per specifici problemi - possono
incontrare difficoltà a Scuola, devono essere aiutati a realizzare pienamente le loro potenzialità. Fra essi,
alunni e studenti con DSA (Disturbo Specifico dell'Apprendimento) sono stati oggetto di importanti
interventi normativi, che hanno ormai definito un quadro ben strutturato di norme tese ad assicurare il loro
diritto allo studio.
Tuttavia, è bene precisare che alcune tipologie di disturbi, non esplicitati nella legge 170/2010, danno diritto
ad usufruire delle stesse misure ivi previste in quanto presentano problematiche specifiche in presenza di
competenze intellettive nella norma. Si tratta, in particolare, dei disturbi con specifiche problematiche
nell'area del linguaggio (disturbi specifici del linguaggio o – più in generale- presenza di bassa intelligenza
verbale associata ad alta intelligenza non verbale) o, al contrario, nelle aree non verbali (come nel caso del
disturbo della coordinazione motoria, della disprassia, del disturbo non-verbale o – più in generale - di bassa
intelligenza non verbale associata ad alta intelligenza verbale, qualora però queste condizioni compromettano
sostanzialmente la realizzazione delle potenzialità dell'alunno) o di altre problematiche severe che possono
compromettere il percorso scolastico (come per es. un disturbo dello spettro autistico lieve, qualora non
rientri nelle casistiche previste dalla legge 104).
Un approccio educativo, non meramente clinico – secondo quanto si è accennato in premessa – dovrebbe dar
modo di individuare strategie e metodologie di intervento correlate alle esigenze educative speciali, nella
prospettiva di una scuola sempre più inclusiva e accogliente, senza bisogno di ulteriori precisazioni di
carattere normativo.
Al riguardo, la legge 53/2003 e la legge 170/2010 costituiscono norme primarie di riferimento cui ispirarsi
per le iniziative da intraprendere con questi casi.
\section*{1.3 Alunni con deficit da disturbo dell'attenzione e dell'iperattività}
Un discorso particolare si deve fare a proposito di alunni e studenti con problemi di controllo attentivo e/o
dell'attività, spesso definiti con l'acronimo A.D.H.D. (Attention Deficit Hyperactivity Disorder),
corrispondente all'acronimo che si usava per l'Italiano di D.D.A.I. –- Deficit da disturbo dell'attenzione e
dell'iperattività.
L'ADHD si può riscontrare anche spesso associato ad un DSA o ad altre problematiche, ha una causa
neurobiologica e genera difficoltà di pianificazione, di apprendimento e di socializzazione con i coetanei. Si
è stimato che il disturbo, in forma grave tale da compromettere il percorso scolastico, è presente in circa l'1\%
della popolazione scolastica, cioè quasi 80.000 alunni (fonte I.S.S),
Con notevole frequenza l'ADHD è in comorbilità con uno o più disturbi dell'età evolutiva: disturbo
oppositivo provocatorio; disturbo della condotta in adolescenza; disturbi specifici dell'apprendimento;
disturbi d'ansia; disturbi dell'umore, etc.
Il percorso migliore per la presa in carico del bambino/ragazzo con ADHD si attua senz'altro quando è
presente una sinergia fra famiglia, scuola e clinica. Le informazioni fornite dagli insegnanti hanno una parte
importante per il completamento della diagnosi e la collaborazione della scuola è un anello fondamentale nel
processo riabilitativo.
In alcuni casi il quadro clinico particolarmente grave – anche per la comorbilità con altre patologie - richiede
l'assegnazione dell'insegnante di sostegno, come previsto dalla legge 104/92. Tuttavia, vi sono moltissimi
ragazzi con ADHD che, in ragione della minor gravità del disturbo, non ottengono la certificazione di
disabilità, ma hanno pari diritto a veder tutelato il loro successo formativo.
Vi è quindi la necessità di estendere a tutti gli alunni con bisogni educativi speciali le misure previste dalla
Legge 170 per alunni e studenti con disturbi specifici di apprendimento.
\section*{1.4 Funzionamento cognitivo limite}
Anche gli alunni con potenziali intellettivi non ottimali, descritti generalmente con le espressioni di
funzionamento cognitivo (intellettivo) limite (o borderline), ma anche con altre espressioni (per es. disturbo
evolutivo specifico misto, codice F83) e specifiche differenziazioni - qualora non rientrino nelle previsioni
delle leggi 104 o 170 - richiedono particolare considerazione. Si può stimare che questi casi si aggirino
intorno al 2,5\% dell'intera popolazione scolastica, cioè circa 200.000 alunni.
Si tratta di bambini o ragazzi il cui QI globale (quoziente intellettivo) risponde a una misura che va dai 70
agli 85 punti e non presenta elementi di specificità. Per alcuni di loro il ritardo è legato a fattori
neurobiologici ed è frequentemente in comorbilità con altri disturbi. Per altri, si tratta soltanto di una forma
lieve di difficoltà tale per cui, se adeguatamente sostenuti e indirizzati verso i percorsi scolastici più consoni
alle loro caratteristiche, gli interessati potranno avere una vita normale. Gli interventi educativi e didattici
hanno come sempre ed anche in questi casi un'importanza fondamentale.
\section*{1.5 Adozione di strategie di intervento per i BES}
Dalle considerazioni sopra esposte si evidenzia, in particolare, la necessità di elaborare un percorso
individualizzato e personalizzato per alunni e studenti con bisogni educativi speciali, anche attraverso la
redazione di un Piano Didattico Personalizzato, individuale o anche riferito a tutti i bambini della classe con
BES, ma articolato, che serva come strumento di lavoro in itinere per gli insegnanti ed abbia la funzione di
documentare alle famiglie le strategie di intervento programmate.
Le scuole – con determinazioni assunte dai Consigli di classe, risultanti dall'esame della documentazione
clinica presentata dalle famiglie e sulla base di considerazioni di carattere psicopedagogico e didattico –
possono avvalersi per tutti gli alunni con bisogni educativi speciali degli strumenti compensativi e delle
misure dispensative previste dalle disposizioni attuative della Legge 170/2010 (DM 5669/2011), meglio
descritte nelle allegate Linee guida.
\section*{1.6 Formazione}
Si è detto che vi è una sempre maggiore complessità nelle nostre classi, dove si intrecciano i temi della
disabilità, dei disturbi evolutivi specifici, con le problematiche del disagio sociale e dell'inclusione degli
alunni stranieri. Per questo è sempre più urgente adottare una didattica che sia 'denominatore comune' per
tutti gli alunni e che non lasci indietro nessuno: una didattica inclusiva più che una didattica speciale.
Al fine di corrispondere alle esigenze formative che emergono dai nuovi contesti della scuola italiana, alle
richieste di approfondimento e accrescimento delle competenze degli stessi docenti e dirigenti scolastici, il
MIUR ha sottoscritto un accordo quadro con le Università presso le quali sono attivati corsi di scienze della
formazione finalizzato all'attivazione di corsi di perfezionamento professionale e/o master rivolti al
personale della scuola.
A partire dall'anno accademico 2011/2012 sono stati attivati 35 corsi/master in “Didattica e psicopedagogia
dei disturbi specifici di apprendimento” in tutto il territorio nazionale.
A seguito dei positivi riscontri relativi alla suddetta azione, la Direzione generale per lo Studente,
l'Integrazione, la Partecipazione e la Comunicazione d'intesa con la Direzione Generale per il Personale
scolastico – con la quale ha sottoscritto un'apposita convenzione con alcune università italiane mirata alla
costituzione di una rete delle facoltà/dipartimenti di scienze della formazione – ha predisposto una ulteriore
offerta formativa che si attiverà sin dal corrente anno scolastico su alcune specifiche tematiche emergenti in
tema di disabilità, con corsi/master dedicati alla didattica e psicopedagogia per l'autismo, l'ADHD, le
disabilità intellettive e i funzionamenti intellettivi limite, l'educazione psicomotoria inclusiva e le disabilità
sensoriali.
L'attivazione dei percorsi di alta formazione dovrà contemperare l'esigenza di rispondere al fabbisogno
rilevato ed a requisiti di carattere tecnico-scientifico da parte delle università che si renderanno disponibili a
tenere i corsi.
\section*{2. Organizzazione territoriale per l'ottimale realizzazione dell'inclusione scolastica}
\subsection*{2.1.1 I CTS - Centri Territoriali di Supporto: distribuzione sul territorio}
I Centri Territoriali di Supporto (CTS) sono stati istituiti dagli Uffici Scolastici Regionali in accordo con il
MIUR mediante il Progetto “Nuove Tecnologie e Disabilità”. I Centri sono collocati presso scuole polo e la
loro sede coincide con quella dell'istituzione scolastica che li accoglie.
È pertanto facoltà degli Uffici Scolastici Regionali integrare o riorganizzare la rete regionale dei CTS, secondo
eventuali nuove necessità emerse in ordine alla qualità e alla distribuzione del servizio.
Si ritiene, a questo riguardo, opportuna la presenza di un CTS almeno su un territorio corrispondente ad ogni
provincia della Regione, fatte salve le aree metropolitane che, per densità di popolazione, possono
necessitare di uno o più CTS dedicati.
Un'equa distribuzione sul territorio facilita il fatto che i CTS divengano punti di riferimento per le scuole e
coordinino le proprie attività con Province, Comuni, Municipi, Servizi Sanitari, Associazioni delle persone
con disabilità e dei loro familiari, Centri di ricerca, di formazione e di documentazione, anche istituiti dalle
predette associazioni, nel rispetto di strategie generali eventualmente definite a livello di Ufficio Scolastico
Regionale e di Ministero centrale. Il coordinamento con il territorio assicura infatti ai CTS una migliore
efficienza ed efficacia nella gestione delle risorse disponibili e aumenta la capacità complessiva del sistema
di offrire servizi adeguati. Sarà cura degli Uffici Scolastici Regionali operare il raccordo tra i CTS e i GLIR,
oltre che raccordare i GLIP con i nuovi organismi previsti nella presente Direttiva.
Ad un livello territoriale meno esteso, che può coincidere ad esempio con il distretto socio-sanitario, è
risultato utile individuare altre scuole polo facenti parte di una rete per l'inclusione scolastica.
Tale esperienza è stata già sperimentata con successo in alcune regioni in cui ai CTS, di livello provinciale,
sono stati affiancati i CTI-Centri Territoriali per l'Inclusione, di livello distrettuale.
La creazione di una rete diffusa e ben strutturata tra tutte le scuole ed omogenea nella sua articolazione rende
concreta la possibilità per i docenti di avere punti di contatto e di riferimento per tutte le problematiche
inerenti i Bisogni Educativi Speciali.
A livello di singole scuole, è auspicabile una riflessione interna che, tenendo conto delle risorse presenti,
individui possibili modelli di relazione con la rete dei CTS e dei CTI, al fine di assicurare la massima
ricaduta possibile delle azioni di consulenza, formazione, monitoraggio e raccolta di buone pratiche,
perseguendo l'obiettivo di un sempre maggior coinvolgimento degli insegnanti curricolari, attraverso – ad
esempio – la costituzione di gruppi di lavoro per l'inclusione scolastica. Occorre in buona sostanza
pervenire ad un reale coinvolgimento dei Collegi dei Docenti e dei Consigli di Istituto che porti all'adozione di una politica (nel senso di “policy”) interna delle scuole per l'inclusione, che assuma una
reale trasversalità e centralità rispetto al complesso dell'offerta formativa.
L'organizzazione territoriale per l'inclusione prevede quindi:
\begin{itemize}
	\item i GLH a livello di singola scuola, eventualmente affiancati da Gruppi di lavoro per l'Inclusione; i
	GLH di rete o distrettuali
	\item i Centri Territoriali per l'Inclusione (CTI) a livello di distretto sociosanitario e
	\item almeno un CTS a livello provinciale.
\end{itemize}
Al fine di consentire un'adeguata comunicazione, a livello regionale, delle funzioni, delle attività e della
collocazione geografica dei CTS, ogni Centro o rete di Centri predispone e aggiorna un proprio sito web, il
cui link sarà selezionabile anche dal portale dell'Ufficio Scolastico Regionale. Tali link sono inseriti nel
Portale MIUR dei Centri Territoriali di Supporto: www.istruzione.cts.it
Sul sito dei CTS si possono prevedere pagine web per ciascun CTI ed eventualmente uno spazio per i GLH
di rete per favorire lo scambio aggiornato e la conoscenza delle attività del territorio.

\subsection*{2.1.2 L'Equipe di docenti specializzati (docenti curricolari e di sostegno)}
Ferme restando la formazione e le competenze di carattere generale in merito all'inclusione, tanto dei docenti
per le attività di sostegno quanto per i docenti curricolari, possono essere necessari interventi di esperti che
offrano soluzioni rapide e concrete per determinate problematiche funzionali. Si fa riferimento anzitutto a
risorse interne ossia a docenti che nell'ambito della propria esperienza professionale e dei propri studi
abbiano maturato competenze su tematiche specifiche della disabilità o dei disturbi evolutivi specifici.
Possono pertanto fare capo ai CTS equipe di docenti specializzati - sia curricolari sia per il sostegno - che
offrono alle scuole, in ambito provinciale, supporto e consulenza specifica sulla didattica dell'inclusione. La
presenza di docenti curricolari nell'equipe, così come nei GLH di istituto e di rete costituisce un elemento
importante nell'ottica di una vera inclusione scolastica.
Può essere preso ad esempio di tale modello lo Sportello Provinciale Autismo attivato in alcuni CTS, che, in
collaborazione con l'Ufficio Scolastico Regionale, con i Centri Territoriali per l'Integrazione e le
Associazioni delle persone con disabilità e dei loro familiari, valorizzando la professionalità di un gruppo di
insegnanti esperti e formati, offre ai docenti di quella provincia una serie di servizi di consulenza – da
realizzarsi anche presso la scuola richiedente - per garantire l'efficacia dell'integrazione scolastica degli
alunni e degli studenti con autismo.
\section*{2.2. Funzioni dei Centri Territoriali di Supporto}
L'effettiva capacità delle nuove tecnologie di raggiungere obiettivi di miglioramento nel processo di
apprendimento – insegnamento, sviluppo e socializzazione dipende da una serie di fattori strategici che
costituiscono alcune funzioni basilari dei Centri Territoriali di Supporto.
\subsection*{2.2.1 Informazione e formazione}
I CTS informano i docenti, gli alunni, gli studenti e i loro genitori delle risorse tecnologiche disponibili, sia
gratuite sia commerciali. Per tale scopo, organizzano incontri di presentazione di nuovi ausili, ne danno
notizia sul sito web oppure direttamente agli insegnanti o alle famiglie che manifestino interesse alle novità
in materia.
I CTS organizzano iniziative di formazione sui temi dell'inclusione scolastica e sui BES, nonché nell'ambito
delle tecnologie per l'integrazione, rivolte al personale scolastico, agli alunni o alle loro famiglie, nei modi e
nei tempi che ritengano opportuni.
Al fine di una maggiore efficienza della spesa, i CTS organizzano le iniziative di formazione anche in rete
con altri Centri Territoriali di Supporto, in collaborazione con altri organismi.
I CTS valutano e propongono ai propri utenti soluzioni di software freeware a partire da quelli realizzati
mediante l'Azione 6 del Progetto “Nuove Tecnologie e Disabilità”
\subsection*{2.2.2 Consulenza}
Oltre ad una formazione generale sull'uso delle tecnologie per l'integrazione rivolta agli insegnanti, è
necessario, per realizzare a pieno le potenzialità offerte dalle tecnologie stesse, il contributo di un esperto che
individui quale sia l'ausilio più appropriato da acquisire, soprattutto per le situazioni più complesse. I CTS
offrono pertanto consulenza in tale ambito, coadiuvando le scuole nella scelta dell'ausilio e accompagnando
gli insegnanti nell'acquisizione di competenze o pratiche didattiche che ne rendano efficace l'uso.
La consulenza offerta dai Centri non riguarda solo l'individuazione dell'ausilio più appropriato per l'alunno,
ma anche le modalità didattiche da attuare per inserire il percorso di apprendimento dello studente che
utilizza le tecnologie per l'integrazione nel più ampio ambito delle attività di classe e le modalità di
collaborazione con la famiglia per facilitare le attività di studio a casa.
La consulenza si estende gradualmente a tutto l'ambito della disabilità e dei disturbi evolutivi specifici, non
soltanto alle tematiche connesse all'uso delle nuove tecnologie.
\subsection*{2.2.3 Gestione degli ausili e comodato d'uso}
I CTS acquistano ausili adeguati alle esigenze territoriali per svolgere le azioni previste nei punti 2.1. e 2.2 e
per avviare il servizio di comodato d'uso dietro presentazione di un progetto da parte delle scuole. Grazie
alla loro dotazione, possono consentire, prima dell'acquisto definitivo da parte della scuola o della richiesta
dell'ausilio al CTS, di provare e di verificare l'efficacia, per un determinato alunno, dell'ausilio stesso.
Nel caso del comodato d'uso di ausilio di proprietà del CTS, questo deve seguire l'alunno anche se cambia
scuola nell'ambito della stessa provincia, soprattutto nel passaggio di ciclo. In alcune province, in accordo con
gli Uffici Scolastici Regionali, alcuni CTS gestiscono l'acquisto degli ausili e la loro distribuzione agli alunni
sul territorio di riferimento, anche assegnandoli in comodato d'uso.
I CTS possono definire accordi con le Ausilioteche e/o Centri Ausili presenti sul territorio al fine di una
condivisa gestione degli ausili in questione, sulla base dell'Accordo quadro con la rete nazionale dei centri di
consulenza sugli ausili.
\subsection*{2.2.4 Buone pratiche e attività di ricerca e sperimentazione}
I CTS raccolgono le buone pratiche di inclusione realizzate dalle istituzioni scolastiche e, opportunamente
documentate, le condividono con le scuole del territorio di riferimento, sia mediante l'attività di
informazione, anche attraverso il sito internet, sia nella fase di formazione o consulenza. Promuovono inoltre
ogni iniziativa atta a stimolare la realizzazione di buone pratiche nelle scuole di riferimento, curandone la
validazione e la successiva diffusione.
I CTS sono inoltre Centri di attività di ricerca didattica e di sperimentazione di nuovi ausili, hardware o
software, da realizzare anche mediante la collaborazione con altre scuole o CTS, Università e Centri di
Ricerca e, in particolare, con l'ITD-CNR di Genova, sulla base di apposita convenzione.
\subsection*{2.2.5 Piano annuale di intervento}
Per ogni anno scolastico, i CTS, autonomamente o in rete, definiscono il piano annuale di intervento relativo
ad acquisti e iniziative di formazione. Nel piano, quindi, sono indicati gli acquisti degli ausili necessari, nei
limiti delle risorse disponibili e a ciò destinate, su richiesta della scuola e assegnati tramite comodato d'uso.
È opportuno che l'ausilio da acquistare sia individuato da un esperto operatore del CTS, con l'eventuale
supporto – se necessario - di esperti esterni indipendenti. Periodicamente, insieme ai docenti dell'alunno, è
verificata l'efficacia dell'ausilio medesimo.
Sono pianificati anche gli interventi formativi, tenendo conto dei bisogni emergenti dal territorio e delle
strategie e priorità generali individuate dagli Uffici Scolastici Regionali e dal MIUR.
\subsection*{2.2.6 Risorse economiche} Ogni anno il CTS riceve i fondi dal MIUR per le azioni previste ai punti 2.2.1 e 2.2.2 (informazione e
formazione condotta direttamente dagli operatori e/o esperti), 2.2.3 (acquisti ausili) e per il funzionamento
del CTS (spese di missione, spese per attività di formazione / auto formazione degli operatori). Altre risorse
possono essere messe a disposizione dagli Uffici Scolastici Regionali.
\subsection*{2.2.7 Promozione di intese territoriali per l'inclusione}
I CTS potranno farsi promotori, in rete con le Istituzioni scolastiche, di intese e accordi territoriali con i
servizi sociosanitari del territorio finalizzati all'elaborazione condivisa di procedure per l'integrazione dei
servizi in ambito scolastico, l'utilizzo concordato e condiviso di risorse professionali e/o finanziarie e l'avvio
di progetti finalizzati al miglioramento del livello di inclusività delle scuole e alla prevenzione/contrasto del
disagio in ambito scolastico
\section*{2.3 Regolamento dei CTS}
Ogni CTS si dota di un proprio regolamento in linea con la presente direttiva.
\section*{2.4 Organizzazione interna dei CTS}
\subsection*{2.4.1 Il Dirigente Scolastico}
I CTS sono incardinati in istituzioni scolastiche, pertanto il Dirigente della scuola ha la responsabilità
amministrativa per quanto concerne la gestione e l'organizzazione del Centro. Coerentemente con il suo
profilo professionale il Dirigente ha il compito - possibilmente previa formazione sulle risorse normative,
materiali ed umane in riferimento ai bisogni educativi speciali - di promuovere i rapporti del CTS con il
territorio e di garantirne il miglior funzionamento, l'efficienza e l'efficacia.
\subsection*{2.4.2 Gli Operatori. Equipe di docenti curricolari e di sostegno specializzati}
In ogni CTS dovrebbero essere presenti tre operatori, di cui almeno uno specializzato sui Disturbi Specifici
di Apprendimento, come previsto dall'art. 8 del Decreto 5669/2011. Si porrà attenzione a che le competenze
sulle disabilità siano approfondite ed ampie, dalle disabilità intellettive a quelle sensoriali.
È opportuno individuare gli operatori fra i docenti curricolari e di sostegno, che possono garantire continuità
di servizio, almeno per tre anni consecutivi.
Gli operatori possono essere in servizio nelle scuole sede di CTS o in altre scuole, tuttavia anche in questo
secondo caso deve essere assicurato il regolare funzionamento della struttura.
Gli operatori sono tenuti a partecipare a momenti formativi in presenza (tale formazione viene riconosciuta a
tutti gli effetti come servizio) in occasione di eventi organizzati dagli stessi CTS o di iniziative a carattere
regionale e nazionale rilevanti in tema di inclusione, ma anche on line attraverso il portale nazionale di cui al
punto 2.4.6.
Inoltre, sempre nell'ottica di formare e dare strumenti operativi adeguati alle diverse problematiche nonché
di specializzare i docenti dell'equipe, gli USR provvedono a riservare un adeguato numero di posti per gli
operatori dei CTS nei corsi/master promossi dal MIUR.
Nel momento in cui un operatore formato ed esperto modifichi la sede di servizio e non possa pertanto
svolgere la propria attività nel CTS, verrà sostituito da un altro docente che sarà formato dagli operatori
presenti e da appositi corsi di formazione, anche in modalità e-learning, che saranno resi disponibili dal
MIUR e dagli Uffici Scolastici Regionali. La procedura per la sostituzione degli operatori avviene con le
stesse modalità della selezione del personale comandato. Si istituisce presso ogni Ufficio Scolastico
Regionale una commissione, all'interno della quale devono essere presenti alcuni operatori CTS.
\subsection*{2.4.3 Il Comitato Tecnico Scientifico}
I CTS possono dotarsi di un Comitato Tecnico Scientifico al fine di definire le linee generali di intervento -
nel rispetto delle eventuali priorità assegnate a livello di Ministero e Ufficio Scolastico Regionale - e le
iniziative da realizzare sul territorio a breve e medio termine.
Il Comitato Tecnico Scientifico redige il Piano Annuale di Intervento di cui al punto 2.4.
Fanno parte del Comitato Tecnico Scientifico il Dirigente Scolastico, un rappresentante degli operatori del
CTS, un rappresentante designato dall'U.S.R., e, ove possibile, un rappresentante dei Servizi Sanitari. È
auspicabile che partecipino alle riunioni o facciano parte del Comitato anche i referenti CTI, i rappresentanti
degli Enti Locali, delle Associazioni delle persone con disabilità e dei loro familiari, nonché esperti in
specifiche tematiche connesse con le tecnologie per l'integrazione.
\subsection*{2.4.4 Referente regionale dei CTS}
Per ogni regione gli operatori del CTS individuano un referente rappresentante dei CTS a livello regionale.
Tale rappresentante resta in carica due anni.
I referenti regionali dei CTS, in collaborazione con il referente per la Disabilità/DSA dell'Ufficio Scolastico
Regionale – possibilmente individuato tra personale dirigente e ispettivo - hanno compiti di raccordo,
consulenza e coordinamento delle attività, nonché hanno la funzione di proporre nuove iniziative da attuare a
livello regionale o da presentare al Coordinamento nazionale di cui al punto successivo.
\subsection*{2.4.5 Coordinamento nazionale dei CTS}
Presso la Direzione Generale per lo Studente, l'Integrazione, la Partecipazione e la Comunicazione del
MIUR è costituito il Coordinamento nazionale dei CTS.
Lo scopo di tale organismo è garantire il migliore funzionamento della rete nazionale dei CTS. Esso ha
compiti di consulenza, programmazione e monitoraggio, nel rispetto delle prerogative dell'Amministrazione
centrale e degli Uffici Scolastici Regionali, comunque rappresentati nel Coordinamento stesso.
Fanno parte del Coordinamento nazionale:
-Un rappresentante del MIUR
-I referenti per la Disabilità/DSA degli Uffici Scolastici Regionali
-I referenti regionali CTS
-Un rappresentante del Ministero della Salute
-Un rappresentante del Ministero delle politiche sociali e del lavoro
-Eventuali rappresentanti della FISH e della FAND
-Docenti universitari o esperti nelle tecnologie per l'integrazione.

Il Coordinamento nazionale si rinnova ogni due anni.
Il Comitato tecnico è costituito dal rappresentante del MIUR, che lo presiede, e da una rappresentanza di 4
referenti CTS e 4 referenti per la disabilità/DSA degli Uffici Scolastici Regionali.

\subsection*{2.4.6 Portale}
Viene predisposto un portale come ambiente di apprendimento – insegnamento e scambio di informazioni e
consulenza.
All'interno del portale sono ricompresi i siti Handytecno ed Essediquadro, rispettivamente dedicati agli ausili
ed al servizio di documentazione dei software didattici.
È inoltre presente una mappa completa dei CTS e dei CTI, con eventuali siti ad essi collegati.
Una pagina web è dedicata alle Associazioni delle persone con disabilità e dei loro familiari, completa di
indirizzi e link ai vari siti, oltre ai link diretti alle sezioni del sito MIUR relative a disabilità e DSA.
Infine, sono previste le seguenti aree:
\begin{itemize}
	\item formazione, con percorsi dedicati alle famiglie ed al personale della scuola, dove trovare video lezioni e
	web conference oltre che materiale didattico in formato digitale;
	\item forum per scambi di informazioni tra operatori, famiglie, associazioni, operatori degli altri enti;
	\item News per le novità di tutto il territorio nazionale ed europeo, anche in collaborazione con la European
	Agency for special needs education;
	\item un'Area Riservata per scambi di consulenze, confronti su problematiche, su modalità operative dove
	trovarsi periodicamente.
\end{itemize}
Il portale rispetta i requisiti previsti dalla Legge n. 4/2004 sull'accessibilità dei siti web.

\begin{tabular*}{\textwidth}%
	{@{\extracolsep{\fill}}lc}
Roma, 27 dicembre 2012&IL MINISTRO\\
	&f.to Francesco Profumo
\end{tabular*}