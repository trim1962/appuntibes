N. 00329/2011 REG.PROV.COLL.
N. 00342/2011 REG.RIC.

REPUBBLICA ITALIANA
IN NOME DEL POPOLO ITALIANO
Il Tribunale Amministrativo Regionale per l’Umbria
(Sezione Prima)
ha pronunciato la presente
SENTENZA
ex art. 60 cod. proc. amm.;
sul ricorso numero di registro generale 342 del 2011, proposto da:
xxxxxxxxxxxxxx e xxxxxxxxxxxxxxx, in proprio ed in qualità di genitori del minore xxxxxxxxxxxxxxx, rappresentati e difesi dall’avv. Maria Veronica Laurenzi, con domicilio eletto presso Gerardo Gatti in Perugia, corso Vannucci, 63; 
contro
Ministero dell’Istruzione dell’Università e della Ricerca – Istituto Istruzione Superiore “Pontano Sansi – Leoncillo Leonardi” di Spoleto - Consiglio di Classe della 3 A, a.s. 2010/2011, rappresentati e difesi dall’Avvocatura Distrettuale dello Stato, anche domiciliataria per legge in Perugia, via degli Offici, 14; 
per l’annullamento
- del certificato di studio n. 703 del 5 luglio 2011 a firma del Dirigente Scolastico dell’Istituto di Istruzione Superiore “Sansi-Leonardi” di Spoleto con il quale si determina la non ammissione alla classe successiva di xxxxxxxxxxxxxxx, figlio minore dei ricorrenti, frequentante per la prima volta la classe 3°, corso ordinario, nell’anno scolastico 2010/2011;
- delle operazioni di scrutinio relative alla formazione del predetto giudizio di “non ammissione”;
- di ogni altro atto e/o provvedimento presupposto, connesso o conseguente, ancorché allo stato non conosciuti;

Visti il ricorso e i relativi allegati;
Visto l’atto di costituzione in giudizio delle Amministrazioni scolastiche intimate;
Viste le memorie difensive;
Visti tutti gli atti della causa;
Relatore nella camera di consiglio del giorno 28 settembre 2011 il dott. Pierfrancesco Ungari e uditi per le parti i difensori come specificato nel verbale;
Sentite le stesse parti ai sensi dell’art. 60 cod. proc. amm.;
Ritenuto e considerato in fatto e diritto quanto segue.

FATTO e DIRITTO
1. Viene impugnata la non ammissione alla classe successiva di un alunno che nell’a.s. 2010/2011 ha frequentato la 3^ classe dell’Istituto di Istruzione Superiore “Pontano Sansi - Leoncillo Leonardi” di Spoleto, riportando all’esame di fine anno, utile anche al rilascio del diploma di “Maestro d’Arte”, insufficienze in sette materie.
2. Il ragazzo è affetto dalla nascita da un disturbo specifico dell’apprendimento (c.d. DSA, che ricomprende diversi fenomeni, quali dislessia, discalculia e disgrafia, spesso, così come nel caso in esame, presenti in modo associato), rivalutato annualmente dalla A.U.S.L. n. 3, mediante certificazioni che la famiglia presenta alla scuola all’inizio di ogni anno.
I genitori lamentano che la scuola, in violazione della legge 170/2010 (articoli 2 e 5), non abbia predisposto né realizzato durante l’anno le misure idonee a consentire la realizzazione di un percorso di studio tale da permettere al figlio di superare le proprie difficoltà nella realizzazione degli obiettivi scolastici. In particolare, che la scuola abbia fatto sostenere al figlio un esame su tutte le materie, nei modi e nei tempi degli altri studenti (eccettuati l’utilizzo della calcolatrice e “disorganici aiuti una tantum” da parte di alcuni docenti), senza porre in essere gli accorgimenti didattici volti a rendere per lui sostenibile detta verifica finale.
Lamentano anche di non essere stati resi edotti prima dell’esame della condizione scolastica del figlio, anche in questo caso in violazione della legge 170/2010 (articolo 3) e di non aver quindi potuto adottare le opportune iniziative.
3. Per l’Amministrazione scolastica resiste, controdeducendo puntualmente, l’Avvocatura Distrettuale dello Stato.
4. Giova ricordare che le misure educative e didattiche di supporto per gli studenti con diagnosi di DSA sono previste dall’articolo 5 della legge 170/2010, (specificamente dedicata ai disturbi specifici di apprendimento in ambito scolastico), secondo cui : <<Gli studenti con diagnosi di DSA hanno diritto a fruire di appositi provvedimenti dispensativi e compensativi di flessibilità didattica nel corso dei cicli di istruzione e formazione e negli studi universitari >> (comma 1), <<Agli studenti con DSA le istituzioni scolastiche, a valere sulle risorse specifiche e disponibili a legislazione vigente iscritte nello stato di previsione del Ministero dell’istruzione, dell’università e della ricerca, garantiscono: a) l’uso di una didattica individualizzata e personalizzata, con forme efficaci e flessibili di lavoro scolastico che tengano conto anche di caratteristiche peculiari dei soggetti, quali il bilinguismo, adottando una metodologia e una strategia educativa adeguate; b) l’introduzione di strumenti compensativi, compresi i mezzi di apprendimento alternativi e le tecnologie informatiche, nonché misure dispensative da alcune prestazioni non essenziali ai fini della qualità dei concetti da apprendere; c) per l’insegnamento delle lingue straniere, l’uso di strumenti compensativi che favoriscano la comunicazione verbale e che assicurino ritmi graduali di apprendimento, prevedendo anche, ove risulti utile, la possibilità dell’esonero>> (comma 2), <<Le misure di cui al comma 2 devono essere sottoposte periodicamente a monitoraggio per valutarne l’efficacia e il raggiungimento degli obiettivi>> (comma 3) e <<Agli studenti con DSA sono garantite, durante il percorso di istruzione e di formazione scolastica e universitaria, adeguate forme di verifica e di valutazione, anche per quanto concerne gli esami di Stato e di ammissione all’università nonché gli esami universitari. >> (comma 4).
5. In sintesi, la legge richiama la necessità che la scuola elabori e realizzi, in sede di insegnamento, verifica e valutazione, un percorso formativo personalizzato, che tenga conto delle esigenze e delle potenzialità specifiche di ciascun studente con DSA, ed indica a tal fine “strumenti compensativi” (che si sostanziano nell’introduzione di mezzi di apprendimento alternativi e nell’uso di tecnologie informatiche) e “misure dispensative” (che si sostanziano nella riduzione del programma o nell’esenzione dalle lingue straniere), che spetta ai docenti individuare ed attuare in concreto.
6. Ora, i ricorrenti hanno elencato gli strumenti compensativi - uso del registratore in classe, uso di tabelle della memoria, possibile utilizzo di computer con programmi di video scrittura con correttore ortografico – e le possibili “dispense” - tempi più lunghi per le prove scritte e per lo studio, organizzazione di interrogazioni programmate, assegnazione di compiti a casa in misura ridotta, verifiche prevalentemente orali - che, a loro dire, avrebbero dovuto essere, ma non sono stati adottati.
L’Avvocatura dello Stato ha dato conto che l’Istituto ha predisposto a sostegno del ragazzo “interventi individuali mirati e uno specifico Piano didattico individualizzato”, ed ha elencato (oltre che riassunto in un quadro sinottico) le misure compensative, le modalità di verifica ed i criteri di valutazione, elaborati per ciascuna materia dai docenti tenendo in considerazione le certificazioni della A.U.S.L., il dialogo con la famiglia, la conoscenza dello studente negli anni precedenti, i relativi giudizi osservativi e risultati di apprendimento.
7. Con la memoria finale, i ricorrenti, riconoscendo l’elaborazione del Piano didattico, ribattono che si tratta di un modello non personalizzato, bensì riprodotto di anno in anno per qualunque alunno sia interessato da DSA.
Ad avviso del Collegio, la utilizzazione di una sorta di “modello” di intervento dedicato agli alunni affetti da DSA non comporta di per sé la non attuazione della legge 170/2010, anche tenuto conto che la norma si preoccupa di chiarire che gli interventi previsti sono sì garantiti, ma “a valere sulle risorse specifiche e disponibili a legislazione vigente”, vale a dire nella misura in cui le scuole abbiano le risorse finanziarie, organizzative ed umane sufficienti a realizzarli. E possono considerarsi notorie le difficoltà in cui si dibattono gli istituti scolastici, in questi ultimi anni caratterizzati da una costante riduzione di dette risorse.
Può aggiungersi che la classe frequentata dal ragazzo è composta da altri 9 studenti, dei quali: 1 studente con handicap psico-fisico grave, 1 con handicap psichico grave, 1 con handicap psichico medio, 1 trasferito da altro istituto in corso d’anno con necessità di recupero in quasi tutte le materie, 1 trasferito da altro istituto in corso d’anno e seguito dai servizi sociali, 1 con grave situazione economico-familiare e deprivazione culturale, 2 in condizioni di normalità e 1 studente eccellente. Da tale particolare composizione (non è questa la sede per valutare quanto opportuna, non essendo peraltro tale profilo oggetto di censure), dall’esiguità del numero degli studenti, che consente con i docenti “un lavoro 1 ad 1”, dalla circostanza che le materie di indirizzo previste dal piano di studi (disegno professionale, laboratorio fotografia, disegno dal vero, plastica) coprono circa il 70 % del monte ore totale e sono tali da non potersi svolgere se non in forma individuale in un contesto di natura laboratoriale ed operativo, la scuola fa (plausibilmente) discendere che l’individualizzazione e la personalizzazione dei percorsi formativi ed il sostegno non siano l’eccezione, bensì la regola quotidiana.
Ancora, i ricorrenti sottolineano che, per rendere credibile il piano didattico, non può essere invocata la continuità del corpo docente che ha seguito il ragazzo anche nelle prime due classi del corso, posto che gli insegnanti di italiano e di matematica sono invece stati sostituiti in quest’ultimo anno.
Al riguardo, occorre osservare che per tutte le altre materie i docenti sono rimasti gli stessi dall’inizio del corso.
Non sembra poi possa considerarsi sintomo di inadeguatezza la circostanza che il Piano sia datato 13 ottobre 2010, cioè ad un momento in cui, secondo i ricorrenti, non sarebbe stato possibile per i docenti avere già chiare le misure da intraprendere.
I ricorrenti censurano anche la concreta realizzazione di quanto previsto nel Piano, sostenendo che non sono mai state concordate e programmate le interrogazioni, né assegnati tempi più lunghi per le verifiche e per l’esame, né usati computer con videoscrittura, correttore ortografico, registratore, audiolibri.
L’Amministrazione ha per contro precisato (cfr. verbali dei Consigli di Classe in data 1 febbraio, 19 maggio e 6 giugno 2011) che sono state attuate prove differenziate per tutte le materie previste (italiano, matematica, fisica); che i docenti hanno spiegato il programma utilizzando mappe concettuali, schematizzazioni, esemplificazioni, ripasso di consolidamento; che è stato consentito allo studente l’uso di mappe concettuali ed appunti al momento delle prove scritte e orali; che sono stati concessi tempi più lunghi per le verifiche; che è stata omessa la considerazione degli errori ortografici; che è stato limitato il carico dei compiti a casa.
Anche per quanto concerne l’esame di qualifica, risulta che siano stati concessi (a tutta la classe) tempi più lunghi per lo svolgimento delle prove scritte (6 ore per italiano, 4 per matematica, 6 per laboratorio – peraltro, risulta che il ragazzo sia uscito con largo anticipo rispetto all’orario consentito) e proposte prove più semplici rispetto a quelle effettuate durante l’anno; che siano stato fornito dal docente lo “strumento d’aiuto” (calcolatrice) per la prova di matematica; che non siano stati considerati gli errori ortografici nella prova di italiano, dando nel contempo più importanza alla sostanza che alla forma; che siano stati limitati gli argomenti richiesti durante la prova orale, consentendo l’utilizzo di mappe concettuali e libro aperto.
In sostanza, l’unico intervento (che la stessa scuola riconosce) non attuato, tra quelli di cui i ricorrenti lamentano la mancanza, concerne la registrazione delle lezioni in classe. Sulla rilevanza di tale omissione i ricorrenti insistono, sempre con la memoria finale.
L’Avvocatura risponde sottolineando che si è trattato di una scelta della scuola dettata dall’intenzione di non assecondare la già spiccata negligenza dell’alunno (il quale si presentava a scuola privo spesso dei materiali didattici e dimostrava scarsa propensione allo studio pomeridiano) mediante l’utilizzo di uno strumento “delegante”, che avrebbe potuto essere nocivo per il suo sviluppo cognitivo e lo stimolo delle sue capacità di concentrazione (lo stesso motivo ha condotto a mantenere nei limiti del necessario l’uso del PC e della videoscrittura).
Il Collegio (per quanto sia noto che molti specialisti di settore suggeriscono l’uso del registratore) ritiene di non poter sindacare tale specifica valutazione, che rientra nel novero delle possibili scelte tecnico discrezionali demandate alla scuola.
Più volte, del resto, risulta sottolineato (cfr. verbali del collegio dei docenti in data 9 novembre 2010, 1 febbraio, 30 marzo, 3 maggio, 19 maggio 2011) che le difficoltà di apprendimento e le carenze pregresse accumulate dallo studente non sono state colmate anche a causa del suo scarso impegno nello studio, non soltanto nelle materie “teoriche” ma anche nella maggior parte di quelle di carattere operativo e laboratoriale, nelle quali l’incidenza dei fattori cognitivi è assai minore.
In sintesi, in base agli atti non sembra possibile imputare l’insuccesso scolastico ad omissioni e lacune significative del Piano didattico personalizzato e della sua attuazione.
8. Quanto alla comunicazione con i genitori ricorrenti, risultano trasmessi il pagellino interquadrimestrale (a novembre 2010) e la pagella del primo quadrimestre (a febbraio 2011), ed il pagellino interquadrimestrale del secondo quadrimestre (ad aprile), con relative schede delle carenze registrate e lettera di convocazione presso il coordinatore di classe. La scuola sostiene anche che i ricorrenti avrebbero usufruito di numerosi colloqui con i docenti, e che più volte sarebbero state loro segnalate le assenze del figlio dall’istituto: in totale, 50 giorni, entità cospicua che tuttavia il consiglio di classe ha deciso di non considerare circostanza ostativa all’ammissione all’esame di qualifica. I ricorrenti ribattono che dal libretto delle giustificazioni risultano firmate giustificazioni per soli 10 giorni di assenza (l’ultima delle quali risalente al febbraio 2011), e lamentano quindi che non siano state segnalate le altre; la circostanza sembra effettivamente dimostrare una lacuna nella gestione del rapporto da parte della scuola, ma tale rilievo, da solo, non può evidentemente inficiare il provvedimento impugnato.
9. Il ricorso non può pertanto essere accolto.
10. Sussistono tuttavia giustificati motivi per disporre l’integrale compensazione tra le parti delle spese di giudizio.

P.Q.M.
Il Tribunale Amministrativo Regionale per l’Umbria, definitivamente pronunciando sul ricorso, come in epigrafe proposto, lo respinge.

Spese compensate.

Ordina che la presente sentenza sia eseguita dall’autorità amministrativa.

Così deciso in Perugia nella camera di consiglio del giorno 28 settembre 2011 con l’intervento dei magistrati:
Carlo Luigi Cardoni, Presidente FF
Pierfrancesco Ungari, Consigliere, Estensore
Stefano Fantini, Consigliere

L’ESTENSORE			IL PRESIDENTE

DEPOSITATA IN SEGRETERIA
Il 13/10/2011
IL SEGRETARIO
(Art. 89, co. 3, cod. proc. amm.)
